
% Default to the notebook output style

    


% Inherit from the specified cell style.




    
\documentclass[11pt]{article}

    
    
    \usepackage[T1]{fontenc}
    % Nicer default font (+ math font) than Computer Modern for most use cases
    \usepackage{mathpazo}

    % Basic figure setup, for now with no caption control since it's done
    % automatically by Pandoc (which extracts ![](path) syntax from Markdown).
    \usepackage{graphicx}
    % We will generate all images so they have a width \maxwidth. This means
    % that they will get their normal width if they fit onto the page, but
    % are scaled down if they would overflow the margins.
    \makeatletter
    \def\maxwidth{\ifdim\Gin@nat@width>\linewidth\linewidth
    \else\Gin@nat@width\fi}
    \makeatother
    \let\Oldincludegraphics\includegraphics
    % Set max figure width to be 80% of text width, for now hardcoded.
    \renewcommand{\includegraphics}[1]{\Oldincludegraphics[width=.8\maxwidth]{#1}}
    % Ensure that by default, figures have no caption (until we provide a
    % proper Figure object with a Caption API and a way to capture that
    % in the conversion process - todo).
    \usepackage{caption}
    \DeclareCaptionLabelFormat{nolabel}{}
    \captionsetup{labelformat=nolabel}

    \usepackage{adjustbox} % Used to constrain images to a maximum size 
    \usepackage{xcolor} % Allow colors to be defined
    \usepackage{enumerate} % Needed for markdown enumerations to work
    \usepackage{geometry} % Used to adjust the document margins
    \usepackage{amsmath} % Equations
    \usepackage{amssymb} % Equations
    \usepackage{textcomp} % defines textquotesingle
    % Hack from http://tex.stackexchange.com/a/47451/13684:
    \AtBeginDocument{%
        \def\PYZsq{\textquotesingle}% Upright quotes in Pygmentized code
    }
    \usepackage{upquote} % Upright quotes for verbatim code
    \usepackage{eurosym} % defines \euro
    \usepackage[mathletters]{ucs} % Extended unicode (utf-8) support
    \usepackage[utf8x]{inputenc} % Allow utf-8 characters in the tex document
    \usepackage{fancyvrb} % verbatim replacement that allows latex
    \usepackage{grffile} % extends the file name processing of package graphics 
                         % to support a larger range 
    % The hyperref package gives us a pdf with properly built
    % internal navigation ('pdf bookmarks' for the table of contents,
    % internal cross-reference links, web links for URLs, etc.)
    \usepackage{hyperref}
    \usepackage{longtable} % longtable support required by pandoc >1.10
    \usepackage{booktabs}  % table support for pandoc > 1.12.2
    \usepackage[inline]{enumitem} % IRkernel/repr support (it uses the enumerate* environment)
    \usepackage[normalem]{ulem} % ulem is needed to support strikethroughs (\sout)
                                % normalem makes italics be italics, not underlines
    

    
    
    % Colors for the hyperref package
    \definecolor{urlcolor}{rgb}{0,.145,.698}
    \definecolor{linkcolor}{rgb}{.71,0.21,0.01}
    \definecolor{citecolor}{rgb}{.12,.54,.11}

    % ANSI colors
    \definecolor{ansi-black}{HTML}{3E424D}
    \definecolor{ansi-black-intense}{HTML}{282C36}
    \definecolor{ansi-red}{HTML}{E75C58}
    \definecolor{ansi-red-intense}{HTML}{B22B31}
    \definecolor{ansi-green}{HTML}{00A250}
    \definecolor{ansi-green-intense}{HTML}{007427}
    \definecolor{ansi-yellow}{HTML}{DDB62B}
    \definecolor{ansi-yellow-intense}{HTML}{B27D12}
    \definecolor{ansi-blue}{HTML}{208FFB}
    \definecolor{ansi-blue-intense}{HTML}{0065CA}
    \definecolor{ansi-magenta}{HTML}{D160C4}
    \definecolor{ansi-magenta-intense}{HTML}{A03196}
    \definecolor{ansi-cyan}{HTML}{60C6C8}
    \definecolor{ansi-cyan-intense}{HTML}{258F8F}
    \definecolor{ansi-white}{HTML}{C5C1B4}
    \definecolor{ansi-white-intense}{HTML}{A1A6B2}

    % commands and environments needed by pandoc snippets
    % extracted from the output of `pandoc -s`
    \providecommand{\tightlist}{%
      \setlength{\itemsep}{0pt}\setlength{\parskip}{0pt}}
    \DefineVerbatimEnvironment{Highlighting}{Verbatim}{commandchars=\\\{\}}
    % Add ',fontsize=\small' for more characters per line
    \newenvironment{Shaded}{}{}
    \newcommand{\KeywordTok}[1]{\textcolor[rgb]{0.00,0.44,0.13}{\textbf{{#1}}}}
    \newcommand{\DataTypeTok}[1]{\textcolor[rgb]{0.56,0.13,0.00}{{#1}}}
    \newcommand{\DecValTok}[1]{\textcolor[rgb]{0.25,0.63,0.44}{{#1}}}
    \newcommand{\BaseNTok}[1]{\textcolor[rgb]{0.25,0.63,0.44}{{#1}}}
    \newcommand{\FloatTok}[1]{\textcolor[rgb]{0.25,0.63,0.44}{{#1}}}
    \newcommand{\CharTok}[1]{\textcolor[rgb]{0.25,0.44,0.63}{{#1}}}
    \newcommand{\StringTok}[1]{\textcolor[rgb]{0.25,0.44,0.63}{{#1}}}
    \newcommand{\CommentTok}[1]{\textcolor[rgb]{0.38,0.63,0.69}{\textit{{#1}}}}
    \newcommand{\OtherTok}[1]{\textcolor[rgb]{0.00,0.44,0.13}{{#1}}}
    \newcommand{\AlertTok}[1]{\textcolor[rgb]{1.00,0.00,0.00}{\textbf{{#1}}}}
    \newcommand{\FunctionTok}[1]{\textcolor[rgb]{0.02,0.16,0.49}{{#1}}}
    \newcommand{\RegionMarkerTok}[1]{{#1}}
    \newcommand{\ErrorTok}[1]{\textcolor[rgb]{1.00,0.00,0.00}{\textbf{{#1}}}}
    \newcommand{\NormalTok}[1]{{#1}}
    
    % Additional commands for more recent versions of Pandoc
    \newcommand{\ConstantTok}[1]{\textcolor[rgb]{0.53,0.00,0.00}{{#1}}}
    \newcommand{\SpecialCharTok}[1]{\textcolor[rgb]{0.25,0.44,0.63}{{#1}}}
    \newcommand{\VerbatimStringTok}[1]{\textcolor[rgb]{0.25,0.44,0.63}{{#1}}}
    \newcommand{\SpecialStringTok}[1]{\textcolor[rgb]{0.73,0.40,0.53}{{#1}}}
    \newcommand{\ImportTok}[1]{{#1}}
    \newcommand{\DocumentationTok}[1]{\textcolor[rgb]{0.73,0.13,0.13}{\textit{{#1}}}}
    \newcommand{\AnnotationTok}[1]{\textcolor[rgb]{0.38,0.63,0.69}{\textbf{\textit{{#1}}}}}
    \newcommand{\CommentVarTok}[1]{\textcolor[rgb]{0.38,0.63,0.69}{\textbf{\textit{{#1}}}}}
    \newcommand{\VariableTok}[1]{\textcolor[rgb]{0.10,0.09,0.49}{{#1}}}
    \newcommand{\ControlFlowTok}[1]{\textcolor[rgb]{0.00,0.44,0.13}{\textbf{{#1}}}}
    \newcommand{\OperatorTok}[1]{\textcolor[rgb]{0.40,0.40,0.40}{{#1}}}
    \newcommand{\BuiltInTok}[1]{{#1}}
    \newcommand{\ExtensionTok}[1]{{#1}}
    \newcommand{\PreprocessorTok}[1]{\textcolor[rgb]{0.74,0.48,0.00}{{#1}}}
    \newcommand{\AttributeTok}[1]{\textcolor[rgb]{0.49,0.56,0.16}{{#1}}}
    \newcommand{\InformationTok}[1]{\textcolor[rgb]{0.38,0.63,0.69}{\textbf{\textit{{#1}}}}}
    \newcommand{\WarningTok}[1]{\textcolor[rgb]{0.38,0.63,0.69}{\textbf{\textit{{#1}}}}}
    
    
    % Define a nice break command that doesn't care if a line doesn't already
    % exist.
    \def\br{\hspace*{\fill} \\* }
    % Math Jax compatability definitions
    \def\gt{>}
    \def\lt{<}
    % Document parameters
    \title{jz2977\_hw2}
    
    
    

    % Pygments definitions
    
\makeatletter
\def\PY@reset{\let\PY@it=\relax \let\PY@bf=\relax%
    \let\PY@ul=\relax \let\PY@tc=\relax%
    \let\PY@bc=\relax \let\PY@ff=\relax}
\def\PY@tok#1{\csname PY@tok@#1\endcsname}
\def\PY@toks#1+{\ifx\relax#1\empty\else%
    \PY@tok{#1}\expandafter\PY@toks\fi}
\def\PY@do#1{\PY@bc{\PY@tc{\PY@ul{%
    \PY@it{\PY@bf{\PY@ff{#1}}}}}}}
\def\PY#1#2{\PY@reset\PY@toks#1+\relax+\PY@do{#2}}

\expandafter\def\csname PY@tok@w\endcsname{\def\PY@tc##1{\textcolor[rgb]{0.73,0.73,0.73}{##1}}}
\expandafter\def\csname PY@tok@c\endcsname{\let\PY@it=\textit\def\PY@tc##1{\textcolor[rgb]{0.25,0.50,0.50}{##1}}}
\expandafter\def\csname PY@tok@cp\endcsname{\def\PY@tc##1{\textcolor[rgb]{0.74,0.48,0.00}{##1}}}
\expandafter\def\csname PY@tok@k\endcsname{\let\PY@bf=\textbf\def\PY@tc##1{\textcolor[rgb]{0.00,0.50,0.00}{##1}}}
\expandafter\def\csname PY@tok@kp\endcsname{\def\PY@tc##1{\textcolor[rgb]{0.00,0.50,0.00}{##1}}}
\expandafter\def\csname PY@tok@kt\endcsname{\def\PY@tc##1{\textcolor[rgb]{0.69,0.00,0.25}{##1}}}
\expandafter\def\csname PY@tok@o\endcsname{\def\PY@tc##1{\textcolor[rgb]{0.40,0.40,0.40}{##1}}}
\expandafter\def\csname PY@tok@ow\endcsname{\let\PY@bf=\textbf\def\PY@tc##1{\textcolor[rgb]{0.67,0.13,1.00}{##1}}}
\expandafter\def\csname PY@tok@nb\endcsname{\def\PY@tc##1{\textcolor[rgb]{0.00,0.50,0.00}{##1}}}
\expandafter\def\csname PY@tok@nf\endcsname{\def\PY@tc##1{\textcolor[rgb]{0.00,0.00,1.00}{##1}}}
\expandafter\def\csname PY@tok@nc\endcsname{\let\PY@bf=\textbf\def\PY@tc##1{\textcolor[rgb]{0.00,0.00,1.00}{##1}}}
\expandafter\def\csname PY@tok@nn\endcsname{\let\PY@bf=\textbf\def\PY@tc##1{\textcolor[rgb]{0.00,0.00,1.00}{##1}}}
\expandafter\def\csname PY@tok@ne\endcsname{\let\PY@bf=\textbf\def\PY@tc##1{\textcolor[rgb]{0.82,0.25,0.23}{##1}}}
\expandafter\def\csname PY@tok@nv\endcsname{\def\PY@tc##1{\textcolor[rgb]{0.10,0.09,0.49}{##1}}}
\expandafter\def\csname PY@tok@no\endcsname{\def\PY@tc##1{\textcolor[rgb]{0.53,0.00,0.00}{##1}}}
\expandafter\def\csname PY@tok@nl\endcsname{\def\PY@tc##1{\textcolor[rgb]{0.63,0.63,0.00}{##1}}}
\expandafter\def\csname PY@tok@ni\endcsname{\let\PY@bf=\textbf\def\PY@tc##1{\textcolor[rgb]{0.60,0.60,0.60}{##1}}}
\expandafter\def\csname PY@tok@na\endcsname{\def\PY@tc##1{\textcolor[rgb]{0.49,0.56,0.16}{##1}}}
\expandafter\def\csname PY@tok@nt\endcsname{\let\PY@bf=\textbf\def\PY@tc##1{\textcolor[rgb]{0.00,0.50,0.00}{##1}}}
\expandafter\def\csname PY@tok@nd\endcsname{\def\PY@tc##1{\textcolor[rgb]{0.67,0.13,1.00}{##1}}}
\expandafter\def\csname PY@tok@s\endcsname{\def\PY@tc##1{\textcolor[rgb]{0.73,0.13,0.13}{##1}}}
\expandafter\def\csname PY@tok@sd\endcsname{\let\PY@it=\textit\def\PY@tc##1{\textcolor[rgb]{0.73,0.13,0.13}{##1}}}
\expandafter\def\csname PY@tok@si\endcsname{\let\PY@bf=\textbf\def\PY@tc##1{\textcolor[rgb]{0.73,0.40,0.53}{##1}}}
\expandafter\def\csname PY@tok@se\endcsname{\let\PY@bf=\textbf\def\PY@tc##1{\textcolor[rgb]{0.73,0.40,0.13}{##1}}}
\expandafter\def\csname PY@tok@sr\endcsname{\def\PY@tc##1{\textcolor[rgb]{0.73,0.40,0.53}{##1}}}
\expandafter\def\csname PY@tok@ss\endcsname{\def\PY@tc##1{\textcolor[rgb]{0.10,0.09,0.49}{##1}}}
\expandafter\def\csname PY@tok@sx\endcsname{\def\PY@tc##1{\textcolor[rgb]{0.00,0.50,0.00}{##1}}}
\expandafter\def\csname PY@tok@m\endcsname{\def\PY@tc##1{\textcolor[rgb]{0.40,0.40,0.40}{##1}}}
\expandafter\def\csname PY@tok@gh\endcsname{\let\PY@bf=\textbf\def\PY@tc##1{\textcolor[rgb]{0.00,0.00,0.50}{##1}}}
\expandafter\def\csname PY@tok@gu\endcsname{\let\PY@bf=\textbf\def\PY@tc##1{\textcolor[rgb]{0.50,0.00,0.50}{##1}}}
\expandafter\def\csname PY@tok@gd\endcsname{\def\PY@tc##1{\textcolor[rgb]{0.63,0.00,0.00}{##1}}}
\expandafter\def\csname PY@tok@gi\endcsname{\def\PY@tc##1{\textcolor[rgb]{0.00,0.63,0.00}{##1}}}
\expandafter\def\csname PY@tok@gr\endcsname{\def\PY@tc##1{\textcolor[rgb]{1.00,0.00,0.00}{##1}}}
\expandafter\def\csname PY@tok@ge\endcsname{\let\PY@it=\textit}
\expandafter\def\csname PY@tok@gs\endcsname{\let\PY@bf=\textbf}
\expandafter\def\csname PY@tok@gp\endcsname{\let\PY@bf=\textbf\def\PY@tc##1{\textcolor[rgb]{0.00,0.00,0.50}{##1}}}
\expandafter\def\csname PY@tok@go\endcsname{\def\PY@tc##1{\textcolor[rgb]{0.53,0.53,0.53}{##1}}}
\expandafter\def\csname PY@tok@gt\endcsname{\def\PY@tc##1{\textcolor[rgb]{0.00,0.27,0.87}{##1}}}
\expandafter\def\csname PY@tok@err\endcsname{\def\PY@bc##1{\setlength{\fboxsep}{0pt}\fcolorbox[rgb]{1.00,0.00,0.00}{1,1,1}{\strut ##1}}}
\expandafter\def\csname PY@tok@kc\endcsname{\let\PY@bf=\textbf\def\PY@tc##1{\textcolor[rgb]{0.00,0.50,0.00}{##1}}}
\expandafter\def\csname PY@tok@kd\endcsname{\let\PY@bf=\textbf\def\PY@tc##1{\textcolor[rgb]{0.00,0.50,0.00}{##1}}}
\expandafter\def\csname PY@tok@kn\endcsname{\let\PY@bf=\textbf\def\PY@tc##1{\textcolor[rgb]{0.00,0.50,0.00}{##1}}}
\expandafter\def\csname PY@tok@kr\endcsname{\let\PY@bf=\textbf\def\PY@tc##1{\textcolor[rgb]{0.00,0.50,0.00}{##1}}}
\expandafter\def\csname PY@tok@bp\endcsname{\def\PY@tc##1{\textcolor[rgb]{0.00,0.50,0.00}{##1}}}
\expandafter\def\csname PY@tok@fm\endcsname{\def\PY@tc##1{\textcolor[rgb]{0.00,0.00,1.00}{##1}}}
\expandafter\def\csname PY@tok@vc\endcsname{\def\PY@tc##1{\textcolor[rgb]{0.10,0.09,0.49}{##1}}}
\expandafter\def\csname PY@tok@vg\endcsname{\def\PY@tc##1{\textcolor[rgb]{0.10,0.09,0.49}{##1}}}
\expandafter\def\csname PY@tok@vi\endcsname{\def\PY@tc##1{\textcolor[rgb]{0.10,0.09,0.49}{##1}}}
\expandafter\def\csname PY@tok@vm\endcsname{\def\PY@tc##1{\textcolor[rgb]{0.10,0.09,0.49}{##1}}}
\expandafter\def\csname PY@tok@sa\endcsname{\def\PY@tc##1{\textcolor[rgb]{0.73,0.13,0.13}{##1}}}
\expandafter\def\csname PY@tok@sb\endcsname{\def\PY@tc##1{\textcolor[rgb]{0.73,0.13,0.13}{##1}}}
\expandafter\def\csname PY@tok@sc\endcsname{\def\PY@tc##1{\textcolor[rgb]{0.73,0.13,0.13}{##1}}}
\expandafter\def\csname PY@tok@dl\endcsname{\def\PY@tc##1{\textcolor[rgb]{0.73,0.13,0.13}{##1}}}
\expandafter\def\csname PY@tok@s2\endcsname{\def\PY@tc##1{\textcolor[rgb]{0.73,0.13,0.13}{##1}}}
\expandafter\def\csname PY@tok@sh\endcsname{\def\PY@tc##1{\textcolor[rgb]{0.73,0.13,0.13}{##1}}}
\expandafter\def\csname PY@tok@s1\endcsname{\def\PY@tc##1{\textcolor[rgb]{0.73,0.13,0.13}{##1}}}
\expandafter\def\csname PY@tok@mb\endcsname{\def\PY@tc##1{\textcolor[rgb]{0.40,0.40,0.40}{##1}}}
\expandafter\def\csname PY@tok@mf\endcsname{\def\PY@tc##1{\textcolor[rgb]{0.40,0.40,0.40}{##1}}}
\expandafter\def\csname PY@tok@mh\endcsname{\def\PY@tc##1{\textcolor[rgb]{0.40,0.40,0.40}{##1}}}
\expandafter\def\csname PY@tok@mi\endcsname{\def\PY@tc##1{\textcolor[rgb]{0.40,0.40,0.40}{##1}}}
\expandafter\def\csname PY@tok@il\endcsname{\def\PY@tc##1{\textcolor[rgb]{0.40,0.40,0.40}{##1}}}
\expandafter\def\csname PY@tok@mo\endcsname{\def\PY@tc##1{\textcolor[rgb]{0.40,0.40,0.40}{##1}}}
\expandafter\def\csname PY@tok@ch\endcsname{\let\PY@it=\textit\def\PY@tc##1{\textcolor[rgb]{0.25,0.50,0.50}{##1}}}
\expandafter\def\csname PY@tok@cm\endcsname{\let\PY@it=\textit\def\PY@tc##1{\textcolor[rgb]{0.25,0.50,0.50}{##1}}}
\expandafter\def\csname PY@tok@cpf\endcsname{\let\PY@it=\textit\def\PY@tc##1{\textcolor[rgb]{0.25,0.50,0.50}{##1}}}
\expandafter\def\csname PY@tok@c1\endcsname{\let\PY@it=\textit\def\PY@tc##1{\textcolor[rgb]{0.25,0.50,0.50}{##1}}}
\expandafter\def\csname PY@tok@cs\endcsname{\let\PY@it=\textit\def\PY@tc##1{\textcolor[rgb]{0.25,0.50,0.50}{##1}}}

\def\PYZbs{\char`\\}
\def\PYZus{\char`\_}
\def\PYZob{\char`\{}
\def\PYZcb{\char`\}}
\def\PYZca{\char`\^}
\def\PYZam{\char`\&}
\def\PYZlt{\char`\<}
\def\PYZgt{\char`\>}
\def\PYZsh{\char`\#}
\def\PYZpc{\char`\%}
\def\PYZdl{\char`\$}
\def\PYZhy{\char`\-}
\def\PYZsq{\char`\'}
\def\PYZdq{\char`\"}
\def\PYZti{\char`\~}
% for compatibility with earlier versions
\def\PYZat{@}
\def\PYZlb{[}
\def\PYZrb{]}
\makeatother


    % Exact colors from NB
    \definecolor{incolor}{rgb}{0.0, 0.0, 0.5}
    \definecolor{outcolor}{rgb}{0.545, 0.0, 0.0}



    
    % Prevent overflowing lines due to hard-to-break entities
    \sloppy 
    % Setup hyperref package
    \hypersetup{
      breaklinks=true,  % so long urls are correctly broken across lines
      colorlinks=true,
      urlcolor=urlcolor,
      linkcolor=linkcolor,
      citecolor=citecolor,
      }
    % Slightly bigger margins than the latex defaults
    
    \geometry{verbose,tmargin=1in,bmargin=1in,lmargin=1in,rmargin=1in}
    
    

    \begin{document}
    
    
    \maketitle
    
    

    
    \section{COMS 4731 Computer Vision -\/- Homework
2}\label{coms-4731-computer-vision----homework-2}

\begin{itemize}
\item
  This homework contains the following components:

  \begin{itemize}
  \tightlist
  \item
    \textbf{Problem 1: Image Denoising (40 points)}

    \begin{itemize}
    \tightlist
    \item
      Implement a mean filter using "for" loop.
    \item
      Implement the \texttt{convolve\_image} function.
    \item
      Implement a mean filter using a filter matrix.
    \item
      Implement a Gaussian filter.
    \end{itemize}
  \item
    \textbf{Problem 2: Edge Detection (30 points)}

    \begin{itemize}
    \tightlist
    \item
      Implement a delta filter.
    \item
      Implement a Laplacian filter.
    \end{itemize}
  \item
    \textbf{Problem 3: Hybrid Images (30 points)}

    \begin{itemize}
    \tightlist
    \item
      Fourier transform.
    \item
      Implement low and high pass filters and apply them to images.
    \item
      Create a hybrid image using high-pass and low-pass fitlered
      images.
    \end{itemize}
  \end{itemize}
\item
  Your job is to implement the sections marked with \texttt{TODO} to
  complete the tasks.
\item
  Submission.

  \begin{itemize}
  \tightlist
  \item
    Please submit the notebook (ipynb and pdf) including the output of
    all cells.
  \end{itemize}
\end{itemize}

    \section{Problem 1: Image Denoising}\label{problem-1-image-denoising}

Taking pictures at night is challenging because there is less light that
hits the film or camera sensor. To still capture an image in low light,
we need to change our camera settings to capture more light. One way is
to increase the exposure time, but if there is motion in the scene, this
leads to blur. Another way is to use sensitive film that still responds
to low intensity light. However, the trade-off is that this higher
sensitivity increases the amount of noise captured, which often shows up
as grain on photos. In this problem, your task is to clean up the noise
with signal processing.

    \subsection{Visualizing the Grain}\label{visualizing-the-grain}

To start off, let's load up the image and visualize the image we want to
denoise.

    \begin{Verbatim}[commandchars=\\\{\}]
{\color{incolor}In [{\color{incolor}1}]:} \PY{k+kn}{import} \PY{n+nn}{numpy} \PY{k}{as} \PY{n+nn}{np}
        \PY{k+kn}{import} \PY{n+nn}{matplotlib}\PY{n+nn}{.}\PY{n+nn}{pyplot} \PY{k}{as} \PY{n+nn}{plt}
        \PY{k+kn}{from} \PY{n+nn}{PIL} \PY{k}{import} \PY{n}{Image}
        \PY{k+kn}{from} \PY{n+nn}{IPython} \PY{k}{import} \PY{n}{display}
        \PY{k+kn}{from} \PY{n+nn}{scipy}\PY{n+nn}{.}\PY{n+nn}{signal} \PY{k}{import} \PY{n}{convolve2d}
        \PY{k+kn}{from} \PY{n+nn}{math} \PY{k}{import} \PY{o}{*}
        \PY{k+kn}{import} \PY{n+nn}{time}
        \PY{o}{\PYZpc{}}\PY{k}{matplotlib} inline
        
        \PY{n}{plt}\PY{o}{.}\PY{n}{rcParams}\PY{p}{[}\PY{l+s+s1}{\PYZsq{}}\PY{l+s+s1}{figure.figsize}\PY{l+s+s1}{\PYZsq{}}\PY{p}{]} \PY{o}{=} \PY{p}{[}\PY{l+m+mi}{7}\PY{p}{,} \PY{l+m+mi}{7}\PY{p}{]}
        
        \PY{k}{def} \PY{n+nf}{load\PYZus{}image}\PY{p}{(}\PY{n}{filename}\PY{p}{)}\PY{p}{:}
            \PY{n}{img} \PY{o}{=} \PY{n}{np}\PY{o}{.}\PY{n}{asarray}\PY{p}{(}\PY{n}{Image}\PY{o}{.}\PY{n}{open}\PY{p}{(}\PY{n}{filename}\PY{p}{)}\PY{p}{)}
            \PY{n}{img} \PY{o}{=} \PY{n}{img}\PY{o}{.}\PY{n}{astype}\PY{p}{(}\PY{l+s+s2}{\PYZdq{}}\PY{l+s+s2}{float32}\PY{l+s+s2}{\PYZdq{}}\PY{p}{)} \PY{o}{/} \PY{l+m+mf}{255.}
            \PY{k}{return} \PY{n}{img}
        
        \PY{k}{def} \PY{n+nf}{gray2rgb}\PY{p}{(}\PY{n}{image}\PY{p}{)}\PY{p}{:}
            \PY{k}{return} \PY{n}{np}\PY{o}{.}\PY{n}{repeat}\PY{p}{(}\PY{n}{np}\PY{o}{.}\PY{n}{expand\PYZus{}dims}\PY{p}{(}\PY{n}{image}\PY{p}{,} \PY{l+m+mi}{2}\PY{p}{)}\PY{p}{,} \PY{l+m+mi}{3}\PY{p}{,} \PY{n}{axis}\PY{o}{=}\PY{l+m+mi}{2}\PY{p}{)}
        
        \PY{k}{def} \PY{n+nf}{show\PYZus{}image}\PY{p}{(}\PY{n}{img}\PY{p}{)}\PY{p}{:}
            \PY{k}{if} \PY{n+nb}{len}\PY{p}{(}\PY{n}{img}\PY{o}{.}\PY{n}{shape}\PY{p}{)} \PY{o}{==} \PY{l+m+mi}{2}\PY{p}{:}
                \PY{n}{img} \PY{o}{=} \PY{n}{gray2rgb}\PY{p}{(}\PY{n}{img}\PY{p}{)}
            \PY{n}{plt}\PY{o}{.}\PY{n}{imshow}\PY{p}{(}\PY{n}{img}\PY{p}{,} \PY{n}{interpolation}\PY{o}{=}\PY{l+s+s1}{\PYZsq{}}\PY{l+s+s1}{nearest}\PY{l+s+s1}{\PYZsq{}}\PY{p}{)}
        
        \PY{c+c1}{\PYZsh{} load the image}
        \PY{n}{im} \PY{o}{=} \PY{n}{load\PYZus{}image}\PY{p}{(}\PY{l+s+s1}{\PYZsq{}}\PY{l+s+s1}{noisy\PYZus{}image.jpg}\PY{l+s+s1}{\PYZsq{}}\PY{p}{)}
        \PY{n}{im} \PY{o}{=} \PY{n}{im}\PY{o}{.}\PY{n}{mean}\PY{p}{(}\PY{n}{axis}\PY{o}{=}\PY{l+m+mi}{2}\PY{p}{)} \PY{c+c1}{\PYZsh{} convert to grayscale}
        \PY{n}{show\PYZus{}image}\PY{p}{(}\PY{n}{im}\PY{p}{)}
\end{Verbatim}


    \begin{center}
    \adjustimage{max size={0.9\linewidth}{0.9\paperheight}}{output_3_0.png}
    \end{center}
    { \hspace*{\fill} \\}
    
    \subsection{Mean Filter using "for"
loop}\label{mean-filter-using-for-loop}

Let's try to remove this grain with a mean filter. For every pixel in
the image, we want to take an average (mean) of the neighboring
pictures. Implement this operation using "for" loops and visualize the
result:

    \begin{Verbatim}[commandchars=\\\{\}]
{\color{incolor}In [{\color{incolor}2}]:} \PY{n}{im\PYZus{}pad} \PY{o}{=} \PY{n}{np}\PY{o}{.}\PY{n}{pad}\PY{p}{(}\PY{n}{im}\PY{p}{,} \PY{l+m+mi}{5}\PY{p}{,} \PY{n}{mode}\PY{o}{=}\PY{l+s+s1}{\PYZsq{}}\PY{l+s+s1}{constant}\PY{l+s+s1}{\PYZsq{}}\PY{p}{)}  \PY{c+c1}{\PYZsh{} pad the border of the original image}
        \PY{n}{im\PYZus{}out} \PY{o}{=} \PY{n}{np}\PY{o}{.}\PY{n}{zeros\PYZus{}like}\PY{p}{(}\PY{n}{im}\PY{p}{)}  \PY{c+c1}{\PYZsh{} initialize the output image array}
        
        \PY{l+s+sd}{\PYZsq{}\PYZsq{}\PYZsq{} TODO: Implement a mean filter using \PYZdq{}for\PYZdq{} loop here (modify the im\PYZus{}out matrix). \PYZsq{}\PYZsq{}\PYZsq{}}
        \PY{k}{for} \PY{n}{i} \PY{o+ow}{in} \PY{n+nb}{range}\PY{p}{(}\PY{n}{im\PYZus{}out}\PY{o}{.}\PY{n}{shape}\PY{p}{[}\PY{l+m+mi}{0}\PY{p}{]}\PY{p}{)}\PY{p}{:}
            \PY{k}{for} \PY{n}{j} \PY{o+ow}{in} \PY{n+nb}{range}\PY{p}{(}\PY{n}{im\PYZus{}out}\PY{o}{.}\PY{n}{shape}\PY{p}{[}\PY{l+m+mi}{1}\PY{p}{]}\PY{p}{)}\PY{p}{:}
                \PY{n}{im\PYZus{}out}\PY{p}{[}\PY{n}{i}\PY{p}{,}\PY{n}{j}\PY{p}{]} \PY{o}{=} \PY{n}{np}\PY{o}{.}\PY{n}{sum}\PY{p}{(}\PY{n}{im\PYZus{}pad}\PY{p}{[}\PY{n}{i}\PY{o}{+}\PY{l+m+mi}{4}\PY{p}{:}\PY{n}{i}\PY{o}{+}\PY{l+m+mi}{7}\PY{p}{,}\PY{n}{j}\PY{o}{+}\PY{l+m+mi}{4}\PY{p}{:}\PY{n}{j}\PY{o}{+}\PY{l+m+mi}{7}\PY{p}{]}\PY{p}{)} \PY{o}{/} \PY{l+m+mi}{9}
        
        \PY{n}{show\PYZus{}image}\PY{p}{(}\PY{n}{im\PYZus{}out}\PY{p}{)}
\end{Verbatim}


    \begin{center}
    \adjustimage{max size={0.9\linewidth}{0.9\paperheight}}{output_5_0.png}
    \end{center}
    { \hspace*{\fill} \\}
    
    \subsection{\texorpdfstring{Implement the \texttt{convolve\_image}
function.}{Implement the convolve\_image function.}}\label{implement-the-convolve_image-function.}

In practice, applying filters to images can be more efficient by using
convolution, which is a function that takes as input the raw image and a
filter matrix, and outputs the convolved image. Implement your
\texttt{convolve\_image} function below.

    \begin{Verbatim}[commandchars=\\\{\}]
{\color{incolor}In [{\color{incolor}3}]:} \PY{k}{def} \PY{n+nf}{convolve\PYZus{}image}\PY{p}{(}\PY{n}{image}\PY{p}{,} \PY{n}{filter\PYZus{}matrix}\PY{p}{)}\PY{p}{:}
            \PY{l+s+sd}{\PYZsq{}\PYZsq{}\PYZsq{} Convolve a 2D image using the filter matrix.}
        \PY{l+s+sd}{    Args:}
        \PY{l+s+sd}{        image: a 2D numpy array.}
        \PY{l+s+sd}{        filter\PYZus{}matrix: a 2D numpy array.}
        \PY{l+s+sd}{    Returns:}
        \PY{l+s+sd}{        the convolved image, which is a 2D numpy array same size as the input image.}
        \PY{l+s+sd}{        }
        \PY{l+s+sd}{    TODO: Implement the convolve\PYZus{}image function here. }
        \PY{l+s+sd}{    \PYZsq{}\PYZsq{}\PYZsq{}}
            
            \PY{n}{kernel\PYZus{}h}\PY{p}{,} \PY{n}{kernel\PYZus{}w} \PY{o}{=} \PY{n}{filter\PYZus{}matrix}\PY{o}{.}\PY{n}{shape}\PY{p}{[}\PY{l+m+mi}{0}\PY{p}{]}\PY{p}{,} \PY{n}{filter\PYZus{}matrix}\PY{o}{.}\PY{n}{shape}\PY{p}{[}\PY{l+m+mi}{1}\PY{p}{]}
            
            \PY{k}{if} \PY{n}{kernel\PYZus{}h} \PY{o}{!=} \PY{n}{kernel\PYZus{}w}\PY{p}{:}
                \PY{k}{raise} \PY{n+ne}{ValueError}\PY{p}{(}\PY{l+s+s1}{\PYZsq{}}\PY{l+s+s1}{Inconsistant kernel size!}\PY{l+s+s1}{\PYZsq{}}\PY{p}{)}
            
            \PY{k}{if} \PY{o+ow}{not} \PY{p}{(}\PY{n}{kernel\PYZus{}h} \PY{o}{\PYZam{}} \PY{l+m+mi}{1}\PY{p}{)}\PY{p}{:}
                \PY{k}{raise} \PY{n+ne}{ValueError}\PY{p}{(}\PY{l+s+s1}{\PYZsq{}}\PY{l+s+s1}{kernel size should be odd numbers!}\PY{l+s+s1}{\PYZsq{}}\PY{p}{)}
            
            \PY{n}{pad\PYZus{}size} \PY{o}{=} \PY{n}{kernel\PYZus{}h} \PY{o}{/}\PY{o}{/} \PY{l+m+mi}{2}
            \PY{n}{im\PYZus{}pad} \PY{o}{=} \PY{n}{np}\PY{o}{.}\PY{n}{pad}\PY{p}{(}\PY{n}{image}\PY{p}{,} \PY{n}{pad\PYZus{}size}\PY{p}{,} \PY{n}{mode}\PY{o}{=}\PY{l+s+s1}{\PYZsq{}}\PY{l+s+s1}{constant}\PY{l+s+s1}{\PYZsq{}}\PY{p}{)}
            \PY{n}{im\PYZus{}out} \PY{o}{=} \PY{n}{np}\PY{o}{.}\PY{n}{zeros\PYZus{}like}\PY{p}{(}\PY{n}{image}\PY{p}{)}
            \PY{k}{for} \PY{n}{i} \PY{o+ow}{in} \PY{n+nb}{range}\PY{p}{(}\PY{n}{im\PYZus{}out}\PY{o}{.}\PY{n}{shape}\PY{p}{[}\PY{l+m+mi}{0}\PY{p}{]}\PY{p}{)}\PY{p}{:}
                \PY{k}{for} \PY{n}{j} \PY{o+ow}{in} \PY{n+nb}{range}\PY{p}{(}\PY{n}{im\PYZus{}out}\PY{o}{.}\PY{n}{shape}\PY{p}{[}\PY{l+m+mi}{1}\PY{p}{]}\PY{p}{)}\PY{p}{:}
                    \PY{n}{im\PYZus{}out}\PY{p}{[}\PY{n}{i}\PY{p}{,}\PY{n}{j}\PY{p}{]} \PY{o}{=} \PY{n}{np}\PY{o}{.}\PY{n}{sum}\PY{p}{(}\PY{n}{filter\PYZus{}matrix} \PY{o}{*} \PY{n}{im\PYZus{}pad}\PY{p}{[}\PY{l+m+mi}{1}\PY{o}{+}\PY{n}{i}\PY{o}{\PYZhy{}}\PY{n}{pad\PYZus{}size}\PY{p}{:}\PY{l+m+mi}{2}\PY{o}{+}\PY{n}{i}\PY{o}{+}\PY{n}{pad\PYZus{}size}\PY{p}{,}\PY{l+m+mi}{1}\PY{o}{+}\PY{n}{j}\PY{o}{\PYZhy{}}\PY{n}{pad\PYZus{}size}\PY{p}{:}\PY{l+m+mi}{2}\PY{o}{+}\PY{n}{j}\PY{o}{+}\PY{n}{pad\PYZus{}size}\PY{p}{]}\PY{p}{)}
                    
            \PY{k}{return} \PY{n}{im\PYZus{}out}
\end{Verbatim}


    \subsection{Mean Filter with
Convolution}\label{mean-filter-with-convolution}

Implement this same operation with a convolution instead. Fill in the
mean filter matrix here, and visualize the convolution result.

    \begin{Verbatim}[commandchars=\\\{\}]
{\color{incolor}In [{\color{incolor}4}]:} \PY{l+s+sd}{\PYZsq{}\PYZsq{}\PYZsq{} TODO: Create a mean filter matrix here. \PYZsq{}\PYZsq{}\PYZsq{}}
        \PY{n}{mean\PYZus{}filt} \PY{o}{=} \PY{l+m+mi}{1}\PY{o}{/}\PY{l+m+mi}{9} \PY{o}{*} \PY{n}{np}\PY{o}{.}\PY{n}{ones}\PY{p}{(}\PY{p}{(}\PY{l+m+mi}{3}\PY{p}{,}\PY{l+m+mi}{3}\PY{p}{)}\PY{p}{)} 
\end{Verbatim}


    Apply mean filter convolution using your \texttt{convolve\_image}
function and the \texttt{mean\_filt} matrix.

    \begin{Verbatim}[commandchars=\\\{\}]
{\color{incolor}In [{\color{incolor}5}]:} \PY{n}{show\PYZus{}image}\PY{p}{(}\PY{n}{convolve\PYZus{}image}\PY{p}{(}\PY{n}{im}\PY{p}{,} \PY{n}{mean\PYZus{}filt}\PY{p}{)}\PY{p}{)}
\end{Verbatim}


    \begin{center}
    \adjustimage{max size={0.9\linewidth}{0.9\paperheight}}{output_11_0.png}
    \end{center}
    { \hspace*{\fill} \\}
    
    Compare your convolution result with the
\texttt{scipy.signal.convolve2d} function (they should look the same).

    \begin{Verbatim}[commandchars=\\\{\}]
{\color{incolor}In [{\color{incolor}6}]:} \PY{n}{show\PYZus{}image}\PY{p}{(}\PY{n}{convolve2d}\PY{p}{(}\PY{n}{im}\PY{p}{,} \PY{n}{mean\PYZus{}filt}\PY{p}{)}\PY{p}{)}
\end{Verbatim}


    \begin{center}
    \adjustimage{max size={0.9\linewidth}{0.9\paperheight}}{output_13_0.png}
    \end{center}
    { \hspace*{\fill} \\}
    
    Note: In the sections below, we will use the
\texttt{scipy.signal.convolve2d} function for grading. But fill free to
test your \texttt{convolve\_image} function on other filters as well.

    \subsection{Gaussian Filter}\label{gaussian-filter}

Instead of using a mean filter, let's use a Gaussian filter. Create a 2D
Gaussian filter, and plot the result of the convolution.

Hint: You can first construct a one dimensional Gaussian, then use it to
create a 2D dimensional Gaussian.

    \begin{Verbatim}[commandchars=\\\{\}]
{\color{incolor}In [{\color{incolor}7}]:} \PY{k}{def} \PY{n+nf}{gaussian\PYZus{}filter}\PY{p}{(}\PY{n}{sigma}\PY{p}{,} \PY{n}{k}\PY{o}{=}\PY{l+m+mi}{20}\PY{p}{)}\PY{p}{:}
            \PY{l+s+sd}{\PYZsq{}\PYZsq{}\PYZsq{} }
        \PY{l+s+sd}{    Args:}
        \PY{l+s+sd}{        sigma: the standard deviation of Gaussian kernel.}
        \PY{l+s+sd}{        k: controls size of the filter matrix. }
        \PY{l+s+sd}{    Returns:}
        \PY{l+s+sd}{        a 2D Gaussian filter matrix of the size (2k+1, 2k+1).}
        \PY{l+s+sd}{        }
        \PY{l+s+sd}{    TODO: Implement a Gaussian filter here. }
        \PY{l+s+sd}{    \PYZsq{}\PYZsq{}\PYZsq{}} 
            \PY{n}{g\PYZus{}filter} \PY{o}{=} \PY{n}{np}\PY{o}{.}\PY{n}{zeros}\PY{p}{(}\PY{p}{(}\PY{l+m+mi}{2}\PY{o}{*}\PY{n}{k}\PY{o}{+}\PY{l+m+mi}{1}\PY{p}{,} \PY{l+m+mi}{2}\PY{o}{*}\PY{n}{k}\PY{o}{+}\PY{l+m+mi}{1}\PY{p}{)}\PY{p}{)}
            
            \PY{k}{for} \PY{n}{i} \PY{o+ow}{in} \PY{n+nb}{range}\PY{p}{(}\PY{n}{k}\PY{o}{+}\PY{l+m+mi}{1}\PY{p}{)}\PY{p}{:}
                \PY{k}{for} \PY{n}{j} \PY{o+ow}{in} \PY{n+nb}{range}\PY{p}{(}\PY{n}{i}\PY{p}{,}\PY{n}{k}\PY{o}{+}\PY{l+m+mi}{1}\PY{p}{)}\PY{p}{:}
                    \PY{n}{g\PYZus{}filter}\PY{p}{[}\PY{n}{k}\PY{o}{+}\PY{n}{i}\PY{p}{,} \PY{n}{k}\PY{o}{+}\PY{n}{j}\PY{p}{]} \PY{o}{=} \PY{l+m+mi}{1} \PY{o}{/} \PY{p}{(}\PY{l+m+mi}{2}\PY{o}{*}\PY{n}{np}\PY{o}{.}\PY{n}{pi}\PY{o}{*}\PY{n}{sigma}\PY{o}{*}\PY{n}{sigma}\PY{p}{)} \PY{o}{*} \PY{n}{np}\PY{o}{.}\PY{n}{exp}\PY{p}{(}\PY{o}{\PYZhy{}}\PY{p}{(}\PY{n}{i}\PY{o}{*}\PY{n}{i} \PY{o}{+} \PY{n}{j}\PY{o}{*}\PY{n}{j}\PY{p}{)} \PY{o}{/} \PY{p}{(}\PY{l+m+mi}{2}\PY{o}{*}\PY{n}{sigma}\PY{o}{*}\PY{n}{sigma}\PY{p}{)}\PY{p}{)}
                    \PY{n}{g\PYZus{}filter}\PY{p}{[}\PY{n}{k}\PY{o}{\PYZhy{}}\PY{n}{i}\PY{p}{,} \PY{n}{k}\PY{o}{+}\PY{n}{j}\PY{p}{]} \PY{o}{=} \PY{n}{g\PYZus{}filter}\PY{p}{[}\PY{n}{k}\PY{o}{+}\PY{n}{i}\PY{p}{,} \PY{n}{k}\PY{o}{+}\PY{n}{j}\PY{p}{]}
                    \PY{n}{g\PYZus{}filter}\PY{p}{[}\PY{n}{k}\PY{o}{\PYZhy{}}\PY{n}{i}\PY{p}{,} \PY{n}{k}\PY{o}{\PYZhy{}}\PY{n}{j}\PY{p}{]} \PY{o}{=} \PY{n}{g\PYZus{}filter}\PY{p}{[}\PY{n}{k}\PY{o}{+}\PY{n}{i}\PY{p}{,} \PY{n}{k}\PY{o}{+}\PY{n}{j}\PY{p}{]}
                    \PY{n}{g\PYZus{}filter}\PY{p}{[}\PY{n}{k}\PY{o}{+}\PY{n}{i}\PY{p}{,} \PY{n}{k}\PY{o}{\PYZhy{}}\PY{n}{j}\PY{p}{]} \PY{o}{=} \PY{n}{g\PYZus{}filter}\PY{p}{[}\PY{n}{k}\PY{o}{+}\PY{n}{i}\PY{p}{,} \PY{n}{k}\PY{o}{+}\PY{n}{j}\PY{p}{]}
                    
                    \PY{n}{g\PYZus{}filter}\PY{p}{[}\PY{n}{k}\PY{o}{+}\PY{n}{j}\PY{p}{,} \PY{n}{k}\PY{o}{+}\PY{n}{i}\PY{p}{]} \PY{o}{=} \PY{n}{g\PYZus{}filter}\PY{p}{[}\PY{n}{k}\PY{o}{+}\PY{n}{i}\PY{p}{,} \PY{n}{k}\PY{o}{+}\PY{n}{j}\PY{p}{]}
                    \PY{n}{g\PYZus{}filter}\PY{p}{[}\PY{n}{k}\PY{o}{\PYZhy{}}\PY{n}{j}\PY{p}{,} \PY{n}{k}\PY{o}{+}\PY{n}{i}\PY{p}{]} \PY{o}{=} \PY{n}{g\PYZus{}filter}\PY{p}{[}\PY{n}{k}\PY{o}{+}\PY{n}{i}\PY{p}{,} \PY{n}{k}\PY{o}{+}\PY{n}{j}\PY{p}{]}
                    \PY{n}{g\PYZus{}filter}\PY{p}{[}\PY{n}{k}\PY{o}{\PYZhy{}}\PY{n}{j}\PY{p}{,} \PY{n}{k}\PY{o}{\PYZhy{}}\PY{n}{i}\PY{p}{]} \PY{o}{=} \PY{n}{g\PYZus{}filter}\PY{p}{[}\PY{n}{k}\PY{o}{+}\PY{n}{i}\PY{p}{,} \PY{n}{k}\PY{o}{+}\PY{n}{j}\PY{p}{]}
                    \PY{n}{g\PYZus{}filter}\PY{p}{[}\PY{n}{k}\PY{o}{+}\PY{n}{j}\PY{p}{,} \PY{n}{k}\PY{o}{\PYZhy{}}\PY{n}{i}\PY{p}{]} \PY{o}{=} \PY{n}{g\PYZus{}filter}\PY{p}{[}\PY{n}{k}\PY{o}{+}\PY{n}{i}\PY{p}{,} \PY{n}{k}\PY{o}{+}\PY{n}{j}\PY{p}{]}
        
        
        
                    
            \PY{n}{g\PYZus{}filter} \PY{o}{=} \PY{n}{g\PYZus{}filter} \PY{o}{/} \PY{n}{np}\PY{o}{.}\PY{n}{sum}\PY{p}{(}\PY{n}{g\PYZus{}filter}\PY{p}{)}
            
            \PY{k}{return} \PY{n}{g\PYZus{}filter}
            
        \PY{n}{show\PYZus{}image}\PY{p}{(}\PY{n}{convolve2d}\PY{p}{(}\PY{n}{im}\PY{p}{,} \PY{n}{gaussian\PYZus{}filter}\PY{p}{(}\PY{l+m+mi}{2}\PY{p}{)}\PY{p}{)}\PY{p}{)}
\end{Verbatim}


    \begin{center}
    \adjustimage{max size={0.9\linewidth}{0.9\paperheight}}{output_16_0.png}
    \end{center}
    { \hspace*{\fill} \\}
    
    The amount the image is blurred changes depending on the sigma
parameter. Change the sigma parameter to see what happens. Try a few
different values.

    \begin{Verbatim}[commandchars=\\\{\}]
{\color{incolor}In [{\color{incolor}8}]:} \PY{n}{show\PYZus{}image}\PY{p}{(}\PY{n}{convolve2d}\PY{p}{(}\PY{n}{im}\PY{p}{,} \PY{n}{gaussian\PYZus{}filter}\PY{p}{(}\PY{l+m+mi}{5}\PY{p}{)}\PY{p}{)}\PY{p}{)}
\end{Verbatim}


    \begin{center}
    \adjustimage{max size={0.9\linewidth}{0.9\paperheight}}{output_18_0.png}
    \end{center}
    { \hspace*{\fill} \\}
    
    \subsection{Visualizing Gaussian
Filter}\label{visualizing-gaussian-filter}

Try changing the sigma parameter below to visualize the Gaussian filter
directly. This gives you an idea of how different sigma values create
different convolved images.

    \begin{Verbatim}[commandchars=\\\{\}]
{\color{incolor}In [{\color{incolor}9}]:} \PY{n}{plt}\PY{o}{.}\PY{n}{imshow}\PY{p}{(}\PY{n}{gaussian\PYZus{}filter}\PY{p}{(}\PY{n}{sigma}\PY{o}{=}\PY{l+m+mi}{2}\PY{p}{)}\PY{p}{)}
\end{Verbatim}


\begin{Verbatim}[commandchars=\\\{\}]
{\color{outcolor}Out[{\color{outcolor}9}]:} <matplotlib.image.AxesImage at 0x28c5d7b9978>
\end{Verbatim}
            
    \begin{center}
    \adjustimage{max size={0.9\linewidth}{0.9\paperheight}}{output_20_1.png}
    \end{center}
    { \hspace*{\fill} \\}
    
    \section{Problem 2: Edge Detection}\label{problem-2-edge-detection}

There are a variety of filters that we can use for different tasks. One
such task is edge detection, which is useful for finding the boundaries
regions in an image. In this part, your task is to use convolutions to
find edges in images. Let's first load up an edgy image.

    \begin{Verbatim}[commandchars=\\\{\}]
{\color{incolor}In [{\color{incolor}10}]:} \PY{n}{im} \PY{o}{=} \PY{n}{load\PYZus{}image}\PY{p}{(}\PY{l+s+s1}{\PYZsq{}}\PY{l+s+s1}{edge\PYZus{}detection\PYZus{}image.jpg}\PY{l+s+s1}{\PYZsq{}}\PY{p}{)}
         \PY{n}{im} \PY{o}{=} \PY{n}{im}\PY{o}{.}\PY{n}{mean}\PY{p}{(}\PY{n}{axis}\PY{o}{=}\PY{l+m+mi}{2}\PY{p}{)} \PY{c+c1}{\PYZsh{} convert to grayscale}
         \PY{n}{show\PYZus{}image}\PY{p}{(}\PY{n}{im}\PY{p}{)}
\end{Verbatim}


    \begin{center}
    \adjustimage{max size={0.9\linewidth}{0.9\paperheight}}{output_22_0.png}
    \end{center}
    { \hspace*{\fill} \\}
    
    \subsection{Delta Filters}\label{delta-filters}

The simplest edge detector is a delta filter. Implement a delta filter
below, and convolve it with the image. Show the result.

    \begin{Verbatim}[commandchars=\\\{\}]
{\color{incolor}In [{\color{incolor}11}]:} \PY{n}{delta\PYZus{}filt} \PY{o}{=} \PY{n}{np}\PY{o}{.}\PY{n}{array}\PY{p}{(}\PY{p}{[}
             \PY{p}{[}\PY{o}{\PYZhy{}}\PY{l+m+mi}{1}\PY{p}{,}\PY{l+m+mi}{1}\PY{p}{]}
         \PY{p}{]}\PY{p}{)}
         \PY{l+s+sd}{\PYZsq{}\PYZsq{}\PYZsq{} TODO: Implement a delta filter here. \PYZsq{}\PYZsq{}\PYZsq{}}
         
         \PY{n}{plt}\PY{o}{.}\PY{n}{imshow}\PY{p}{(}\PY{n}{convolve2d}\PY{p}{(}\PY{n}{im}\PY{p}{,} \PY{n}{delta\PYZus{}filt}\PY{p}{)}\PY{p}{,} \PY{n}{cmap}\PY{o}{=}\PY{l+s+s1}{\PYZsq{}}\PY{l+s+s1}{gray}\PY{l+s+s1}{\PYZsq{}}\PY{p}{)}
\end{Verbatim}


\begin{Verbatim}[commandchars=\\\{\}]
{\color{outcolor}Out[{\color{outcolor}11}]:} <matplotlib.image.AxesImage at 0x28c5a039c18>
\end{Verbatim}
            
    \begin{center}
    \adjustimage{max size={0.9\linewidth}{0.9\paperheight}}{output_24_1.png}
    \end{center}
    { \hspace*{\fill} \\}
    
    \subsection{Noise}\label{noise}

The issue with the delta filter is that it is sensitive to noise in the
image. Let's add some Gaussian noise to the image below, and visualize
what happens. The edges should be hard to see.

    \begin{Verbatim}[commandchars=\\\{\}]
{\color{incolor}In [{\color{incolor}12}]:} \PY{n}{im} \PY{o}{=} \PY{n}{load\PYZus{}image}\PY{p}{(}\PY{l+s+s1}{\PYZsq{}}\PY{l+s+s1}{edge\PYZus{}detection\PYZus{}image.jpg}\PY{l+s+s1}{\PYZsq{}}\PY{p}{)}
         \PY{n}{im} \PY{o}{=} \PY{n}{im}\PY{o}{.}\PY{n}{mean}\PY{p}{(}\PY{n}{axis}\PY{o}{=}\PY{l+m+mi}{2}\PY{p}{)}
         \PY{n}{im} \PY{o}{=} \PY{n}{im} \PY{o}{+} \PY{l+m+mf}{0.2}\PY{o}{*}\PY{n}{np}\PY{o}{.}\PY{n}{random}\PY{o}{.}\PY{n}{randn}\PY{p}{(}\PY{o}{*}\PY{n}{im}\PY{o}{.}\PY{n}{shape}\PY{p}{)}
         
         \PY{n}{f}\PY{p}{,} \PY{n}{axarr} \PY{o}{=} \PY{n}{plt}\PY{o}{.}\PY{n}{subplots}\PY{p}{(}\PY{l+m+mi}{1}\PY{p}{,}\PY{l+m+mi}{2}\PY{p}{)}
         \PY{n}{axarr}\PY{p}{[}\PY{l+m+mi}{0}\PY{p}{]}\PY{o}{.}\PY{n}{imshow}\PY{p}{(}\PY{n}{im}\PY{p}{,} \PY{n}{cmap}\PY{o}{=}\PY{l+s+s1}{\PYZsq{}}\PY{l+s+s1}{gray}\PY{l+s+s1}{\PYZsq{}}\PY{p}{)}
         \PY{n}{axarr}\PY{p}{[}\PY{l+m+mi}{1}\PY{p}{]}\PY{o}{.}\PY{n}{imshow}\PY{p}{(}\PY{n}{convolve2d}\PY{p}{(}\PY{n}{im}\PY{p}{,} \PY{n}{delta\PYZus{}filt}\PY{p}{)}\PY{p}{,} \PY{n}{cmap}\PY{o}{=}\PY{l+s+s1}{\PYZsq{}}\PY{l+s+s1}{gray}\PY{l+s+s1}{\PYZsq{}}\PY{p}{)}
\end{Verbatim}


\begin{Verbatim}[commandchars=\\\{\}]
{\color{outcolor}Out[{\color{outcolor}12}]:} <matplotlib.image.AxesImage at 0x28c5d175438>
\end{Verbatim}
            
    \begin{center}
    \adjustimage{max size={0.9\linewidth}{0.9\paperheight}}{output_26_1.png}
    \end{center}
    { \hspace*{\fill} \\}
    
    \subsection{Laplacian Filters}\label{laplacian-filters}

Laplacian filters are edge detectors that are robust to noise (Why is
this? Think about how the filter is constructed.). Implement a Laplacian
filter below for both horizontal and vertical edges.

    \begin{Verbatim}[commandchars=\\\{\}]
{\color{incolor}In [{\color{incolor}13}]:} \PY{n}{x\PYZus{}delta} \PY{o}{=} \PY{n}{np}\PY{o}{.}\PY{n}{array}\PY{p}{(}
             \PY{p}{[}\PY{p}{[}\PY{o}{\PYZhy{}}\PY{l+m+mi}{1}\PY{p}{,}\PY{l+m+mi}{1}\PY{p}{]}\PY{p}{]}
         \PY{p}{)}
         
         \PY{n}{y\PYZus{}delta} \PY{o}{=} \PY{n}{np}\PY{o}{.}\PY{n}{array}\PY{p}{(}
             \PY{p}{[}\PY{p}{[}\PY{o}{\PYZhy{}}\PY{l+m+mi}{1}\PY{p}{]}\PY{p}{,}\PY{p}{[}\PY{l+m+mi}{1}\PY{p}{]}\PY{p}{]}
         \PY{p}{)}
         
         \PY{n}{lap\PYZus{}x\PYZus{}filt} \PY{o}{=} \PY{n}{convolve2d}\PY{p}{(}\PY{n}{convolve2d}\PY{p}{(}\PY{n}{gaussian\PYZus{}filter}\PY{p}{(}\PY{n}{sigma}\PY{o}{=}\PY{l+m+mi}{2}\PY{p}{)}\PY{p}{,}\PY{n}{x\PYZus{}delta}\PY{p}{)}\PY{p}{,}\PY{n}{x\PYZus{}delta}\PY{p}{)}
         \PY{l+s+sd}{\PYZsq{}\PYZsq{}\PYZsq{} TODO: Implement a Laplacian filter for horizontal edges. \PYZsq{}\PYZsq{}\PYZsq{}}
         \PY{n}{lap\PYZus{}y\PYZus{}filt} \PY{o}{=} \PY{n}{convolve2d}\PY{p}{(}\PY{n}{convolve2d}\PY{p}{(}\PY{n}{gaussian\PYZus{}filter}\PY{p}{(}\PY{n}{sigma}\PY{o}{=}\PY{l+m+mi}{2}\PY{p}{)}\PY{p}{,}\PY{n}{y\PYZus{}delta}\PY{p}{)}\PY{p}{,}\PY{n}{y\PYZus{}delta}\PY{p}{)}
         \PY{l+s+sd}{\PYZsq{}\PYZsq{}\PYZsq{} TODO: Implement a Laplacian filter for vertical edges. \PYZsq{}\PYZsq{}\PYZsq{}}
         
         \PY{n}{f}\PY{p}{,} \PY{n}{axarr} \PY{o}{=} \PY{n}{plt}\PY{o}{.}\PY{n}{subplots}\PY{p}{(}\PY{l+m+mi}{2}\PY{p}{,}\PY{l+m+mi}{2}\PY{p}{)}
         \PY{n}{axarr}\PY{p}{[}\PY{l+m+mi}{0}\PY{p}{,}\PY{l+m+mi}{0}\PY{p}{]}\PY{o}{.}\PY{n}{imshow}\PY{p}{(}\PY{n}{convolve2d}\PY{p}{(}\PY{n}{im}\PY{p}{,} \PY{n}{lap\PYZus{}y\PYZus{}filt}\PY{p}{)}\PY{p}{,} \PY{n}{cmap}\PY{o}{=}\PY{l+s+s1}{\PYZsq{}}\PY{l+s+s1}{gray}\PY{l+s+s1}{\PYZsq{}}\PY{p}{)}
         \PY{n}{axarr}\PY{p}{[}\PY{l+m+mi}{0}\PY{p}{,}\PY{l+m+mi}{1}\PY{p}{]}\PY{o}{.}\PY{n}{imshow}\PY{p}{(}\PY{n}{convolve2d}\PY{p}{(}\PY{n}{im}\PY{p}{,} \PY{n}{lap\PYZus{}x\PYZus{}filt}\PY{p}{)}\PY{p}{,} \PY{n}{cmap}\PY{o}{=}\PY{l+s+s1}{\PYZsq{}}\PY{l+s+s1}{gray}\PY{l+s+s1}{\PYZsq{}}\PY{p}{)}
         \PY{n}{axarr}\PY{p}{[}\PY{l+m+mi}{1}\PY{p}{,}\PY{l+m+mi}{0}\PY{p}{]}\PY{o}{.}\PY{n}{imshow}\PY{p}{(}\PY{n}{lap\PYZus{}y\PYZus{}filt}\PY{p}{,} \PY{n}{cmap}\PY{o}{=}\PY{l+s+s1}{\PYZsq{}}\PY{l+s+s1}{gray}\PY{l+s+s1}{\PYZsq{}}\PY{p}{)}
         \PY{n}{axarr}\PY{p}{[}\PY{l+m+mi}{1}\PY{p}{,}\PY{l+m+mi}{1}\PY{p}{]}\PY{o}{.}\PY{n}{imshow}\PY{p}{(}\PY{n}{lap\PYZus{}x\PYZus{}filt}\PY{p}{,} \PY{n}{cmap}\PY{o}{=}\PY{l+s+s1}{\PYZsq{}}\PY{l+s+s1}{gray}\PY{l+s+s1}{\PYZsq{}}\PY{p}{)}
\end{Verbatim}


\begin{Verbatim}[commandchars=\\\{\}]
{\color{outcolor}Out[{\color{outcolor}13}]:} <matplotlib.image.AxesImage at 0x28c5d4e88d0>
\end{Verbatim}
            
    \begin{center}
    \adjustimage{max size={0.9\linewidth}{0.9\paperheight}}{output_28_1.png}
    \end{center}
    { \hspace*{\fill} \\}
    
    \section{Problem 3: Hybrid Images}\label{problem-3-hybrid-images}

Hybrid images is a technique to combine two images in one. Depending on
the distance you view the image, you will see a different image. This is
done by merging the high-frequency components of one image with the
low-frequency components of a second image. In this problem, you are
going to use the Fourier transform to make these images. But first,
let's visualize the two images we will merge together.

    \begin{Verbatim}[commandchars=\\\{\}]
{\color{incolor}In [{\color{incolor}14}]:} \PY{k+kn}{from} \PY{n+nn}{numpy}\PY{n+nn}{.}\PY{n+nn}{fft} \PY{k}{import} \PY{n}{fft2}\PY{p}{,} \PY{n}{fftshift}\PY{p}{,} \PY{n}{ifftshift}\PY{p}{,} \PY{n}{ifft2}
         
         \PY{n}{dog} \PY{o}{=} \PY{n}{load\PYZus{}image}\PY{p}{(}\PY{l+s+s1}{\PYZsq{}}\PY{l+s+s1}{dog.jpg}\PY{l+s+s1}{\PYZsq{}}\PY{p}{)}\PY{o}{.}\PY{n}{mean}\PY{p}{(}\PY{n}{axis}\PY{o}{=}\PY{o}{\PYZhy{}}\PY{l+m+mi}{1}\PY{p}{)}\PY{p}{[}\PY{p}{:}\PY{p}{,} \PY{l+m+mi}{25}\PY{p}{:}\PY{o}{\PYZhy{}}\PY{l+m+mi}{24}\PY{p}{]}
         \PY{n}{cat} \PY{o}{=} \PY{n}{load\PYZus{}image}\PY{p}{(}\PY{l+s+s1}{\PYZsq{}}\PY{l+s+s1}{cat.jpg}\PY{l+s+s1}{\PYZsq{}}\PY{p}{)}\PY{o}{.}\PY{n}{mean}\PY{p}{(}\PY{n}{axis}\PY{o}{=}\PY{o}{\PYZhy{}}\PY{l+m+mi}{1}\PY{p}{)}\PY{p}{[}\PY{p}{:}\PY{p}{,} \PY{l+m+mi}{25}\PY{p}{:}\PY{o}{\PYZhy{}}\PY{l+m+mi}{24}\PY{p}{]}
         
         \PY{n}{f}\PY{p}{,} \PY{n}{axarr} \PY{o}{=} \PY{n}{plt}\PY{o}{.}\PY{n}{subplots}\PY{p}{(}\PY{l+m+mi}{1}\PY{p}{,}\PY{l+m+mi}{2}\PY{p}{)}
         \PY{n}{axarr}\PY{p}{[}\PY{l+m+mi}{0}\PY{p}{]}\PY{o}{.}\PY{n}{imshow}\PY{p}{(}\PY{n}{dog}\PY{p}{,} \PY{n}{cmap}\PY{o}{=}\PY{l+s+s1}{\PYZsq{}}\PY{l+s+s1}{gray}\PY{l+s+s1}{\PYZsq{}}\PY{p}{)}
         \PY{n}{axarr}\PY{p}{[}\PY{l+m+mi}{1}\PY{p}{]}\PY{o}{.}\PY{n}{imshow}\PY{p}{(}\PY{n}{cat}\PY{p}{,} \PY{n}{cmap}\PY{o}{=}\PY{l+s+s1}{\PYZsq{}}\PY{l+s+s1}{gray}\PY{l+s+s1}{\PYZsq{}}\PY{p}{)}
\end{Verbatim}


\begin{Verbatim}[commandchars=\\\{\}]
{\color{outcolor}Out[{\color{outcolor}14}]:} <matplotlib.image.AxesImage at 0x28c5d59ec88>
\end{Verbatim}
            
    \begin{center}
    \adjustimage{max size={0.9\linewidth}{0.9\paperheight}}{output_30_1.png}
    \end{center}
    { \hspace*{\fill} \\}
    
    \subsection{Fourier Transform}\label{fourier-transform}

In the code box below, compute the Fourier transform of the two images.
You can use the fft2 function. You can also use the fftshift function,
which may help in the next section.

    \begin{Verbatim}[commandchars=\\\{\}]
{\color{incolor}In [{\color{incolor}15}]:} \PY{n}{cat\PYZus{}fft} \PY{o}{=} \PY{n}{fft2}\PY{p}{(}\PY{n}{cat}\PY{p}{)}  
         \PY{l+s+sd}{\PYZsq{}\PYZsq{}\PYZsq{} TODO: compute the Fourier transform of the cat. \PYZsq{}\PYZsq{}\PYZsq{}}
         \PY{n}{dog\PYZus{}fft} \PY{o}{=} \PY{n}{fft2}\PY{p}{(}\PY{n}{dog}\PY{p}{)}  
         \PY{l+s+sd}{\PYZsq{}\PYZsq{}\PYZsq{} TODO: compute the Fourier transform of the dog. \PYZsq{}\PYZsq{}\PYZsq{}}
         
         \PY{n}{cat\PYZus{}fft} \PY{o}{=} \PY{n}{fftshift}\PY{p}{(}\PY{n}{cat\PYZus{}fft}\PY{p}{)}
         \PY{n}{dog\PYZus{}fft} \PY{o}{=} \PY{n}{fftshift}\PY{p}{(}\PY{n}{dog\PYZus{}fft}\PY{p}{)}
         
         
         \PY{c+c1}{\PYZsh{} Visualize the magnitude and phase of cat\PYZus{}fft. This is a complex number, so we visualize}
         \PY{c+c1}{\PYZsh{} the magnitude and angle of the complex number.}
         \PY{c+c1}{\PYZsh{} Curious fact: most of the information for natural images is stored in the phase (angle).}
         \PY{n}{f}\PY{p}{,} \PY{n}{axarr} \PY{o}{=} \PY{n}{plt}\PY{o}{.}\PY{n}{subplots}\PY{p}{(}\PY{l+m+mi}{1}\PY{p}{,}\PY{l+m+mi}{2}\PY{p}{)}
         \PY{n}{axarr}\PY{p}{[}\PY{l+m+mi}{0}\PY{p}{]}\PY{o}{.}\PY{n}{imshow}\PY{p}{(}\PY{n}{np}\PY{o}{.}\PY{n}{log}\PY{p}{(}\PY{n}{np}\PY{o}{.}\PY{n}{abs}\PY{p}{(}\PY{n}{cat\PYZus{}fft}\PY{p}{)}\PY{p}{)}\PY{p}{,} \PY{n}{cmap}\PY{o}{=}\PY{l+s+s1}{\PYZsq{}}\PY{l+s+s1}{gray}\PY{l+s+s1}{\PYZsq{}}\PY{p}{)}
         \PY{n}{axarr}\PY{p}{[}\PY{l+m+mi}{1}\PY{p}{]}\PY{o}{.}\PY{n}{imshow}\PY{p}{(}\PY{n}{np}\PY{o}{.}\PY{n}{angle}\PY{p}{(}\PY{n}{cat\PYZus{}fft}\PY{p}{)}\PY{p}{,} \PY{n}{cmap}\PY{o}{=}\PY{l+s+s1}{\PYZsq{}}\PY{l+s+s1}{gray}\PY{l+s+s1}{\PYZsq{}}\PY{p}{)}
\end{Verbatim}


\begin{Verbatim}[commandchars=\\\{\}]
{\color{outcolor}Out[{\color{outcolor}15}]:} <matplotlib.image.AxesImage at 0x28c5d634e80>
\end{Verbatim}
            
    \begin{center}
    \adjustimage{max size={0.9\linewidth}{0.9\paperheight}}{output_32_1.png}
    \end{center}
    { \hspace*{\fill} \\}
    
    \subsection{Low and High Pass Filters}\label{low-and-high-pass-filters}

By masking the Fourier transform, you can compute both low and high pass
of the images. In Fourier space, write code below to create the mask for
a high pass filter of the cat, and the mask for a low pass filter of the
dog. Then, convert back to image space and visualize these images.

You may need to use the functions ifft2 and ifftshift.

    \begin{Verbatim}[commandchars=\\\{\}]
{\color{incolor}In [{\color{incolor}16}]:} \PY{l+s+sd}{\PYZsq{}\PYZsq{}\PYZsq{} TODO: Create the mask for a high pass filter of the cat. \PYZsq{}\PYZsq{}\PYZsq{}}
         \PY{n}{cat\PYZus{}center} \PY{o}{=} \PY{p}{(}\PY{n}{cat}\PY{o}{.}\PY{n}{shape}\PY{p}{[}\PY{l+m+mi}{0}\PY{p}{]} \PY{o}{/}\PY{o}{/} \PY{l+m+mi}{2} \PY{o}{+} \PY{l+m+mi}{1}\PY{p}{,} \PY{n}{cat}\PY{o}{.}\PY{n}{shape}\PY{p}{[}\PY{l+m+mi}{1}\PY{p}{]} \PY{o}{/}\PY{o}{/} \PY{l+m+mi}{2} \PY{o}{+} \PY{l+m+mi}{1}\PY{p}{)}
         
         
         \PY{n}{threshold} \PY{o}{=} \PY{l+m+mi}{5}
         \PY{n}{high\PYZus{}mask} \PY{o}{=} \PY{n}{np}\PY{o}{.}\PY{n}{ones\PYZus{}like}\PY{p}{(}\PY{n}{cat}\PY{p}{)}
         \PY{n}{high\PYZus{}mask}\PY{p}{[}\PY{n}{cat\PYZus{}center}\PY{p}{[}\PY{l+m+mi}{0}\PY{p}{]}\PY{o}{\PYZhy{}}\PY{n}{threshold}\PY{p}{:}\PY{n}{cat\PYZus{}center}\PY{p}{[}\PY{l+m+mi}{0}\PY{p}{]}\PY{o}{+}\PY{n}{threshold}\PY{p}{,}\PY{n}{cat\PYZus{}center}\PY{p}{[}\PY{l+m+mi}{1}\PY{p}{]}\PY{o}{\PYZhy{}}\PY{n}{threshold}\PY{p}{:}\PY{n}{cat\PYZus{}center}\PY{p}{[}\PY{l+m+mi}{1}\PY{p}{]}\PY{o}{+}\PY{n}{threshold}\PY{p}{]} \PY{o}{=} \PY{l+m+mf}{0.}
         
         \PY{l+s+sd}{\PYZsq{}\PYZsq{}\PYZsq{} TODO: Create the mask for a low pass filter of the dog. \PYZsq{}\PYZsq{}\PYZsq{}}
         \PY{n}{dog\PYZus{}center} \PY{o}{=} \PY{p}{(}\PY{n}{dog}\PY{o}{.}\PY{n}{shape}\PY{p}{[}\PY{l+m+mi}{0}\PY{p}{]} \PY{o}{/}\PY{o}{/} \PY{l+m+mi}{2} \PY{o}{+} \PY{l+m+mi}{1}\PY{p}{,} \PY{n}{dog}\PY{o}{.}\PY{n}{shape}\PY{p}{[}\PY{l+m+mi}{1}\PY{p}{]} \PY{o}{/}\PY{o}{/} \PY{l+m+mi}{2} \PY{o}{+} \PY{l+m+mi}{1}\PY{p}{)}
         
         
         \PY{n}{low\PYZus{}mask} \PY{o}{=} \PY{n}{np}\PY{o}{.}\PY{n}{zeros\PYZus{}like}\PY{p}{(}\PY{n}{dog}\PY{p}{)}
         \PY{n}{low\PYZus{}mask}\PY{p}{[}\PY{n}{dog\PYZus{}center}\PY{p}{[}\PY{l+m+mi}{0}\PY{p}{]}\PY{o}{\PYZhy{}}\PY{n}{threshold}\PY{p}{:}\PY{n}{dog\PYZus{}center}\PY{p}{[}\PY{l+m+mi}{0}\PY{p}{]}\PY{o}{+}\PY{n}{threshold}\PY{p}{,}\PY{n}{dog\PYZus{}center}\PY{p}{[}\PY{l+m+mi}{1}\PY{p}{]}\PY{o}{\PYZhy{}}\PY{n}{threshold}\PY{p}{:}\PY{n}{dog\PYZus{}center}\PY{p}{[}\PY{l+m+mi}{1}\PY{p}{]}\PY{o}{+}\PY{n}{threshold}\PY{p}{]} \PY{o}{=} \PY{l+m+mf}{1.}
         
         \PY{l+s+sd}{\PYZsq{}\PYZsq{}\PYZsq{} TODO: Apply the high pass filter on the cat and convert back to image space. \PYZsq{}\PYZsq{}\PYZsq{}}
         
         \PY{n}{cat\PYZus{}filtered} \PY{o}{=} \PY{n}{np}\PY{o}{.}\PY{n}{multiply}\PY{p}{(}\PY{n}{cat\PYZus{}fft}\PY{p}{,}\PY{n}{high\PYZus{}mask}\PY{p}{)}
         \PY{n}{cat\PYZus{}filtered} \PY{o}{=} \PY{n}{ifftshift}\PY{p}{(}\PY{n}{cat\PYZus{}filtered}\PY{p}{)}
         \PY{n}{cat\PYZus{}filtered} \PY{o}{=} \PY{n}{np}\PY{o}{.}\PY{n}{abs}\PY{p}{(}\PY{n}{ifft2}\PY{p}{(}\PY{n}{cat\PYZus{}filtered}\PY{p}{)}\PY{p}{)}
         \PY{l+s+sd}{\PYZsq{}\PYZsq{}\PYZsq{} TODO: Apply the low pass filter on the dog and convert back to image space. \PYZsq{}\PYZsq{}\PYZsq{}} 
         
         \PY{n}{dog\PYZus{}filtered} \PY{o}{=} \PY{n}{np}\PY{o}{.}\PY{n}{multiply}\PY{p}{(}\PY{n}{dog\PYZus{}fft}\PY{p}{,}\PY{n}{low\PYZus{}mask}\PY{p}{)}
         \PY{n}{dog\PYZus{}filtered} \PY{o}{=} \PY{n}{ifftshift}\PY{p}{(}\PY{n}{dog\PYZus{}filtered}\PY{p}{)}
         \PY{n}{dog\PYZus{}filtered} \PY{o}{=} \PY{n}{np}\PY{o}{.}\PY{n}{abs}\PY{p}{(}\PY{n}{ifft2}\PY{p}{(}\PY{n}{dog\PYZus{}filtered}\PY{p}{)}\PY{p}{)}
         
         
         \PY{n}{f}\PY{p}{,} \PY{n}{axarr} \PY{o}{=} \PY{n}{plt}\PY{o}{.}\PY{n}{subplots}\PY{p}{(}\PY{l+m+mi}{1}\PY{p}{,}\PY{l+m+mi}{2}\PY{p}{)}
         \PY{n}{axarr}\PY{p}{[}\PY{l+m+mi}{0}\PY{p}{]}\PY{o}{.}\PY{n}{imshow}\PY{p}{(}\PY{n}{dog\PYZus{}filtered}\PY{p}{,} \PY{n}{cmap}\PY{o}{=}\PY{l+s+s1}{\PYZsq{}}\PY{l+s+s1}{gray}\PY{l+s+s1}{\PYZsq{}}\PY{p}{)}
         \PY{n}{axarr}\PY{p}{[}\PY{l+m+mi}{1}\PY{p}{]}\PY{o}{.}\PY{n}{imshow}\PY{p}{(}\PY{n}{cat\PYZus{}filtered}\PY{p}{,} \PY{n}{cmap}\PY{o}{=}\PY{l+s+s1}{\PYZsq{}}\PY{l+s+s1}{gray}\PY{l+s+s1}{\PYZsq{}}\PY{p}{)}
\end{Verbatim}


\begin{Verbatim}[commandchars=\\\{\}]
{\color{outcolor}Out[{\color{outcolor}16}]:} <matplotlib.image.AxesImage at 0x28c5d6cf978>
\end{Verbatim}
            
    \begin{center}
    \adjustimage{max size={0.9\linewidth}{0.9\paperheight}}{output_34_1.png}
    \end{center}
    { \hspace*{\fill} \\}
    
    \subsection{Hybrid Image Results}\label{hybrid-image-results}

Now that we have the high pass and low pass fitlered images, we can
create a hybrid image by adding them. Write the code to combine the
images below, and visualize the hybrd image.

Depending on whether you are close or far away from your monitor, you
should see either a cat or a dog. Try creating a few different hybrid
images from your own photos or photos you found. Submit them, and we
will show the coolest ones in class.

    \begin{Verbatim}[commandchars=\\\{\}]
{\color{incolor}In [{\color{incolor}17}]:} \PY{l+s+sd}{\PYZsq{}\PYZsq{}\PYZsq{} TODO: Compute the hybrid image here. \PYZsq{}\PYZsq{}\PYZsq{}}
         
         \PY{n}{hybrid} \PY{o}{=} \PY{n}{cat\PYZus{}fft} \PY{o}{*} \PY{n}{high\PYZus{}mask} \PY{o}{+} \PY{n}{dog\PYZus{}fft} \PY{o}{*} \PY{n}{low\PYZus{}mask}
         \PY{n}{hybrid} \PY{o}{=} \PY{n}{np}\PY{o}{.}\PY{n}{abs}\PY{p}{(}\PY{n}{ifft2}\PY{p}{(}\PY{n}{ifftshift}\PY{p}{(}\PY{n}{hybrid}\PY{p}{)}\PY{p}{)}\PY{p}{)}
         
         \PY{n}{plt}\PY{o}{.}\PY{n}{imshow}\PY{p}{(}\PY{n}{hybrid}\PY{p}{,} \PY{n}{cmap}\PY{o}{=}\PY{l+s+s1}{\PYZsq{}}\PY{l+s+s1}{gray}\PY{l+s+s1}{\PYZsq{}}\PY{p}{)}
\end{Verbatim}


\begin{Verbatim}[commandchars=\\\{\}]
{\color{outcolor}Out[{\color{outcolor}17}]:} <matplotlib.image.AxesImage at 0x28c5d739e10>
\end{Verbatim}
            
    \begin{center}
    \adjustimage{max size={0.9\linewidth}{0.9\paperheight}}{output_36_1.png}
    \end{center}
    { \hspace*{\fill} \\}
    
    \subsection{Acknowledgements}\label{acknowledgements}

This homework is based on assignments from Aude Oliva at MIT, and James
Hays at Georgia Tech.


    % Add a bibliography block to the postdoc
    
    
    
    \end{document}
