
% Default to the notebook output style

    


% Inherit from the specified cell style.




    
\documentclass[11pt]{article}

    
    
    \usepackage[T1]{fontenc}
    % Nicer default font (+ math font) than Computer Modern for most use cases
    \usepackage{mathpazo}

    % Basic figure setup, for now with no caption control since it's done
    % automatically by Pandoc (which extracts ![](path) syntax from Markdown).
    \usepackage{graphicx}
    % We will generate all images so they have a width \maxwidth. This means
    % that they will get their normal width if they fit onto the page, but
    % are scaled down if they would overflow the margins.
    \makeatletter
    \def\maxwidth{\ifdim\Gin@nat@width>\linewidth\linewidth
    \else\Gin@nat@width\fi}
    \makeatother
    \let\Oldincludegraphics\includegraphics
    % Set max figure width to be 80% of text width, for now hardcoded.
    \renewcommand{\includegraphics}[1]{\Oldincludegraphics[width=.8\maxwidth]{#1}}
    % Ensure that by default, figures have no caption (until we provide a
    % proper Figure object with a Caption API and a way to capture that
    % in the conversion process - todo).
    \usepackage{caption}
    \DeclareCaptionLabelFormat{nolabel}{}
    \captionsetup{labelformat=nolabel}

    \usepackage{adjustbox} % Used to constrain images to a maximum size 
    \usepackage{xcolor} % Allow colors to be defined
    \usepackage{enumerate} % Needed for markdown enumerations to work
    \usepackage{geometry} % Used to adjust the document margins
    \usepackage{amsmath} % Equations
    \usepackage{amssymb} % Equations
    \usepackage{textcomp} % defines textquotesingle
    % Hack from http://tex.stackexchange.com/a/47451/13684:
    \AtBeginDocument{%
        \def\PYZsq{\textquotesingle}% Upright quotes in Pygmentized code
    }
    \usepackage{upquote} % Upright quotes for verbatim code
    \usepackage{eurosym} % defines \euro
    \usepackage[mathletters]{ucs} % Extended unicode (utf-8) support
    \usepackage[utf8x]{inputenc} % Allow utf-8 characters in the tex document
    \usepackage{fancyvrb} % verbatim replacement that allows latex
    \usepackage{grffile} % extends the file name processing of package graphics 
                         % to support a larger range 
    % The hyperref package gives us a pdf with properly built
    % internal navigation ('pdf bookmarks' for the table of contents,
    % internal cross-reference links, web links for URLs, etc.)
    \usepackage{hyperref}
    \usepackage{longtable} % longtable support required by pandoc >1.10
    \usepackage{booktabs}  % table support for pandoc > 1.12.2
    \usepackage[inline]{enumitem} % IRkernel/repr support (it uses the enumerate* environment)
    \usepackage[normalem]{ulem} % ulem is needed to support strikethroughs (\sout)
                                % normalem makes italics be italics, not underlines
    

    
    
    % Colors for the hyperref package
    \definecolor{urlcolor}{rgb}{0,.145,.698}
    \definecolor{linkcolor}{rgb}{.71,0.21,0.01}
    \definecolor{citecolor}{rgb}{.12,.54,.11}

    % ANSI colors
    \definecolor{ansi-black}{HTML}{3E424D}
    \definecolor{ansi-black-intense}{HTML}{282C36}
    \definecolor{ansi-red}{HTML}{E75C58}
    \definecolor{ansi-red-intense}{HTML}{B22B31}
    \definecolor{ansi-green}{HTML}{00A250}
    \definecolor{ansi-green-intense}{HTML}{007427}
    \definecolor{ansi-yellow}{HTML}{DDB62B}
    \definecolor{ansi-yellow-intense}{HTML}{B27D12}
    \definecolor{ansi-blue}{HTML}{208FFB}
    \definecolor{ansi-blue-intense}{HTML}{0065CA}
    \definecolor{ansi-magenta}{HTML}{D160C4}
    \definecolor{ansi-magenta-intense}{HTML}{A03196}
    \definecolor{ansi-cyan}{HTML}{60C6C8}
    \definecolor{ansi-cyan-intense}{HTML}{258F8F}
    \definecolor{ansi-white}{HTML}{C5C1B4}
    \definecolor{ansi-white-intense}{HTML}{A1A6B2}

    % commands and environments needed by pandoc snippets
    % extracted from the output of `pandoc -s`
    \providecommand{\tightlist}{%
      \setlength{\itemsep}{0pt}\setlength{\parskip}{0pt}}
    \DefineVerbatimEnvironment{Highlighting}{Verbatim}{commandchars=\\\{\}}
    % Add ',fontsize=\small' for more characters per line
    \newenvironment{Shaded}{}{}
    \newcommand{\KeywordTok}[1]{\textcolor[rgb]{0.00,0.44,0.13}{\textbf{{#1}}}}
    \newcommand{\DataTypeTok}[1]{\textcolor[rgb]{0.56,0.13,0.00}{{#1}}}
    \newcommand{\DecValTok}[1]{\textcolor[rgb]{0.25,0.63,0.44}{{#1}}}
    \newcommand{\BaseNTok}[1]{\textcolor[rgb]{0.25,0.63,0.44}{{#1}}}
    \newcommand{\FloatTok}[1]{\textcolor[rgb]{0.25,0.63,0.44}{{#1}}}
    \newcommand{\CharTok}[1]{\textcolor[rgb]{0.25,0.44,0.63}{{#1}}}
    \newcommand{\StringTok}[1]{\textcolor[rgb]{0.25,0.44,0.63}{{#1}}}
    \newcommand{\CommentTok}[1]{\textcolor[rgb]{0.38,0.63,0.69}{\textit{{#1}}}}
    \newcommand{\OtherTok}[1]{\textcolor[rgb]{0.00,0.44,0.13}{{#1}}}
    \newcommand{\AlertTok}[1]{\textcolor[rgb]{1.00,0.00,0.00}{\textbf{{#1}}}}
    \newcommand{\FunctionTok}[1]{\textcolor[rgb]{0.02,0.16,0.49}{{#1}}}
    \newcommand{\RegionMarkerTok}[1]{{#1}}
    \newcommand{\ErrorTok}[1]{\textcolor[rgb]{1.00,0.00,0.00}{\textbf{{#1}}}}
    \newcommand{\NormalTok}[1]{{#1}}
    
    % Additional commands for more recent versions of Pandoc
    \newcommand{\ConstantTok}[1]{\textcolor[rgb]{0.53,0.00,0.00}{{#1}}}
    \newcommand{\SpecialCharTok}[1]{\textcolor[rgb]{0.25,0.44,0.63}{{#1}}}
    \newcommand{\VerbatimStringTok}[1]{\textcolor[rgb]{0.25,0.44,0.63}{{#1}}}
    \newcommand{\SpecialStringTok}[1]{\textcolor[rgb]{0.73,0.40,0.53}{{#1}}}
    \newcommand{\ImportTok}[1]{{#1}}
    \newcommand{\DocumentationTok}[1]{\textcolor[rgb]{0.73,0.13,0.13}{\textit{{#1}}}}
    \newcommand{\AnnotationTok}[1]{\textcolor[rgb]{0.38,0.63,0.69}{\textbf{\textit{{#1}}}}}
    \newcommand{\CommentVarTok}[1]{\textcolor[rgb]{0.38,0.63,0.69}{\textbf{\textit{{#1}}}}}
    \newcommand{\VariableTok}[1]{\textcolor[rgb]{0.10,0.09,0.49}{{#1}}}
    \newcommand{\ControlFlowTok}[1]{\textcolor[rgb]{0.00,0.44,0.13}{\textbf{{#1}}}}
    \newcommand{\OperatorTok}[1]{\textcolor[rgb]{0.40,0.40,0.40}{{#1}}}
    \newcommand{\BuiltInTok}[1]{{#1}}
    \newcommand{\ExtensionTok}[1]{{#1}}
    \newcommand{\PreprocessorTok}[1]{\textcolor[rgb]{0.74,0.48,0.00}{{#1}}}
    \newcommand{\AttributeTok}[1]{\textcolor[rgb]{0.49,0.56,0.16}{{#1}}}
    \newcommand{\InformationTok}[1]{\textcolor[rgb]{0.38,0.63,0.69}{\textbf{\textit{{#1}}}}}
    \newcommand{\WarningTok}[1]{\textcolor[rgb]{0.38,0.63,0.69}{\textbf{\textit{{#1}}}}}
    
    
    % Define a nice break command that doesn't care if a line doesn't already
    % exist.
    \def\br{\hspace*{\fill} \\* }
    % Math Jax compatability definitions
    \def\gt{>}
    \def\lt{<}
    % Document parameters
    \title{Homework5}
    
    
    

    % Pygments definitions
    
\makeatletter
\def\PY@reset{\let\PY@it=\relax \let\PY@bf=\relax%
    \let\PY@ul=\relax \let\PY@tc=\relax%
    \let\PY@bc=\relax \let\PY@ff=\relax}
\def\PY@tok#1{\csname PY@tok@#1\endcsname}
\def\PY@toks#1+{\ifx\relax#1\empty\else%
    \PY@tok{#1}\expandafter\PY@toks\fi}
\def\PY@do#1{\PY@bc{\PY@tc{\PY@ul{%
    \PY@it{\PY@bf{\PY@ff{#1}}}}}}}
\def\PY#1#2{\PY@reset\PY@toks#1+\relax+\PY@do{#2}}

\expandafter\def\csname PY@tok@w\endcsname{\def\PY@tc##1{\textcolor[rgb]{0.73,0.73,0.73}{##1}}}
\expandafter\def\csname PY@tok@c\endcsname{\let\PY@it=\textit\def\PY@tc##1{\textcolor[rgb]{0.25,0.50,0.50}{##1}}}
\expandafter\def\csname PY@tok@cp\endcsname{\def\PY@tc##1{\textcolor[rgb]{0.74,0.48,0.00}{##1}}}
\expandafter\def\csname PY@tok@k\endcsname{\let\PY@bf=\textbf\def\PY@tc##1{\textcolor[rgb]{0.00,0.50,0.00}{##1}}}
\expandafter\def\csname PY@tok@kp\endcsname{\def\PY@tc##1{\textcolor[rgb]{0.00,0.50,0.00}{##1}}}
\expandafter\def\csname PY@tok@kt\endcsname{\def\PY@tc##1{\textcolor[rgb]{0.69,0.00,0.25}{##1}}}
\expandafter\def\csname PY@tok@o\endcsname{\def\PY@tc##1{\textcolor[rgb]{0.40,0.40,0.40}{##1}}}
\expandafter\def\csname PY@tok@ow\endcsname{\let\PY@bf=\textbf\def\PY@tc##1{\textcolor[rgb]{0.67,0.13,1.00}{##1}}}
\expandafter\def\csname PY@tok@nb\endcsname{\def\PY@tc##1{\textcolor[rgb]{0.00,0.50,0.00}{##1}}}
\expandafter\def\csname PY@tok@nf\endcsname{\def\PY@tc##1{\textcolor[rgb]{0.00,0.00,1.00}{##1}}}
\expandafter\def\csname PY@tok@nc\endcsname{\let\PY@bf=\textbf\def\PY@tc##1{\textcolor[rgb]{0.00,0.00,1.00}{##1}}}
\expandafter\def\csname PY@tok@nn\endcsname{\let\PY@bf=\textbf\def\PY@tc##1{\textcolor[rgb]{0.00,0.00,1.00}{##1}}}
\expandafter\def\csname PY@tok@ne\endcsname{\let\PY@bf=\textbf\def\PY@tc##1{\textcolor[rgb]{0.82,0.25,0.23}{##1}}}
\expandafter\def\csname PY@tok@nv\endcsname{\def\PY@tc##1{\textcolor[rgb]{0.10,0.09,0.49}{##1}}}
\expandafter\def\csname PY@tok@no\endcsname{\def\PY@tc##1{\textcolor[rgb]{0.53,0.00,0.00}{##1}}}
\expandafter\def\csname PY@tok@nl\endcsname{\def\PY@tc##1{\textcolor[rgb]{0.63,0.63,0.00}{##1}}}
\expandafter\def\csname PY@tok@ni\endcsname{\let\PY@bf=\textbf\def\PY@tc##1{\textcolor[rgb]{0.60,0.60,0.60}{##1}}}
\expandafter\def\csname PY@tok@na\endcsname{\def\PY@tc##1{\textcolor[rgb]{0.49,0.56,0.16}{##1}}}
\expandafter\def\csname PY@tok@nt\endcsname{\let\PY@bf=\textbf\def\PY@tc##1{\textcolor[rgb]{0.00,0.50,0.00}{##1}}}
\expandafter\def\csname PY@tok@nd\endcsname{\def\PY@tc##1{\textcolor[rgb]{0.67,0.13,1.00}{##1}}}
\expandafter\def\csname PY@tok@s\endcsname{\def\PY@tc##1{\textcolor[rgb]{0.73,0.13,0.13}{##1}}}
\expandafter\def\csname PY@tok@sd\endcsname{\let\PY@it=\textit\def\PY@tc##1{\textcolor[rgb]{0.73,0.13,0.13}{##1}}}
\expandafter\def\csname PY@tok@si\endcsname{\let\PY@bf=\textbf\def\PY@tc##1{\textcolor[rgb]{0.73,0.40,0.53}{##1}}}
\expandafter\def\csname PY@tok@se\endcsname{\let\PY@bf=\textbf\def\PY@tc##1{\textcolor[rgb]{0.73,0.40,0.13}{##1}}}
\expandafter\def\csname PY@tok@sr\endcsname{\def\PY@tc##1{\textcolor[rgb]{0.73,0.40,0.53}{##1}}}
\expandafter\def\csname PY@tok@ss\endcsname{\def\PY@tc##1{\textcolor[rgb]{0.10,0.09,0.49}{##1}}}
\expandafter\def\csname PY@tok@sx\endcsname{\def\PY@tc##1{\textcolor[rgb]{0.00,0.50,0.00}{##1}}}
\expandafter\def\csname PY@tok@m\endcsname{\def\PY@tc##1{\textcolor[rgb]{0.40,0.40,0.40}{##1}}}
\expandafter\def\csname PY@tok@gh\endcsname{\let\PY@bf=\textbf\def\PY@tc##1{\textcolor[rgb]{0.00,0.00,0.50}{##1}}}
\expandafter\def\csname PY@tok@gu\endcsname{\let\PY@bf=\textbf\def\PY@tc##1{\textcolor[rgb]{0.50,0.00,0.50}{##1}}}
\expandafter\def\csname PY@tok@gd\endcsname{\def\PY@tc##1{\textcolor[rgb]{0.63,0.00,0.00}{##1}}}
\expandafter\def\csname PY@tok@gi\endcsname{\def\PY@tc##1{\textcolor[rgb]{0.00,0.63,0.00}{##1}}}
\expandafter\def\csname PY@tok@gr\endcsname{\def\PY@tc##1{\textcolor[rgb]{1.00,0.00,0.00}{##1}}}
\expandafter\def\csname PY@tok@ge\endcsname{\let\PY@it=\textit}
\expandafter\def\csname PY@tok@gs\endcsname{\let\PY@bf=\textbf}
\expandafter\def\csname PY@tok@gp\endcsname{\let\PY@bf=\textbf\def\PY@tc##1{\textcolor[rgb]{0.00,0.00,0.50}{##1}}}
\expandafter\def\csname PY@tok@go\endcsname{\def\PY@tc##1{\textcolor[rgb]{0.53,0.53,0.53}{##1}}}
\expandafter\def\csname PY@tok@gt\endcsname{\def\PY@tc##1{\textcolor[rgb]{0.00,0.27,0.87}{##1}}}
\expandafter\def\csname PY@tok@err\endcsname{\def\PY@bc##1{\setlength{\fboxsep}{0pt}\fcolorbox[rgb]{1.00,0.00,0.00}{1,1,1}{\strut ##1}}}
\expandafter\def\csname PY@tok@kc\endcsname{\let\PY@bf=\textbf\def\PY@tc##1{\textcolor[rgb]{0.00,0.50,0.00}{##1}}}
\expandafter\def\csname PY@tok@kd\endcsname{\let\PY@bf=\textbf\def\PY@tc##1{\textcolor[rgb]{0.00,0.50,0.00}{##1}}}
\expandafter\def\csname PY@tok@kn\endcsname{\let\PY@bf=\textbf\def\PY@tc##1{\textcolor[rgb]{0.00,0.50,0.00}{##1}}}
\expandafter\def\csname PY@tok@kr\endcsname{\let\PY@bf=\textbf\def\PY@tc##1{\textcolor[rgb]{0.00,0.50,0.00}{##1}}}
\expandafter\def\csname PY@tok@bp\endcsname{\def\PY@tc##1{\textcolor[rgb]{0.00,0.50,0.00}{##1}}}
\expandafter\def\csname PY@tok@fm\endcsname{\def\PY@tc##1{\textcolor[rgb]{0.00,0.00,1.00}{##1}}}
\expandafter\def\csname PY@tok@vc\endcsname{\def\PY@tc##1{\textcolor[rgb]{0.10,0.09,0.49}{##1}}}
\expandafter\def\csname PY@tok@vg\endcsname{\def\PY@tc##1{\textcolor[rgb]{0.10,0.09,0.49}{##1}}}
\expandafter\def\csname PY@tok@vi\endcsname{\def\PY@tc##1{\textcolor[rgb]{0.10,0.09,0.49}{##1}}}
\expandafter\def\csname PY@tok@vm\endcsname{\def\PY@tc##1{\textcolor[rgb]{0.10,0.09,0.49}{##1}}}
\expandafter\def\csname PY@tok@sa\endcsname{\def\PY@tc##1{\textcolor[rgb]{0.73,0.13,0.13}{##1}}}
\expandafter\def\csname PY@tok@sb\endcsname{\def\PY@tc##1{\textcolor[rgb]{0.73,0.13,0.13}{##1}}}
\expandafter\def\csname PY@tok@sc\endcsname{\def\PY@tc##1{\textcolor[rgb]{0.73,0.13,0.13}{##1}}}
\expandafter\def\csname PY@tok@dl\endcsname{\def\PY@tc##1{\textcolor[rgb]{0.73,0.13,0.13}{##1}}}
\expandafter\def\csname PY@tok@s2\endcsname{\def\PY@tc##1{\textcolor[rgb]{0.73,0.13,0.13}{##1}}}
\expandafter\def\csname PY@tok@sh\endcsname{\def\PY@tc##1{\textcolor[rgb]{0.73,0.13,0.13}{##1}}}
\expandafter\def\csname PY@tok@s1\endcsname{\def\PY@tc##1{\textcolor[rgb]{0.73,0.13,0.13}{##1}}}
\expandafter\def\csname PY@tok@mb\endcsname{\def\PY@tc##1{\textcolor[rgb]{0.40,0.40,0.40}{##1}}}
\expandafter\def\csname PY@tok@mf\endcsname{\def\PY@tc##1{\textcolor[rgb]{0.40,0.40,0.40}{##1}}}
\expandafter\def\csname PY@tok@mh\endcsname{\def\PY@tc##1{\textcolor[rgb]{0.40,0.40,0.40}{##1}}}
\expandafter\def\csname PY@tok@mi\endcsname{\def\PY@tc##1{\textcolor[rgb]{0.40,0.40,0.40}{##1}}}
\expandafter\def\csname PY@tok@il\endcsname{\def\PY@tc##1{\textcolor[rgb]{0.40,0.40,0.40}{##1}}}
\expandafter\def\csname PY@tok@mo\endcsname{\def\PY@tc##1{\textcolor[rgb]{0.40,0.40,0.40}{##1}}}
\expandafter\def\csname PY@tok@ch\endcsname{\let\PY@it=\textit\def\PY@tc##1{\textcolor[rgb]{0.25,0.50,0.50}{##1}}}
\expandafter\def\csname PY@tok@cm\endcsname{\let\PY@it=\textit\def\PY@tc##1{\textcolor[rgb]{0.25,0.50,0.50}{##1}}}
\expandafter\def\csname PY@tok@cpf\endcsname{\let\PY@it=\textit\def\PY@tc##1{\textcolor[rgb]{0.25,0.50,0.50}{##1}}}
\expandafter\def\csname PY@tok@c1\endcsname{\let\PY@it=\textit\def\PY@tc##1{\textcolor[rgb]{0.25,0.50,0.50}{##1}}}
\expandafter\def\csname PY@tok@cs\endcsname{\let\PY@it=\textit\def\PY@tc##1{\textcolor[rgb]{0.25,0.50,0.50}{##1}}}

\def\PYZbs{\char`\\}
\def\PYZus{\char`\_}
\def\PYZob{\char`\{}
\def\PYZcb{\char`\}}
\def\PYZca{\char`\^}
\def\PYZam{\char`\&}
\def\PYZlt{\char`\<}
\def\PYZgt{\char`\>}
\def\PYZsh{\char`\#}
\def\PYZpc{\char`\%}
\def\PYZdl{\char`\$}
\def\PYZhy{\char`\-}
\def\PYZsq{\char`\'}
\def\PYZdq{\char`\"}
\def\PYZti{\char`\~}
% for compatibility with earlier versions
\def\PYZat{@}
\def\PYZlb{[}
\def\PYZrb{]}
\makeatother


    % Exact colors from NB
    \definecolor{incolor}{rgb}{0.0, 0.0, 0.5}
    \definecolor{outcolor}{rgb}{0.545, 0.0, 0.0}



    
    % Prevent overflowing lines due to hard-to-break entities
    \sloppy 
    % Setup hyperref package
    \hypersetup{
      breaklinks=true,  % so long urls are correctly broken across lines
      colorlinks=true,
      urlcolor=urlcolor,
      linkcolor=linkcolor,
      citecolor=citecolor,
      }
    % Slightly bigger margins than the latex defaults
    
    \geometry{verbose,tmargin=1in,bmargin=1in,lmargin=1in,rmargin=1in}
    
    

    \begin{document}
    
    
    \maketitle
    
    

    
    \section{Homework 5: Stereo}\label{homework-5-stereo}

In this homework, you will implement an two-view stereo algorithm. Given
two images from a stereo rig, you can estimate the 3D depth of the scene
by first calculating the disparity between pixels, and use the camera
intrinsics to triangulate the depth. This homework will focus only on
estimating the disparity between pixels.

You will estimate two stereo algorithms, which are both worth 50 points.
In the first stereo algorithm, you should implement a greedy matching
approach. In the second stereo algorithm, you should implement a stereo
algorithm that incorporates a prior of spatial smoothness, which you
optimize with dynamic programming.

But, before we begin, let's load some helper functions.

    \begin{Verbatim}[commandchars=\\\{\}]
{\color{incolor}In [{\color{incolor}1}]:} \PY{k+kn}{import} \PY{n+nn}{numpy} \PY{k}{as} \PY{n+nn}{np}
        \PY{k+kn}{import} \PY{n+nn}{matplotlib}\PY{n+nn}{.}\PY{n+nn}{pyplot} \PY{k}{as} \PY{n+nn}{plt}
        \PY{k+kn}{from} \PY{n+nn}{PIL} \PY{k}{import} \PY{n}{Image}
        \PY{k+kn}{from} \PY{n+nn}{IPython} \PY{k}{import} \PY{n}{display}
        \PY{k+kn}{from} \PY{n+nn}{scipy}\PY{n+nn}{.}\PY{n+nn}{signal} \PY{k}{import} \PY{n}{convolve2d}
        \PY{k+kn}{from} \PY{n+nn}{math} \PY{k}{import} \PY{o}{*}
        \PY{k+kn}{from} \PY{n+nn}{tqdm} \PY{k}{import} \PY{n}{tqdm}
        \PY{k+kn}{import} \PY{n+nn}{time}
        \PY{n}{plt}\PY{o}{.}\PY{n}{rcParams}\PY{p}{[}\PY{l+s+s1}{\PYZsq{}}\PY{l+s+s1}{figure.figsize}\PY{l+s+s1}{\PYZsq{}}\PY{p}{]} \PY{o}{=} \PY{p}{[}\PY{l+m+mi}{7}\PY{p}{,} \PY{l+m+mi}{7}\PY{p}{]}
        
        \PY{k}{def} \PY{n+nf}{load\PYZus{}image}\PY{p}{(}\PY{n}{filename}\PY{p}{)}\PY{p}{:}
            \PY{n}{img} \PY{o}{=} \PY{n}{np}\PY{o}{.}\PY{n}{asarray}\PY{p}{(}\PY{n}{Image}\PY{o}{.}\PY{n}{open}\PY{p}{(}\PY{n}{filename}\PY{p}{)}\PY{p}{)}
            \PY{n}{img} \PY{o}{=} \PY{n}{img}\PY{o}{.}\PY{n}{astype}\PY{p}{(}\PY{l+s+s2}{\PYZdq{}}\PY{l+s+s2}{float32}\PY{l+s+s2}{\PYZdq{}}\PY{p}{)} \PY{o}{/} \PY{l+m+mf}{255.}
            \PY{k}{return} \PY{n}{img}
        
        \PY{k}{def} \PY{n+nf}{gray2rgb}\PY{p}{(}\PY{n}{image}\PY{p}{)}\PY{p}{:}
            \PY{k}{return} \PY{n}{np}\PY{o}{.}\PY{n}{repeat}\PY{p}{(}\PY{n}{np}\PY{o}{.}\PY{n}{expand\PYZus{}dims}\PY{p}{(}\PY{n}{image}\PY{p}{,} \PY{l+m+mi}{2}\PY{p}{)}\PY{p}{,} \PY{l+m+mi}{3}\PY{p}{,} \PY{n}{axis}\PY{o}{=}\PY{l+m+mi}{2}\PY{p}{)}
        
        \PY{k}{def} \PY{n+nf}{show\PYZus{}image}\PY{p}{(}\PY{n}{img}\PY{p}{)}\PY{p}{:}
            \PY{k}{if} \PY{n+nb}{len}\PY{p}{(}\PY{n}{img}\PY{o}{.}\PY{n}{shape}\PY{p}{)} \PY{o}{==} \PY{l+m+mi}{2}\PY{p}{:}
                \PY{n}{img} \PY{o}{=} \PY{n}{gray2rgb}\PY{p}{(}\PY{n}{img}\PY{p}{)}
            \PY{n}{plt}\PY{o}{.}\PY{n}{imshow}\PY{p}{(}\PY{n}{img}\PY{p}{,} \PY{n}{interpolation}\PY{o}{=}\PY{l+s+s1}{\PYZsq{}}\PY{l+s+s1}{nearest}\PY{l+s+s1}{\PYZsq{}}\PY{p}{)}
\end{Verbatim}


    \subsection{Visualizing Stereo Pairs}\label{visualizing-stereo-pairs}

Let's visualize the images captured by the left and right camera pair.
These images from the Middlebury Stereo Dataset
(http://vision.middlebury.edu/stereo/data/).

    \begin{Verbatim}[commandchars=\\\{\}]
{\color{incolor}In [{\color{incolor}2}]:} \PY{n}{left} \PY{o}{=} \PY{n}{load\PYZus{}image}\PY{p}{(}\PY{l+s+s1}{\PYZsq{}}\PY{l+s+s1}{im0.jpg}\PY{l+s+s1}{\PYZsq{}}\PY{p}{)}
        \PY{n}{right} \PY{o}{=} \PY{n}{load\PYZus{}image}\PY{p}{(}\PY{l+s+s1}{\PYZsq{}}\PY{l+s+s1}{im1.jpg}\PY{l+s+s1}{\PYZsq{}}\PY{p}{)}
        
        \PY{n}{height}\PY{p}{,}\PY{n}{width}\PY{p}{,}\PY{n}{\PYZus{}} \PY{o}{=} \PY{n}{left}\PY{o}{.}\PY{n}{shape}
        
        \PY{n}{pad\PYZus{}size} \PY{o}{=} \PY{l+m+mi}{100}
        
        \PY{n}{left\PYZus{}pad} \PY{o}{=} \PY{n}{np}\PY{o}{.}\PY{n}{pad}\PY{p}{(}\PY{n}{left}\PY{p}{,} \PY{p}{(}\PY{p}{(}\PY{n}{pad\PYZus{}size}\PY{p}{,} \PY{n}{pad\PYZus{}size}\PY{p}{)}\PY{p}{,} \PY{p}{(}\PY{n}{pad\PYZus{}size}\PY{p}{,} \PY{n}{pad\PYZus{}size}\PY{p}{)}\PY{p}{,} \PY{p}{(}\PY{l+m+mi}{0}\PY{p}{,}\PY{l+m+mi}{0}\PY{p}{)}\PY{p}{)}\PY{p}{,} \PY{n}{mode}\PY{o}{=}\PY{l+s+s1}{\PYZsq{}}\PY{l+s+s1}{constant}\PY{l+s+s1}{\PYZsq{}}\PY{p}{)}
        \PY{n}{right\PYZus{}pad} \PY{o}{=} \PY{n}{np}\PY{o}{.}\PY{n}{pad}\PY{p}{(}\PY{n}{right}\PY{p}{,} \PY{p}{(}\PY{p}{(}\PY{n}{pad\PYZus{}size}\PY{p}{,} \PY{n}{pad\PYZus{}size}\PY{p}{)}\PY{p}{,} \PY{p}{(}\PY{n}{pad\PYZus{}size}\PY{p}{,} \PY{n}{pad\PYZus{}size}\PY{p}{)}\PY{p}{,} \PY{p}{(}\PY{l+m+mi}{0}\PY{p}{,}\PY{l+m+mi}{0}\PY{p}{)}\PY{p}{)}\PY{p}{,} \PY{n}{mode}\PY{o}{=}\PY{l+s+s1}{\PYZsq{}}\PY{l+s+s1}{constant}\PY{l+s+s1}{\PYZsq{}}\PY{p}{)}
        
        \PY{n}{show\PYZus{}image}\PY{p}{(}\PY{n}{np}\PY{o}{.}\PY{n}{concatenate}\PY{p}{(}\PY{p}{[}\PY{n}{left}\PY{p}{,} \PY{n}{right}\PY{p}{]}\PY{p}{,} \PY{n}{axis}\PY{o}{=}\PY{l+m+mi}{1}\PY{p}{)}\PY{p}{)}
\end{Verbatim}


    \begin{center}
    \adjustimage{max size={0.9\linewidth}{0.9\paperheight}}{output_3_0.png}
    \end{center}
    { \hspace*{\fill} \\}
    
    \subsection{Problem 1: Greedy Stereo Matching (50
points)}\label{problem-1-greedy-stereo-matching-50-points}

From the left and right image, calculate the disparity between each
pixel using a greedy matching algorithm. You may assume that the stereo
pairs are rectified, which means the camera pairs are only horizontally
translated. In your PDF, be sure to include both the code and the
visualization of the estimated disparity.

\textbf{Distance Function}: There are a variety of distance functions
that you can use. For this problem, you can just use sum of squared
differences between RGB patches.

\textbf{Accuracy}: It is very hard to get the right result with a greedy
approach. However, your approach should at least put the head and
background in right place.

    \begin{Verbatim}[commandchars=\\\{\}]
{\color{incolor}In [{\color{incolor}3}]:} \PY{k}{def} \PY{n+nf}{compute\PYZus{}ssd}\PY{p}{(}\PY{n}{matrix1}\PY{p}{,} \PY{n}{matrix2}\PY{p}{)}\PY{p}{:}
            \PY{c+c1}{\PYZsh{} compute ssd between two 2\PYZhy{}D arrays}
            \PY{k}{return} \PY{n}{np}\PY{o}{.}\PY{n}{sum}\PY{p}{(}\PY{p}{(}\PY{n}{matrix1}\PY{o}{\PYZhy{}}\PY{n}{matrix2}\PY{p}{)}\PY{o}{*}\PY{o}{*}\PY{l+m+mi}{2}\PY{p}{)}
\end{Verbatim}


    \begin{Verbatim}[commandchars=\\\{\}]
{\color{incolor}In [{\color{incolor}4}]:} \PY{c+c1}{\PYZsh{} Some variables that might help you. But you don\PYZsq{}t have to use them.}
        \PY{n}{max\PYZus{}dx} \PY{o}{=} \PY{l+m+mi}{30} \PY{c+c1}{\PYZsh{} max number of displacements to search. make this smaller to speed up}
        \PY{n}{dxs} \PY{o}{=} \PY{n}{np}\PY{o}{.}\PY{n}{linspace}\PY{p}{(}\PY{o}{\PYZhy{}}\PY{n}{max\PYZus{}dx}\PY{p}{,} \PY{n}{max\PYZus{}dx}\PY{p}{,} \PY{n}{num}\PY{o}{=}\PY{l+m+mi}{2}\PY{o}{*}\PY{n}{max\PYZus{}dx}\PY{o}{+}\PY{l+m+mi}{1}\PY{p}{)}\PY{o}{.}\PY{n}{astype}\PY{p}{(}\PY{l+s+s2}{\PYZdq{}}\PY{l+s+s2}{int32}\PY{l+s+s2}{\PYZdq{}}\PY{p}{)}
        \PY{n}{win\PYZus{}size} \PY{o}{=} \PY{l+m+mi}{5} \PY{c+c1}{\PYZsh{} size of window you use for ssd computation}
        
        \PY{k}{def} \PY{n+nf}{compute\PYZus{}disparity}\PY{p}{(}\PY{p}{)}\PY{p}{:}
            \PY{c+c1}{\PYZsh{} TODO: calculate the disparity using a greedy approach between left and right}
            \PY{n}{disparity} \PY{o}{=} \PY{n}{np}\PY{o}{.}\PY{n}{zeros}\PY{p}{(}\PY{p}{(}\PY{n}{height}\PY{p}{,}\PY{n}{width}\PY{p}{)}\PY{p}{)}
            \PY{n}{half\PYZus{}win\PYZus{}size} \PY{o}{=} \PY{n}{win\PYZus{}size} \PY{o}{/}\PY{o}{/} \PY{l+m+mi}{2}
            \PY{n}{best\PYZus{}offset} \PY{o}{=} \PY{l+m+mi}{0}
            \PY{n}{min\PYZus{}ssd} \PY{o}{=} \PY{n}{np}\PY{o}{.}\PY{n}{inf}
            
            \PY{k}{for} \PY{n}{y} \PY{o+ow}{in} \PY{n+nb}{range}\PY{p}{(}\PY{n}{height}\PY{p}{)}\PY{p}{:}
                \PY{k}{for} \PY{n}{x} \PY{o+ow}{in} \PY{n+nb}{range}\PY{p}{(}\PY{n}{width}\PY{p}{)}\PY{p}{:}
                    \PY{n}{left\PYZus{}temp} \PY{o}{=} \PY{n}{left\PYZus{}pad}\PY{p}{[}\PY{n}{pad\PYZus{}size}\PY{o}{+}\PY{n}{y}\PY{o}{\PYZhy{}}\PY{n}{half\PYZus{}win\PYZus{}size}\PY{p}{:}\PY{n}{pad\PYZus{}size}\PY{o}{+}\PY{n}{y}\PY{o}{+}\PY{n}{half\PYZus{}win\PYZus{}size}\PY{o}{+}\PY{l+m+mi}{1}\PY{p}{,}\PY{n}{pad\PYZus{}size}\PY{o}{+}\PY{n}{x}\PY{o}{\PYZhy{}}\PY{n}{half\PYZus{}win\PYZus{}size}\PY{p}{:}\PY{n}{pad\PYZus{}size}\PY{o}{+}\PY{n}{x}\PY{o}{+}\PY{n}{half\PYZus{}win\PYZus{}size}\PY{o}{+}\PY{l+m+mi}{1}\PY{p}{]}
                    \PY{n}{ssd} \PY{o}{=} \PY{p}{[}\PY{p}{]}
                    \PY{k}{for} \PY{n}{offset} \PY{o+ow}{in} \PY{n}{dxs}\PY{p}{:}
                        \PY{n}{right\PYZus{}temp} \PY{o}{=} \PY{n}{right\PYZus{}pad}\PY{p}{[}\PY{n}{pad\PYZus{}size}\PY{o}{+}\PY{n}{y}\PY{o}{\PYZhy{}}\PY{n}{half\PYZus{}win\PYZus{}size}\PY{p}{:}\PY{n}{pad\PYZus{}size}\PY{o}{+}\PY{n}{y}\PY{o}{+}\PY{n}{half\PYZus{}win\PYZus{}size}\PY{o}{+}\PY{l+m+mi}{1}\PY{p}{,}\PY{n}{pad\PYZus{}size}\PY{o}{+}\PY{n}{x}\PY{o}{\PYZhy{}}\PY{n}{half\PYZus{}win\PYZus{}size}\PY{o}{+}\PY{n}{offset}\PY{p}{:}\PY{n}{pad\PYZus{}size}\PY{o}{+}\PY{n}{x}\PY{o}{+}\PY{n}{half\PYZus{}win\PYZus{}size}\PY{o}{+}\PY{l+m+mi}{1}\PY{o}{+}\PY{n}{offset}\PY{p}{]}
                        \PY{n}{ssd\PYZus{}temp} \PY{o}{=} \PY{n}{compute\PYZus{}ssd}\PY{p}{(}\PY{n}{left\PYZus{}temp}\PY{p}{,} \PY{n}{right\PYZus{}temp}\PY{p}{)}
                        \PY{n}{ssd}\PY{o}{.}\PY{n}{append}\PY{p}{(}\PY{n}{ssd\PYZus{}temp}\PY{p}{)}
                        
                    \PY{n}{best\PYZus{}offset} \PY{o}{=} \PY{n}{dxs}\PY{p}{[}\PY{n}{np}\PY{o}{.}\PY{n}{argmin}\PY{p}{(}\PY{n}{ssd}\PY{p}{)}\PY{p}{]}
                    \PY{n}{disparity}\PY{p}{[}\PY{n}{y}\PY{p}{,}\PY{n}{x}\PY{p}{]} \PY{o}{=} \PY{n}{best\PYZus{}offset}
            
            \PY{k}{return} \PY{n}{disparity}
        
        \PY{n}{disparity} \PY{o}{=} \PY{n}{compute\PYZus{}disparity}\PY{p}{(}\PY{p}{)}
        \PY{n}{plt}\PY{o}{.}\PY{n}{matshow}\PY{p}{(}\PY{n}{disparity}\PY{p}{)}
\end{Verbatim}


\begin{Verbatim}[commandchars=\\\{\}]
{\color{outcolor}Out[{\color{outcolor}4}]:} <matplotlib.image.AxesImage at 0x1ffd62b95c0>
\end{Verbatim}
            
    \begin{center}
    \adjustimage{max size={0.9\linewidth}{0.9\paperheight}}{output_6_1.png}
    \end{center}
    { \hspace*{\fill} \\}
    
    \subsection{Problem 2: Dynamic Programming (50
points)}\label{problem-2-dynamic-programming-50-points}

In this problem, implement a dynamic programming based stereo estimation
algorithm that incorporates the spatial smoothness between adjacent
disparities. For the local matching cost, you can use the same
sum-of-squared-differences as above. For the pairwise cost, you can use
L1 distance. In your PDF, include both code and the estimated disparity
map.

\textbf{Hint}: In your algorithms course, you should have covered the
Viterbi algorithm. Apply it to this problem. Write out the objective
function as a recursive function. As you compute the recurrence, keep
track of the pointers to the previous iteration. After you finish the
recursive computation, just walk backwards using the stored pointers.

\textbf{Hint Two}: To help you debug, you can change the strength of the
pairwise term. If you turn off the pairwise term, your implementation
should recover the same solution as the greedy approach. If you make the
strength of the pairwise term very high, your implementation should
assign pixels to be identical disparity.

\textbf{Accuracy}: It is OK if you do not get perfect stereo
reconstruction. Carl's solution gets it mostly right except for the
lamp, which fails due to specular reflection.

    \begin{Verbatim}[commandchars=\\\{\}]
{\color{incolor}In [{\color{incolor}5}]:} \PY{c+c1}{\PYZsh{} Some variables that might help you. But you don\PYZsq{}t have to use them.}
        \PY{n}{max\PYZus{}dx} \PY{o}{=} \PY{l+m+mi}{30} \PY{c+c1}{\PYZsh{} max number of displacements to search. make this smaller to speed up}
        \PY{n}{dxs} \PY{o}{=} \PY{n}{np}\PY{o}{.}\PY{n}{linspace}\PY{p}{(}\PY{o}{\PYZhy{}}\PY{n}{max\PYZus{}dx}\PY{p}{,} \PY{n}{max\PYZus{}dx}\PY{p}{,} \PY{n}{num}\PY{o}{=}\PY{l+m+mi}{2}\PY{o}{*}\PY{n}{max\PYZus{}dx}\PY{o}{+}\PY{l+m+mi}{1}\PY{p}{)}\PY{o}{.}\PY{n}{astype}\PY{p}{(}\PY{l+s+s2}{\PYZdq{}}\PY{l+s+s2}{int32}\PY{l+s+s2}{\PYZdq{}}\PY{p}{)}
        
        \PY{n}{win\PYZus{}size} \PY{o}{=} \PY{l+m+mi}{5} \PY{c+c1}{\PYZsh{} size of window you use for ssd computation}
        \PY{n}{lamb\PYZus{}coeff} \PY{o}{=} \PY{l+m+mi}{1} \PY{c+c1}{\PYZsh{} coefficient in front of pairwise cost}
        
        \PY{k}{def} \PY{n+nf}{l1\PYZus{}score}\PY{p}{(}\PY{n}{a}\PY{p}{,}\PY{n}{b}\PY{p}{)}\PY{p}{:}
            \PY{c+c1}{\PYZsh{} TODO: implement a score for L1 distance}
            \PY{k}{return} \PY{n}{np}\PY{o}{.}\PY{n}{abs}\PY{p}{(}\PY{n}{a}\PY{o}{\PYZhy{}}\PY{n}{b}\PY{p}{)}
        
        \PY{n}{half\PYZus{}win\PYZus{}size} \PY{o}{=} \PY{n}{win\PYZus{}size} \PY{o}{/}\PY{o}{/} \PY{l+m+mi}{2}
        
        \PY{k}{def} \PY{n+nf}{compute\PYZus{}ssd\PYZus{}by\PYZus{}index}\PY{p}{(}\PY{n}{y}\PY{p}{,} \PY{n}{x}\PY{p}{,} \PY{n}{prev\PYZus{}ssd}\PY{o}{=}\PY{k+kc}{None}\PY{p}{)}\PY{p}{:}
            \PY{c+c1}{\PYZsh{} compute ssd given location parameters and offset}
            \PY{n}{left\PYZus{}temp} \PY{o}{=} \PY{n}{left\PYZus{}pad}\PY{p}{[}\PY{n}{pad\PYZus{}size}\PY{o}{+}\PY{n}{y}\PY{o}{\PYZhy{}}\PY{n}{half\PYZus{}win\PYZus{}size}\PY{p}{:}\PY{n}{pad\PYZus{}size}\PY{o}{+}\PY{n}{y}\PY{o}{+}\PY{n}{half\PYZus{}win\PYZus{}size}\PY{o}{+}\PY{l+m+mi}{1}\PY{p}{,}\PY{n}{pad\PYZus{}size}\PY{o}{+}\PY{n}{x}\PY{o}{\PYZhy{}}\PY{n}{half\PYZus{}win\PYZus{}size}\PY{p}{:}\PY{n}{pad\PYZus{}size}\PY{o}{+}\PY{n}{x}\PY{o}{+}\PY{n}{half\PYZus{}win\PYZus{}size}\PY{o}{+}\PY{l+m+mi}{1}\PY{p}{]}
            \PY{n}{ssd} \PY{o}{=} \PY{p}{[}\PY{p}{]}
            
            \PY{k}{if} \PY{n}{x} \PY{o}{!=} \PY{l+m+mi}{0}\PY{p}{:}
                \PY{k}{for} \PY{n}{offset} \PY{o+ow}{in} \PY{n}{dxs}\PY{p}{:}
                    \PY{n}{right\PYZus{}temp} \PY{o}{=} \PY{n}{right\PYZus{}pad}\PY{p}{[}\PY{n}{pad\PYZus{}size}\PY{o}{+}\PY{n}{y}\PY{o}{\PYZhy{}}\PY{n}{half\PYZus{}win\PYZus{}size}\PY{p}{:}\PY{n}{pad\PYZus{}size}\PY{o}{+}\PY{n}{y}\PY{o}{+}\PY{n}{half\PYZus{}win\PYZus{}size}\PY{o}{+}\PY{l+m+mi}{1}\PY{p}{,}\PY{n}{pad\PYZus{}size}\PY{o}{+}\PY{n}{x}\PY{o}{\PYZhy{}}\PY{n}{half\PYZus{}win\PYZus{}size}\PY{o}{+}\PY{n}{offset}\PY{p}{:}\PY{n}{pad\PYZus{}size}\PY{o}{+}\PY{n}{x}\PY{o}{+}\PY{n}{half\PYZus{}win\PYZus{}size}\PY{o}{+}\PY{l+m+mi}{1}\PY{o}{+}\PY{n}{offset}\PY{p}{]}
                    \PY{n}{ssd\PYZus{}temp} \PY{o}{=} \PY{n}{compute\PYZus{}ssd}\PY{p}{(}\PY{n}{left\PYZus{}temp}\PY{p}{,} \PY{n}{right\PYZus{}temp}\PY{p}{)}
                    \PY{n}{new\PYZus{}offset} \PY{o}{=} \PY{n}{lamb\PYZus{}coeff} \PY{o}{*} \PY{n}{np}\PY{o}{.}\PY{n}{abs}\PY{p}{(}\PY{n}{dxs} \PY{o}{\PYZhy{}} \PY{n}{offset}\PY{p}{)} \PY{o}{+} \PY{n}{prev\PYZus{}ssd}
                    \PY{n}{ssd\PYZus{}temp} \PY{o}{+}\PY{o}{=} \PY{n}{np}\PY{o}{.}\PY{n}{min}\PY{p}{(}\PY{n}{new\PYZus{}offset}\PY{p}{)}
                    \PY{n}{ssd}\PY{o}{.}\PY{n}{append}\PY{p}{(}\PY{n}{ssd\PYZus{}temp}\PY{p}{)}
        
            \PY{k}{else}\PY{p}{:}
                \PY{k}{for} \PY{n}{offset} \PY{o+ow}{in} \PY{n}{dxs}\PY{p}{:}
                    \PY{n}{right\PYZus{}temp} \PY{o}{=} \PY{n}{right\PYZus{}pad}\PY{p}{[}\PY{n}{pad\PYZus{}size}\PY{o}{+}\PY{n}{y}\PY{o}{\PYZhy{}}\PY{n}{half\PYZus{}win\PYZus{}size}\PY{p}{:}\PY{n}{pad\PYZus{}size}\PY{o}{+}\PY{n}{y}\PY{o}{+}\PY{n}{half\PYZus{}win\PYZus{}size}\PY{o}{+}\PY{l+m+mi}{1}\PY{p}{,}\PY{n}{pad\PYZus{}size}\PY{o}{+}\PY{n}{x}\PY{o}{\PYZhy{}}\PY{n}{half\PYZus{}win\PYZus{}size}\PY{o}{+}\PY{n}{offset}\PY{p}{:}\PY{n}{pad\PYZus{}size}\PY{o}{+}\PY{n}{x}\PY{o}{+}\PY{n}{half\PYZus{}win\PYZus{}size}\PY{o}{+}\PY{l+m+mi}{1}\PY{o}{+}\PY{n}{offset}\PY{p}{]}
                    \PY{n}{ssd\PYZus{}temp} \PY{o}{=} \PY{n}{compute\PYZus{}ssd}\PY{p}{(}\PY{n}{left\PYZus{}temp}\PY{p}{,} \PY{n}{right\PYZus{}temp}\PY{p}{)}
        \PY{c+c1}{\PYZsh{}             new\PYZus{}offset = lamb\PYZus{}coeff * np.abs(dxs \PYZhy{} offset) + prev\PYZus{}ssd}
        \PY{c+c1}{\PYZsh{}             ssd\PYZus{}temp += np.min(new\PYZus{}offset)}
                    \PY{n}{ssd}\PY{o}{.}\PY{n}{append}\PY{p}{(}\PY{n}{ssd\PYZus{}temp}\PY{p}{)}
             
            \PY{k}{return} \PY{n}{np}\PY{o}{.}\PY{n}{array}\PY{p}{(}\PY{n}{ssd}\PY{p}{)}
        
        \PY{k}{def} \PY{n+nf}{compute\PYZus{}disparity}\PY{p}{(}\PY{p}{)}\PY{p}{:}
            \PY{c+c1}{\PYZsh{} TODO: calculate the disparity using dynamic programming between left and right}
            \PY{n}{disparity} \PY{o}{=} \PY{n}{np}\PY{o}{.}\PY{n}{zeros}\PY{p}{(}\PY{p}{(}\PY{n}{height}\PY{p}{,}\PY{n}{width}\PY{p}{)}\PY{p}{)}
            \PY{k}{for} \PY{n}{y} \PY{o+ow}{in} \PY{n+nb}{range}\PY{p}{(}\PY{n}{height}\PY{p}{)}\PY{p}{:}
                \PY{k}{for} \PY{n}{x} \PY{o+ow}{in} \PY{n+nb}{range}\PY{p}{(}\PY{n}{width}\PY{p}{)}\PY{p}{:}
                    \PY{k}{if} \PY{n}{x} \PY{o}{==} \PY{l+m+mi}{0}\PY{p}{:}   
                        \PY{n}{ssd} \PY{o}{=} \PY{n}{compute\PYZus{}ssd\PYZus{}by\PYZus{}index}\PY{p}{(}\PY{n}{y}\PY{p}{,}\PY{n}{x}\PY{p}{)}
                        \PY{n}{disparity}\PY{p}{[}\PY{n}{y}\PY{p}{,}\PY{n}{x}\PY{p}{]} \PY{o}{=} \PY{n}{dxs}\PY{p}{[}\PY{n}{np}\PY{o}{.}\PY{n}{argmin}\PY{p}{(}\PY{n}{ssd}\PY{p}{)}\PY{p}{]}
                        \PY{n}{prev\PYZus{}ssd} \PY{o}{=} \PY{n}{ssd}
                    \PY{k}{else}\PY{p}{:}
                        \PY{n}{ssd} \PY{o}{=} \PY{n}{compute\PYZus{}ssd\PYZus{}by\PYZus{}index}\PY{p}{(}\PY{n}{y}\PY{p}{,}\PY{n}{x}\PY{p}{,}\PY{n}{prev\PYZus{}ssd}\PY{p}{)}
                        \PY{n}{disparity}\PY{p}{[}\PY{n}{y}\PY{p}{,}\PY{n}{x}\PY{p}{]} \PY{o}{=} \PY{n}{dxs}\PY{p}{[}\PY{n}{np}\PY{o}{.}\PY{n}{argmin}\PY{p}{(}\PY{n}{ssd}\PY{p}{)}\PY{p}{]}
                        \PY{n}{prev\PYZus{}ssd} \PY{o}{=} \PY{n}{ssd}
                        
            
            \PY{k}{return} \PY{n}{disparity}
        
        \PY{n}{disparity} \PY{o}{=} \PY{n}{compute\PYZus{}disparity}\PY{p}{(}\PY{p}{)}
        \PY{n}{plt}\PY{o}{.}\PY{n}{matshow}\PY{p}{(}\PY{n}{disparity}\PY{p}{)}
\end{Verbatim}


\begin{Verbatim}[commandchars=\\\{\}]
{\color{outcolor}Out[{\color{outcolor}5}]:} <matplotlib.image.AxesImage at 0x1ffd63304a8>
\end{Verbatim}
            
    \begin{center}
    \adjustimage{max size={0.9\linewidth}{0.9\paperheight}}{output_8_1.png}
    \end{center}
    { \hspace*{\fill} \\}
    
    \subsection{Deliverables}\label{deliverables}

Export your completed notebook as a PDF. Make sure the PDF includes both
the code and the resulting output. We will be grading both your code and
final output. (Runtime will not impact your score.)


    % Add a bibliography block to the postdoc
    
    
    
    \end{document}
