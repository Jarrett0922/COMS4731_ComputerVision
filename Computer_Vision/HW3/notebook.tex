
% Default to the notebook output style

    


% Inherit from the specified cell style.




    
\documentclass[11pt]{article}

    
    
    \usepackage[T1]{fontenc}
    % Nicer default font (+ math font) than Computer Modern for most use cases
    \usepackage{mathpazo}

    % Basic figure setup, for now with no caption control since it's done
    % automatically by Pandoc (which extracts ![](path) syntax from Markdown).
    \usepackage{graphicx}
    % We will generate all images so they have a width \maxwidth. This means
    % that they will get their normal width if they fit onto the page, but
    % are scaled down if they would overflow the margins.
    \makeatletter
    \def\maxwidth{\ifdim\Gin@nat@width>\linewidth\linewidth
    \else\Gin@nat@width\fi}
    \makeatother
    \let\Oldincludegraphics\includegraphics
    % Set max figure width to be 80% of text width, for now hardcoded.
    \renewcommand{\includegraphics}[1]{\Oldincludegraphics[width=.8\maxwidth]{#1}}
    % Ensure that by default, figures have no caption (until we provide a
    % proper Figure object with a Caption API and a way to capture that
    % in the conversion process - todo).
    \usepackage{caption}
    \DeclareCaptionLabelFormat{nolabel}{}
    \captionsetup{labelformat=nolabel}

    \usepackage{adjustbox} % Used to constrain images to a maximum size 
    \usepackage{xcolor} % Allow colors to be defined
    \usepackage{enumerate} % Needed for markdown enumerations to work
    \usepackage{geometry} % Used to adjust the document margins
    \usepackage{amsmath} % Equations
    \usepackage{amssymb} % Equations
    \usepackage{textcomp} % defines textquotesingle
    % Hack from http://tex.stackexchange.com/a/47451/13684:
    \AtBeginDocument{%
        \def\PYZsq{\textquotesingle}% Upright quotes in Pygmentized code
    }
    \usepackage{upquote} % Upright quotes for verbatim code
    \usepackage{eurosym} % defines \euro
    \usepackage[mathletters]{ucs} % Extended unicode (utf-8) support
    \usepackage[utf8x]{inputenc} % Allow utf-8 characters in the tex document
    \usepackage{fancyvrb} % verbatim replacement that allows latex
    \usepackage{grffile} % extends the file name processing of package graphics 
                         % to support a larger range 
    % The hyperref package gives us a pdf with properly built
    % internal navigation ('pdf bookmarks' for the table of contents,
    % internal cross-reference links, web links for URLs, etc.)
    \usepackage{hyperref}
    \usepackage{longtable} % longtable support required by pandoc >1.10
    \usepackage{booktabs}  % table support for pandoc > 1.12.2
    \usepackage[inline]{enumitem} % IRkernel/repr support (it uses the enumerate* environment)
    \usepackage[normalem]{ulem} % ulem is needed to support strikethroughs (\sout)
                                % normalem makes italics be italics, not underlines
    

    
    
    % Colors for the hyperref package
    \definecolor{urlcolor}{rgb}{0,.145,.698}
    \definecolor{linkcolor}{rgb}{.71,0.21,0.01}
    \definecolor{citecolor}{rgb}{.12,.54,.11}

    % ANSI colors
    \definecolor{ansi-black}{HTML}{3E424D}
    \definecolor{ansi-black-intense}{HTML}{282C36}
    \definecolor{ansi-red}{HTML}{E75C58}
    \definecolor{ansi-red-intense}{HTML}{B22B31}
    \definecolor{ansi-green}{HTML}{00A250}
    \definecolor{ansi-green-intense}{HTML}{007427}
    \definecolor{ansi-yellow}{HTML}{DDB62B}
    \definecolor{ansi-yellow-intense}{HTML}{B27D12}
    \definecolor{ansi-blue}{HTML}{208FFB}
    \definecolor{ansi-blue-intense}{HTML}{0065CA}
    \definecolor{ansi-magenta}{HTML}{D160C4}
    \definecolor{ansi-magenta-intense}{HTML}{A03196}
    \definecolor{ansi-cyan}{HTML}{60C6C8}
    \definecolor{ansi-cyan-intense}{HTML}{258F8F}
    \definecolor{ansi-white}{HTML}{C5C1B4}
    \definecolor{ansi-white-intense}{HTML}{A1A6B2}

    % commands and environments needed by pandoc snippets
    % extracted from the output of `pandoc -s`
    \providecommand{\tightlist}{%
      \setlength{\itemsep}{0pt}\setlength{\parskip}{0pt}}
    \DefineVerbatimEnvironment{Highlighting}{Verbatim}{commandchars=\\\{\}}
    % Add ',fontsize=\small' for more characters per line
    \newenvironment{Shaded}{}{}
    \newcommand{\KeywordTok}[1]{\textcolor[rgb]{0.00,0.44,0.13}{\textbf{{#1}}}}
    \newcommand{\DataTypeTok}[1]{\textcolor[rgb]{0.56,0.13,0.00}{{#1}}}
    \newcommand{\DecValTok}[1]{\textcolor[rgb]{0.25,0.63,0.44}{{#1}}}
    \newcommand{\BaseNTok}[1]{\textcolor[rgb]{0.25,0.63,0.44}{{#1}}}
    \newcommand{\FloatTok}[1]{\textcolor[rgb]{0.25,0.63,0.44}{{#1}}}
    \newcommand{\CharTok}[1]{\textcolor[rgb]{0.25,0.44,0.63}{{#1}}}
    \newcommand{\StringTok}[1]{\textcolor[rgb]{0.25,0.44,0.63}{{#1}}}
    \newcommand{\CommentTok}[1]{\textcolor[rgb]{0.38,0.63,0.69}{\textit{{#1}}}}
    \newcommand{\OtherTok}[1]{\textcolor[rgb]{0.00,0.44,0.13}{{#1}}}
    \newcommand{\AlertTok}[1]{\textcolor[rgb]{1.00,0.00,0.00}{\textbf{{#1}}}}
    \newcommand{\FunctionTok}[1]{\textcolor[rgb]{0.02,0.16,0.49}{{#1}}}
    \newcommand{\RegionMarkerTok}[1]{{#1}}
    \newcommand{\ErrorTok}[1]{\textcolor[rgb]{1.00,0.00,0.00}{\textbf{{#1}}}}
    \newcommand{\NormalTok}[1]{{#1}}
    
    % Additional commands for more recent versions of Pandoc
    \newcommand{\ConstantTok}[1]{\textcolor[rgb]{0.53,0.00,0.00}{{#1}}}
    \newcommand{\SpecialCharTok}[1]{\textcolor[rgb]{0.25,0.44,0.63}{{#1}}}
    \newcommand{\VerbatimStringTok}[1]{\textcolor[rgb]{0.25,0.44,0.63}{{#1}}}
    \newcommand{\SpecialStringTok}[1]{\textcolor[rgb]{0.73,0.40,0.53}{{#1}}}
    \newcommand{\ImportTok}[1]{{#1}}
    \newcommand{\DocumentationTok}[1]{\textcolor[rgb]{0.73,0.13,0.13}{\textit{{#1}}}}
    \newcommand{\AnnotationTok}[1]{\textcolor[rgb]{0.38,0.63,0.69}{\textbf{\textit{{#1}}}}}
    \newcommand{\CommentVarTok}[1]{\textcolor[rgb]{0.38,0.63,0.69}{\textbf{\textit{{#1}}}}}
    \newcommand{\VariableTok}[1]{\textcolor[rgb]{0.10,0.09,0.49}{{#1}}}
    \newcommand{\ControlFlowTok}[1]{\textcolor[rgb]{0.00,0.44,0.13}{\textbf{{#1}}}}
    \newcommand{\OperatorTok}[1]{\textcolor[rgb]{0.40,0.40,0.40}{{#1}}}
    \newcommand{\BuiltInTok}[1]{{#1}}
    \newcommand{\ExtensionTok}[1]{{#1}}
    \newcommand{\PreprocessorTok}[1]{\textcolor[rgb]{0.74,0.48,0.00}{{#1}}}
    \newcommand{\AttributeTok}[1]{\textcolor[rgb]{0.49,0.56,0.16}{{#1}}}
    \newcommand{\InformationTok}[1]{\textcolor[rgb]{0.38,0.63,0.69}{\textbf{\textit{{#1}}}}}
    \newcommand{\WarningTok}[1]{\textcolor[rgb]{0.38,0.63,0.69}{\textbf{\textit{{#1}}}}}
    
    
    % Define a nice break command that doesn't care if a line doesn't already
    % exist.
    \def\br{\hspace*{\fill} \\* }
    % Math Jax compatability definitions
    \def\gt{>}
    \def\lt{<}
    % Document parameters
    \title{Homework3}
    
    
    

    % Pygments definitions
    
\makeatletter
\def\PY@reset{\let\PY@it=\relax \let\PY@bf=\relax%
    \let\PY@ul=\relax \let\PY@tc=\relax%
    \let\PY@bc=\relax \let\PY@ff=\relax}
\def\PY@tok#1{\csname PY@tok@#1\endcsname}
\def\PY@toks#1+{\ifx\relax#1\empty\else%
    \PY@tok{#1}\expandafter\PY@toks\fi}
\def\PY@do#1{\PY@bc{\PY@tc{\PY@ul{%
    \PY@it{\PY@bf{\PY@ff{#1}}}}}}}
\def\PY#1#2{\PY@reset\PY@toks#1+\relax+\PY@do{#2}}

\expandafter\def\csname PY@tok@w\endcsname{\def\PY@tc##1{\textcolor[rgb]{0.73,0.73,0.73}{##1}}}
\expandafter\def\csname PY@tok@c\endcsname{\let\PY@it=\textit\def\PY@tc##1{\textcolor[rgb]{0.25,0.50,0.50}{##1}}}
\expandafter\def\csname PY@tok@cp\endcsname{\def\PY@tc##1{\textcolor[rgb]{0.74,0.48,0.00}{##1}}}
\expandafter\def\csname PY@tok@k\endcsname{\let\PY@bf=\textbf\def\PY@tc##1{\textcolor[rgb]{0.00,0.50,0.00}{##1}}}
\expandafter\def\csname PY@tok@kp\endcsname{\def\PY@tc##1{\textcolor[rgb]{0.00,0.50,0.00}{##1}}}
\expandafter\def\csname PY@tok@kt\endcsname{\def\PY@tc##1{\textcolor[rgb]{0.69,0.00,0.25}{##1}}}
\expandafter\def\csname PY@tok@o\endcsname{\def\PY@tc##1{\textcolor[rgb]{0.40,0.40,0.40}{##1}}}
\expandafter\def\csname PY@tok@ow\endcsname{\let\PY@bf=\textbf\def\PY@tc##1{\textcolor[rgb]{0.67,0.13,1.00}{##1}}}
\expandafter\def\csname PY@tok@nb\endcsname{\def\PY@tc##1{\textcolor[rgb]{0.00,0.50,0.00}{##1}}}
\expandafter\def\csname PY@tok@nf\endcsname{\def\PY@tc##1{\textcolor[rgb]{0.00,0.00,1.00}{##1}}}
\expandafter\def\csname PY@tok@nc\endcsname{\let\PY@bf=\textbf\def\PY@tc##1{\textcolor[rgb]{0.00,0.00,1.00}{##1}}}
\expandafter\def\csname PY@tok@nn\endcsname{\let\PY@bf=\textbf\def\PY@tc##1{\textcolor[rgb]{0.00,0.00,1.00}{##1}}}
\expandafter\def\csname PY@tok@ne\endcsname{\let\PY@bf=\textbf\def\PY@tc##1{\textcolor[rgb]{0.82,0.25,0.23}{##1}}}
\expandafter\def\csname PY@tok@nv\endcsname{\def\PY@tc##1{\textcolor[rgb]{0.10,0.09,0.49}{##1}}}
\expandafter\def\csname PY@tok@no\endcsname{\def\PY@tc##1{\textcolor[rgb]{0.53,0.00,0.00}{##1}}}
\expandafter\def\csname PY@tok@nl\endcsname{\def\PY@tc##1{\textcolor[rgb]{0.63,0.63,0.00}{##1}}}
\expandafter\def\csname PY@tok@ni\endcsname{\let\PY@bf=\textbf\def\PY@tc##1{\textcolor[rgb]{0.60,0.60,0.60}{##1}}}
\expandafter\def\csname PY@tok@na\endcsname{\def\PY@tc##1{\textcolor[rgb]{0.49,0.56,0.16}{##1}}}
\expandafter\def\csname PY@tok@nt\endcsname{\let\PY@bf=\textbf\def\PY@tc##1{\textcolor[rgb]{0.00,0.50,0.00}{##1}}}
\expandafter\def\csname PY@tok@nd\endcsname{\def\PY@tc##1{\textcolor[rgb]{0.67,0.13,1.00}{##1}}}
\expandafter\def\csname PY@tok@s\endcsname{\def\PY@tc##1{\textcolor[rgb]{0.73,0.13,0.13}{##1}}}
\expandafter\def\csname PY@tok@sd\endcsname{\let\PY@it=\textit\def\PY@tc##1{\textcolor[rgb]{0.73,0.13,0.13}{##1}}}
\expandafter\def\csname PY@tok@si\endcsname{\let\PY@bf=\textbf\def\PY@tc##1{\textcolor[rgb]{0.73,0.40,0.53}{##1}}}
\expandafter\def\csname PY@tok@se\endcsname{\let\PY@bf=\textbf\def\PY@tc##1{\textcolor[rgb]{0.73,0.40,0.13}{##1}}}
\expandafter\def\csname PY@tok@sr\endcsname{\def\PY@tc##1{\textcolor[rgb]{0.73,0.40,0.53}{##1}}}
\expandafter\def\csname PY@tok@ss\endcsname{\def\PY@tc##1{\textcolor[rgb]{0.10,0.09,0.49}{##1}}}
\expandafter\def\csname PY@tok@sx\endcsname{\def\PY@tc##1{\textcolor[rgb]{0.00,0.50,0.00}{##1}}}
\expandafter\def\csname PY@tok@m\endcsname{\def\PY@tc##1{\textcolor[rgb]{0.40,0.40,0.40}{##1}}}
\expandafter\def\csname PY@tok@gh\endcsname{\let\PY@bf=\textbf\def\PY@tc##1{\textcolor[rgb]{0.00,0.00,0.50}{##1}}}
\expandafter\def\csname PY@tok@gu\endcsname{\let\PY@bf=\textbf\def\PY@tc##1{\textcolor[rgb]{0.50,0.00,0.50}{##1}}}
\expandafter\def\csname PY@tok@gd\endcsname{\def\PY@tc##1{\textcolor[rgb]{0.63,0.00,0.00}{##1}}}
\expandafter\def\csname PY@tok@gi\endcsname{\def\PY@tc##1{\textcolor[rgb]{0.00,0.63,0.00}{##1}}}
\expandafter\def\csname PY@tok@gr\endcsname{\def\PY@tc##1{\textcolor[rgb]{1.00,0.00,0.00}{##1}}}
\expandafter\def\csname PY@tok@ge\endcsname{\let\PY@it=\textit}
\expandafter\def\csname PY@tok@gs\endcsname{\let\PY@bf=\textbf}
\expandafter\def\csname PY@tok@gp\endcsname{\let\PY@bf=\textbf\def\PY@tc##1{\textcolor[rgb]{0.00,0.00,0.50}{##1}}}
\expandafter\def\csname PY@tok@go\endcsname{\def\PY@tc##1{\textcolor[rgb]{0.53,0.53,0.53}{##1}}}
\expandafter\def\csname PY@tok@gt\endcsname{\def\PY@tc##1{\textcolor[rgb]{0.00,0.27,0.87}{##1}}}
\expandafter\def\csname PY@tok@err\endcsname{\def\PY@bc##1{\setlength{\fboxsep}{0pt}\fcolorbox[rgb]{1.00,0.00,0.00}{1,1,1}{\strut ##1}}}
\expandafter\def\csname PY@tok@kc\endcsname{\let\PY@bf=\textbf\def\PY@tc##1{\textcolor[rgb]{0.00,0.50,0.00}{##1}}}
\expandafter\def\csname PY@tok@kd\endcsname{\let\PY@bf=\textbf\def\PY@tc##1{\textcolor[rgb]{0.00,0.50,0.00}{##1}}}
\expandafter\def\csname PY@tok@kn\endcsname{\let\PY@bf=\textbf\def\PY@tc##1{\textcolor[rgb]{0.00,0.50,0.00}{##1}}}
\expandafter\def\csname PY@tok@kr\endcsname{\let\PY@bf=\textbf\def\PY@tc##1{\textcolor[rgb]{0.00,0.50,0.00}{##1}}}
\expandafter\def\csname PY@tok@bp\endcsname{\def\PY@tc##1{\textcolor[rgb]{0.00,0.50,0.00}{##1}}}
\expandafter\def\csname PY@tok@fm\endcsname{\def\PY@tc##1{\textcolor[rgb]{0.00,0.00,1.00}{##1}}}
\expandafter\def\csname PY@tok@vc\endcsname{\def\PY@tc##1{\textcolor[rgb]{0.10,0.09,0.49}{##1}}}
\expandafter\def\csname PY@tok@vg\endcsname{\def\PY@tc##1{\textcolor[rgb]{0.10,0.09,0.49}{##1}}}
\expandafter\def\csname PY@tok@vi\endcsname{\def\PY@tc##1{\textcolor[rgb]{0.10,0.09,0.49}{##1}}}
\expandafter\def\csname PY@tok@vm\endcsname{\def\PY@tc##1{\textcolor[rgb]{0.10,0.09,0.49}{##1}}}
\expandafter\def\csname PY@tok@sa\endcsname{\def\PY@tc##1{\textcolor[rgb]{0.73,0.13,0.13}{##1}}}
\expandafter\def\csname PY@tok@sb\endcsname{\def\PY@tc##1{\textcolor[rgb]{0.73,0.13,0.13}{##1}}}
\expandafter\def\csname PY@tok@sc\endcsname{\def\PY@tc##1{\textcolor[rgb]{0.73,0.13,0.13}{##1}}}
\expandafter\def\csname PY@tok@dl\endcsname{\def\PY@tc##1{\textcolor[rgb]{0.73,0.13,0.13}{##1}}}
\expandafter\def\csname PY@tok@s2\endcsname{\def\PY@tc##1{\textcolor[rgb]{0.73,0.13,0.13}{##1}}}
\expandafter\def\csname PY@tok@sh\endcsname{\def\PY@tc##1{\textcolor[rgb]{0.73,0.13,0.13}{##1}}}
\expandafter\def\csname PY@tok@s1\endcsname{\def\PY@tc##1{\textcolor[rgb]{0.73,0.13,0.13}{##1}}}
\expandafter\def\csname PY@tok@mb\endcsname{\def\PY@tc##1{\textcolor[rgb]{0.40,0.40,0.40}{##1}}}
\expandafter\def\csname PY@tok@mf\endcsname{\def\PY@tc##1{\textcolor[rgb]{0.40,0.40,0.40}{##1}}}
\expandafter\def\csname PY@tok@mh\endcsname{\def\PY@tc##1{\textcolor[rgb]{0.40,0.40,0.40}{##1}}}
\expandafter\def\csname PY@tok@mi\endcsname{\def\PY@tc##1{\textcolor[rgb]{0.40,0.40,0.40}{##1}}}
\expandafter\def\csname PY@tok@il\endcsname{\def\PY@tc##1{\textcolor[rgb]{0.40,0.40,0.40}{##1}}}
\expandafter\def\csname PY@tok@mo\endcsname{\def\PY@tc##1{\textcolor[rgb]{0.40,0.40,0.40}{##1}}}
\expandafter\def\csname PY@tok@ch\endcsname{\let\PY@it=\textit\def\PY@tc##1{\textcolor[rgb]{0.25,0.50,0.50}{##1}}}
\expandafter\def\csname PY@tok@cm\endcsname{\let\PY@it=\textit\def\PY@tc##1{\textcolor[rgb]{0.25,0.50,0.50}{##1}}}
\expandafter\def\csname PY@tok@cpf\endcsname{\let\PY@it=\textit\def\PY@tc##1{\textcolor[rgb]{0.25,0.50,0.50}{##1}}}
\expandafter\def\csname PY@tok@c1\endcsname{\let\PY@it=\textit\def\PY@tc##1{\textcolor[rgb]{0.25,0.50,0.50}{##1}}}
\expandafter\def\csname PY@tok@cs\endcsname{\let\PY@it=\textit\def\PY@tc##1{\textcolor[rgb]{0.25,0.50,0.50}{##1}}}

\def\PYZbs{\char`\\}
\def\PYZus{\char`\_}
\def\PYZob{\char`\{}
\def\PYZcb{\char`\}}
\def\PYZca{\char`\^}
\def\PYZam{\char`\&}
\def\PYZlt{\char`\<}
\def\PYZgt{\char`\>}
\def\PYZsh{\char`\#}
\def\PYZpc{\char`\%}
\def\PYZdl{\char`\$}
\def\PYZhy{\char`\-}
\def\PYZsq{\char`\'}
\def\PYZdq{\char`\"}
\def\PYZti{\char`\~}
% for compatibility with earlier versions
\def\PYZat{@}
\def\PYZlb{[}
\def\PYZrb{]}
\makeatother


    % Exact colors from NB
    \definecolor{incolor}{rgb}{0.0, 0.0, 0.5}
    \definecolor{outcolor}{rgb}{0.545, 0.0, 0.0}



    
    % Prevent overflowing lines due to hard-to-break entities
    \sloppy 
    % Setup hyperref package
    \hypersetup{
      breaklinks=true,  % so long urls are correctly broken across lines
      colorlinks=true,
      urlcolor=urlcolor,
      linkcolor=linkcolor,
      citecolor=citecolor,
      }
    % Slightly bigger margins than the latex defaults
    
    \geometry{verbose,tmargin=1in,bmargin=1in,lmargin=1in,rmargin=1in}
    
    

    \begin{document}
    
    
    \maketitle
    
    

    
    \section{COMS 4731 Computer Vision -\/- Homework
3}\label{coms-4731-computer-vision----homework-3}

\begin{itemize}
\item
  In this homework, you will construct a panorama by stitching several
  individual and overlapping images together.

  \begin{itemize}
  \tightlist
  \item
    \textbf{Problem 1: Homography (20 points)}

    \begin{itemize}
    \tightlist
    \item
      Implement the \texttt{compute\_homography} function.
    \item
      Implement the \texttt{apply\_homography} function.
    \end{itemize}
  \item
    \textbf{Problem 2: Warping (20 points)}

    \begin{itemize}
    \tightlist
    \item
      Implement the \texttt{backward\_warp\_img} function.
    \end{itemize}
  \item
    \textbf{Problem 3: SIFT and RANSAC (20 points)}

    \begin{itemize}
    \tightlist
    \item
      Implement the \texttt{RANSAC} function.
    \end{itemize}
  \item
    \textbf{Problem 4: Image Blending (20 points)}

    \begin{itemize}
    \tightlist
    \item
      Implement the \texttt{blend\_image\_pair} function.
    \end{itemize}
  \item
    \textbf{Problem 5: Creating Panoramas (20 points)}

    \begin{itemize}
    \tightlist
    \item
      Implement the \texttt{stitch\_img} function.
    \item
      Create a panorama using your own photos.
    \end{itemize}
  \end{itemize}
\item
  Your job is to implement the sections marked with TODO to complete the
  tasks.
\item
  Submission

  \begin{itemize}
  \tightlist
  \item
    Please submit the notebook (ipynb and pdf) including the output of
    all cells.
  \end{itemize}
\item
  Note: Please install OpenCV (version 3.4.2.16) by running the
  following command in the terminal

  \begin{itemize}
  \tightlist
  \item
    \texttt{pip\ install\ opencv-python==3.4.2.16;\ pip\ install\ opencv-contrib-python==3.4.2.16}
  \item
    Otherwise, you may encounter error when running SIFT.
  \end{itemize}
\end{itemize}

    \subsection{Setup}\label{setup}

Before we get started, let's visualize the three separate images we
ultimately want to stitch together.

    \begin{Verbatim}[commandchars=\\\{\}]
{\color{incolor}In [{\color{incolor}1}]:} \PY{k+kn}{import} \PY{n+nn}{numpy} \PY{k}{as} \PY{n+nn}{np}
        \PY{k+kn}{import} \PY{n+nn}{matplotlib}\PY{n+nn}{.}\PY{n+nn}{pyplot} \PY{k}{as} \PY{n+nn}{plt}
        \PY{k+kn}{from} \PY{n+nn}{PIL} \PY{k}{import} \PY{n}{Image}
        \PY{k+kn}{from} \PY{n+nn}{IPython} \PY{k}{import} \PY{n}{display}
        \PY{k+kn}{import} \PY{n+nn}{sys}
        \PY{k+kn}{import} \PY{n+nn}{random}
        \PY{k+kn}{import} \PY{n+nn}{cv2}
        \PY{o}{\PYZpc{}}\PY{k}{matplotlib} inline
\end{Verbatim}


    \begin{Verbatim}[commandchars=\\\{\}]
{\color{incolor}In [{\color{incolor}2}]:} \PY{n}{plt}\PY{o}{.}\PY{n}{rcParams}\PY{p}{[}\PY{l+s+s1}{\PYZsq{}}\PY{l+s+s1}{figure.figsize}\PY{l+s+s1}{\PYZsq{}}\PY{p}{]} \PY{o}{=} \PY{p}{[}\PY{l+m+mi}{15}\PY{p}{,} \PY{l+m+mi}{15}\PY{p}{]}
        
        \PY{k}{def} \PY{n+nf}{load\PYZus{}image}\PY{p}{(}\PY{n}{filename}\PY{p}{)}\PY{p}{:}
            \PY{n}{img} \PY{o}{=} \PY{n}{np}\PY{o}{.}\PY{n}{asarray}\PY{p}{(}\PY{n}{Image}\PY{o}{.}\PY{n}{open}\PY{p}{(}\PY{n}{filename}\PY{p}{)}\PY{p}{)}
            \PY{n}{img} \PY{o}{=} \PY{n}{img}\PY{o}{.}\PY{n}{astype}\PY{p}{(}\PY{l+s+s2}{\PYZdq{}}\PY{l+s+s2}{float32}\PY{l+s+s2}{\PYZdq{}}\PY{p}{)}\PY{o}{/}\PY{l+m+mf}{255.}
            \PY{k}{return} \PY{n}{img}
        
        \PY{k}{def} \PY{n+nf}{show\PYZus{}image}\PY{p}{(}\PY{n}{img}\PY{p}{)}\PY{p}{:}
            \PY{n}{plt}\PY{o}{.}\PY{n}{imshow}\PY{p}{(}\PY{n}{img}\PY{p}{,} \PY{n}{interpolation}\PY{o}{=}\PY{l+s+s1}{\PYZsq{}}\PY{l+s+s1}{nearest}\PY{l+s+s1}{\PYZsq{}}\PY{p}{)}
            
        \PY{n}{center\PYZus{}img} \PY{o}{=} \PY{n}{load\PYZus{}image}\PY{p}{(}\PY{l+s+s2}{\PYZdq{}}\PY{l+s+s2}{mountain\PYZus{}center.png}\PY{l+s+s2}{\PYZdq{}}\PY{p}{)}
        \PY{n}{left\PYZus{}img}   \PY{o}{=} \PY{n}{load\PYZus{}image}\PY{p}{(}\PY{l+s+s2}{\PYZdq{}}\PY{l+s+s2}{mountain\PYZus{}left.png}\PY{l+s+s2}{\PYZdq{}}\PY{p}{)}
        \PY{n}{right\PYZus{}img}  \PY{o}{=} \PY{n}{load\PYZus{}image}\PY{p}{(}\PY{l+s+s2}{\PYZdq{}}\PY{l+s+s2}{mountain\PYZus{}right.png}\PY{l+s+s2}{\PYZdq{}}\PY{p}{)}
        
        \PY{n}{show\PYZus{}image}\PY{p}{(}\PY{n}{np}\PY{o}{.}\PY{n}{concatenate}\PY{p}{(}\PY{p}{[}\PY{n}{left\PYZus{}img}\PY{p}{,} \PY{n}{center\PYZus{}img}\PY{p}{,} \PY{n}{right\PYZus{}img}\PY{p}{]}\PY{p}{,} \PY{n}{axis}\PY{o}{=}\PY{l+m+mi}{1}\PY{p}{)}\PY{p}{)}
\end{Verbatim}


    \begin{center}
    \adjustimage{max size={0.9\linewidth}{0.9\paperheight}}{output_3_0.png}
    \end{center}
    { \hspace*{\fill} \\}
    
    \section{Problem 1: Homography}\label{problem-1-homography}

You should finish implementing two functions below:

\begin{enumerate}
\def\labelenumi{\arabic{enumi}.}
\item
  \textbf{compute\_homography(src, dst)} receives two matrices of
  points, which are each Nx2. The function should return the homography
  matrix H that maps points from the source to the target. This return
  value should be a 3x3 matrix. We have given you most of the solution
  already. You just need to implement the A matrix.
\item
  \textbf{apply\_homography(src, H)} receives points src (Nx2 matrix)
  and the homography transformation H (3x3 matrix). This function should
  use the homography matrix to transform src into the destination.
  Remember that you need to implement this using homogenous coordinates.
\end{enumerate}

    \begin{Verbatim}[commandchars=\\\{\}]
{\color{incolor}In [{\color{incolor}3}]:} \PY{k}{def} \PY{n+nf}{compute\PYZus{}homography}\PY{p}{(}\PY{n}{src}\PY{p}{,} \PY{n}{dst}\PY{p}{)}\PY{p}{:}
            \PY{l+s+sd}{\PYZsq{}\PYZsq{}\PYZsq{}Computes the homography from src to dst.}
        \PY{l+s+sd}{    }
        \PY{l+s+sd}{    Input:}
        \PY{l+s+sd}{        src: source points, shape (n, 2)}
        \PY{l+s+sd}{        dst: destination points, shape (n, 2)}
        \PY{l+s+sd}{    Output:}
        \PY{l+s+sd}{        H: homography from source points to destination points, shape (3, 3)}
        \PY{l+s+sd}{        }
        \PY{l+s+sd}{    TODO: Implement the A matrix. }
        \PY{l+s+sd}{    \PYZsq{}\PYZsq{}\PYZsq{}}
            
            \PY{n}{A} \PY{o}{=} \PY{n}{np}\PY{o}{.}\PY{n}{zeros}\PY{p}{(}\PY{p}{[}\PY{l+m+mi}{2}\PY{o}{*}\PY{n}{src}\PY{o}{.}\PY{n}{shape}\PY{p}{[}\PY{l+m+mi}{0}\PY{p}{]}\PY{p}{,} \PY{l+m+mi}{9}\PY{p}{]}\PY{p}{)}
            \PY{c+c1}{\PYZsh{} Your code here.}
            \PY{k}{for} \PY{n}{i} \PY{o+ow}{in} \PY{n+nb}{range}\PY{p}{(}\PY{n}{src}\PY{o}{.}\PY{n}{shape}\PY{p}{[}\PY{l+m+mi}{0}\PY{p}{]}\PY{p}{)}\PY{p}{:}
                \PY{n}{A}\PY{p}{[}\PY{l+m+mi}{2}\PY{o}{*}\PY{n}{i}\PY{p}{,}\PY{p}{:}\PY{p}{]} \PY{o}{=} \PY{n}{np}\PY{o}{.}\PY{n}{array}\PY{p}{(}\PY{p}{[}\PY{n}{src}\PY{p}{[}\PY{n}{i}\PY{p}{,}\PY{l+m+mi}{0}\PY{p}{]}\PY{p}{,} \PY{n}{src}\PY{p}{[}\PY{n}{i}\PY{p}{,}\PY{l+m+mi}{1}\PY{p}{]}\PY{p}{,} \PY{l+m+mi}{1}\PY{p}{,} \PY{l+m+mi}{0}\PY{p}{,} \PY{l+m+mi}{0}\PY{p}{,} \PY{l+m+mi}{0}\PY{p}{,} \PY{o}{\PYZhy{}}\PY{n}{dst}\PY{p}{[}\PY{n}{i}\PY{p}{,}\PY{l+m+mi}{0}\PY{p}{]}\PY{o}{*}\PY{n}{src}\PY{p}{[}\PY{n}{i}\PY{p}{,}\PY{l+m+mi}{0}\PY{p}{]}\PY{p}{,} \PY{o}{\PYZhy{}}\PY{n}{dst}\PY{p}{[}\PY{n}{i}\PY{p}{,}\PY{l+m+mi}{0}\PY{p}{]}\PY{o}{*}\PY{n}{src}\PY{p}{[}\PY{n}{i}\PY{p}{,}\PY{l+m+mi}{1}\PY{p}{]}\PY{p}{,} \PY{o}{\PYZhy{}}\PY{n}{dst}\PY{p}{[}\PY{n}{i}\PY{p}{,}\PY{l+m+mi}{0}\PY{p}{]}\PY{p}{]}\PY{p}{)}
                \PY{n}{A}\PY{p}{[}\PY{l+m+mi}{2}\PY{o}{*}\PY{n}{i}\PY{o}{+}\PY{l+m+mi}{1}\PY{p}{,}\PY{p}{:}\PY{p}{]} \PY{o}{=} \PY{n}{np}\PY{o}{.}\PY{n}{array}\PY{p}{(}\PY{p}{[}\PY{l+m+mi}{0}\PY{p}{,} \PY{l+m+mi}{0}\PY{p}{,} \PY{l+m+mi}{0}\PY{p}{,} \PY{n}{src}\PY{p}{[}\PY{n}{i}\PY{p}{,}\PY{l+m+mi}{0}\PY{p}{]}\PY{p}{,} \PY{n}{src}\PY{p}{[}\PY{n}{i}\PY{p}{,}\PY{l+m+mi}{1}\PY{p}{]}\PY{p}{,} \PY{l+m+mi}{1}\PY{p}{,} \PY{o}{\PYZhy{}}\PY{n}{dst}\PY{p}{[}\PY{n}{i}\PY{p}{,}\PY{l+m+mi}{1}\PY{p}{]}\PY{o}{*}\PY{n}{src}\PY{p}{[}\PY{n}{i}\PY{p}{,}\PY{l+m+mi}{0}\PY{p}{]}\PY{p}{,} \PY{o}{\PYZhy{}}\PY{n}{dst}\PY{p}{[}\PY{n}{i}\PY{p}{,}\PY{l+m+mi}{1}\PY{p}{]}\PY{o}{*}\PY{n}{src}\PY{p}{[}\PY{n}{i}\PY{p}{,}\PY{l+m+mi}{1}\PY{p}{]}\PY{p}{,} \PY{o}{\PYZhy{}}\PY{n}{dst}\PY{p}{[}\PY{n}{i}\PY{p}{,}\PY{l+m+mi}{1}\PY{p}{]}\PY{p}{]}\PY{p}{)}
            
            \PY{n}{w}\PY{p}{,} \PY{n}{v} \PY{o}{=} \PY{n}{np}\PY{o}{.}\PY{n}{linalg}\PY{o}{.}\PY{n}{eig}\PY{p}{(}\PY{n}{np}\PY{o}{.}\PY{n}{dot}\PY{p}{(}\PY{n}{A}\PY{o}{.}\PY{n}{T}\PY{p}{,} \PY{n}{A}\PY{p}{)}\PY{p}{)}
            \PY{n}{index} \PY{o}{=} \PY{n}{np}\PY{o}{.}\PY{n}{argmin}\PY{p}{(}\PY{n}{w}\PY{p}{)}
            \PY{n}{H} \PY{o}{=} \PY{n}{v}\PY{p}{[}\PY{p}{:}\PY{p}{,} \PY{n}{index}\PY{p}{]}\PY{o}{.}\PY{n}{reshape}\PY{p}{(}\PY{p}{[}\PY{l+m+mi}{3}\PY{p}{,}\PY{l+m+mi}{3}\PY{p}{]}\PY{p}{)}
            \PY{k}{return} \PY{n}{H}
        
        \PY{k}{def} \PY{n+nf}{apply\PYZus{}homography}\PY{p}{(}\PY{n}{src}\PY{p}{,} \PY{n}{H}\PY{p}{)}\PY{p}{:}
            \PY{l+s+sd}{\PYZsq{}\PYZsq{}\PYZsq{}Applies a homography H onto the source points.}
        \PY{l+s+sd}{    }
        \PY{l+s+sd}{    Input:}
        \PY{l+s+sd}{        src: source points, shape (n, 2)}
        \PY{l+s+sd}{        H: homography from source points to destination points, shape (3, 3)}
        \PY{l+s+sd}{    Output:}
        \PY{l+s+sd}{        dst: destination points, shape (n, 2)}
        \PY{l+s+sd}{    }
        \PY{l+s+sd}{    TODO: Implement the apply\PYZus{}homography function}
        \PY{l+s+sd}{    \PYZsq{}\PYZsq{}\PYZsq{}}
            \PY{n}{ones} \PY{o}{=} \PY{n}{np}\PY{o}{.}\PY{n}{ones}\PY{p}{(}\PY{p}{(}\PY{n}{src}\PY{o}{.}\PY{n}{shape}\PY{p}{[}\PY{l+m+mi}{0}\PY{p}{]}\PY{p}{,}\PY{l+m+mi}{1}\PY{p}{)}\PY{p}{)}
            \PY{n}{src} \PY{o}{=} \PY{n}{np}\PY{o}{.}\PY{n}{hstack}\PY{p}{(}\PY{p}{[}\PY{n}{src}\PY{p}{,}\PY{n}{ones}\PY{p}{]}\PY{p}{)}
            \PY{n}{dst} \PY{o}{=} \PY{n}{np}\PY{o}{.}\PY{n}{dot}\PY{p}{(}\PY{n}{H}\PY{p}{,}\PY{n}{src}\PY{o}{.}\PY{n}{T}\PY{p}{)}
            \PY{n}{dst} \PY{o}{=} \PY{n}{dst} \PY{o}{/} \PY{n}{dst}\PY{p}{[}\PY{l+m+mi}{2}\PY{p}{]}
            
            \PY{k}{return} \PY{n}{dst}\PY{p}{[}\PY{p}{:}\PY{l+m+mi}{2}\PY{p}{,}\PY{p}{:}\PY{p}{]}\PY{o}{.}\PY{n}{T}
\end{Verbatim}


    To help you debug the homography code, we have provided a test below.
This uses pairs of points (src\_pts and dst\_pts) to compute the
homography. Then, it applies the homography on held-out points
(test\_pts), and visualizes the correspondence as red lines between the
two images. If you have correctly implemented compute\_homography() and
apply\_homography, the red lines should connect the same points in both
images.

    \begin{Verbatim}[commandchars=\\\{\}]
{\color{incolor}In [{\color{incolor}4}]:} \PY{k}{def} \PY{n+nf}{test\PYZus{}homography}\PY{p}{(}\PY{p}{)}\PY{p}{:}
            \PY{n}{src\PYZus{}img} \PY{o}{=} \PY{n}{load\PYZus{}image}\PY{p}{(}\PY{l+s+s1}{\PYZsq{}}\PY{l+s+s1}{portrait.png}\PY{l+s+s1}{\PYZsq{}}\PY{p}{)}\PY{p}{[}\PY{p}{:}\PY{p}{,} \PY{p}{:}\PY{p}{,} \PY{p}{:}\PY{l+m+mi}{3}\PY{p}{]}
            \PY{n}{dst\PYZus{}img} \PY{o}{=} \PY{n}{load\PYZus{}image}\PY{p}{(}\PY{l+s+s1}{\PYZsq{}}\PY{l+s+s1}{portrait\PYZus{}transformed.png}\PY{l+s+s1}{\PYZsq{}}\PY{p}{)}
            \PY{n}{whole\PYZus{}img} \PY{o}{=} \PY{n}{np}\PY{o}{.}\PY{n}{concatenate}\PY{p}{(}\PY{p}{(}\PY{n}{src\PYZus{}img}\PY{p}{,} \PY{n}{dst\PYZus{}img}\PY{p}{)}\PY{p}{,} \PY{n}{axis}\PY{o}{=}\PY{l+m+mi}{1}\PY{p}{)}
        
            \PY{n}{src\PYZus{}pts} \PY{o}{=} \PY{n}{np}\PY{o}{.}\PY{n}{matrix}\PY{p}{(}\PY{l+s+s1}{\PYZsq{}}\PY{l+s+s1}{347, 313; 502, 341; 386, 571; 621, 508}\PY{l+s+s1}{\PYZsq{}}\PY{p}{)}
            \PY{n}{dst\PYZus{}pts} \PY{o}{=} \PY{n}{np}\PY{o}{.}\PY{n}{matrix}\PY{p}{(}\PY{l+s+s1}{\PYZsq{}}\PY{l+s+s1}{274, 286; 436, 305; 305, 527; 615, 506}\PY{l+s+s1}{\PYZsq{}}\PY{p}{)}
            \PY{n}{H} \PY{o}{=} \PY{n}{compute\PYZus{}homography}\PY{p}{(}\PY{n}{src\PYZus{}pts}\PY{p}{,} \PY{n}{dst\PYZus{}pts}\PY{p}{)}
        
            \PY{n}{test\PYZus{}pts} \PY{o}{=} \PY{n}{np}\PY{o}{.}\PY{n}{matrix}\PY{p}{(}\PY{l+s+s1}{\PYZsq{}}\PY{l+s+s1}{259, 505; 350, 371; 400, 675; 636, 104}\PY{l+s+s1}{\PYZsq{}}\PY{p}{)}
            \PY{n}{match\PYZus{}pts} \PY{o}{=} \PY{n}{apply\PYZus{}homography}\PY{p}{(}\PY{n}{test\PYZus{}pts}\PY{p}{,} \PY{n}{H}\PY{p}{)}
        
            \PY{n}{match\PYZus{}pts\PYZus{}correct} \PY{o}{=} \PY{n}{np}\PY{o}{.}\PY{n}{matrix}\PY{p}{(}\PY{l+s+s1}{\PYZsq{}}\PY{l+s+s1}{195.13761083, 448.12645033;}\PY{l+s+s1}{\PYZsq{}}
                                          \PY{l+s+s1}{\PYZsq{}}\PY{l+s+s1}{275.27269386, 336.54819916;}\PY{l+s+s1}{\PYZsq{}}
                                          \PY{l+s+s1}{\PYZsq{}}\PY{l+s+s1}{317.37663747, 636.78403426;}\PY{l+s+s1}{\PYZsq{}}
                                          \PY{l+s+s1}{\PYZsq{}}\PY{l+s+s1}{618.50438823, 28.78963905}\PY{l+s+s1}{\PYZsq{}}\PY{p}{)}
        
            \PY{n+nb}{print}\PY{p}{(}\PY{l+s+s1}{\PYZsq{}}\PY{l+s+s1}{Your solution differs from our solution by: }\PY{l+s+si}{\PYZpc{}f}\PY{l+s+s1}{\PYZsq{}}
                  \PY{o}{\PYZpc{}} \PY{n}{np}\PY{o}{.}\PY{n}{square}\PY{p}{(}\PY{n}{match\PYZus{}pts} \PY{o}{\PYZhy{}} \PY{n}{match\PYZus{}pts\PYZus{}correct}\PY{p}{)}\PY{o}{.}\PY{n}{sum}\PY{p}{(}\PY{p}{)}\PY{p}{)}
        
            \PY{k}{for} \PY{n}{i} \PY{o+ow}{in} \PY{n+nb}{range}\PY{p}{(}\PY{n}{test\PYZus{}pts}\PY{o}{.}\PY{n}{shape}\PY{p}{[}\PY{l+m+mi}{0}\PY{p}{]}\PY{p}{)}\PY{p}{:}
                \PY{n}{test\PYZus{}x} \PY{o}{=} \PY{n}{test\PYZus{}pts}\PY{p}{[}\PY{n}{i}\PY{p}{,} \PY{l+m+mi}{0}\PY{p}{]}
                \PY{n}{test\PYZus{}y} \PY{o}{=} \PY{n}{test\PYZus{}pts}\PY{p}{[}\PY{n}{i}\PY{p}{,} \PY{l+m+mi}{1}\PY{p}{]}
                \PY{n}{match\PYZus{}x} \PY{o}{=} \PY{n+nb}{int}\PY{p}{(}\PY{n+nb}{round}\PY{p}{(}\PY{n}{match\PYZus{}pts}\PY{p}{[}\PY{n}{i}\PY{p}{,} \PY{l+m+mi}{0}\PY{p}{]} \PY{o}{+} \PY{l+m+mi}{800}\PY{p}{)}\PY{p}{)}
                \PY{n}{match\PYZus{}y} \PY{o}{=} \PY{n+nb}{int}\PY{p}{(}\PY{n+nb}{round}\PY{p}{(}\PY{n}{match\PYZus{}pts}\PY{p}{[}\PY{n}{i}\PY{p}{,} \PY{l+m+mi}{1}\PY{p}{]}\PY{p}{)}\PY{p}{)}
        
                \PY{n}{cv2}\PY{o}{.}\PY{n}{line}\PY{p}{(}\PY{n}{whole\PYZus{}img}\PY{p}{,}
                    \PY{p}{(}\PY{n}{test\PYZus{}x}\PY{p}{,} \PY{n}{test\PYZus{}y}\PY{p}{)}\PY{p}{,} 
                    \PY{p}{(}\PY{n}{match\PYZus{}x}\PY{p}{,} \PY{n}{match\PYZus{}y}\PY{p}{)}\PY{p}{,} 
                    \PY{p}{(}\PY{l+m+mi}{255}\PY{p}{,} \PY{l+m+mi}{0}\PY{p}{,} \PY{l+m+mi}{0}\PY{p}{)}\PY{p}{,} \PY{n}{thickness}\PY{o}{=}\PY{l+m+mi}{5}\PY{p}{)}
                \PY{n}{cv2}\PY{o}{.}\PY{n}{circle}\PY{p}{(}\PY{n}{whole\PYZus{}img}\PY{p}{,}
                    \PY{p}{(}\PY{n}{test\PYZus{}x}\PY{p}{,} \PY{n}{test\PYZus{}y}\PY{p}{)}\PY{p}{,}
                    \PY{l+m+mi}{4}\PY{p}{,} \PY{p}{(}\PY{l+m+mi}{255}\PY{p}{,} \PY{l+m+mi}{0}\PY{p}{,} \PY{l+m+mi}{0}\PY{p}{)}\PY{p}{,} \PY{n}{thickness}\PY{o}{=}\PY{l+m+mi}{10}\PY{p}{)}
                \PY{n}{cv2}\PY{o}{.}\PY{n}{circle}\PY{p}{(}\PY{n}{whole\PYZus{}img}\PY{p}{,}
                    \PY{p}{(}\PY{n}{match\PYZus{}x}\PY{p}{,} \PY{n}{match\PYZus{}y}\PY{p}{)}\PY{p}{,}
                    \PY{l+m+mi}{4}\PY{p}{,} \PY{p}{(}\PY{l+m+mi}{255}\PY{p}{,} \PY{l+m+mi}{0}\PY{p}{,} \PY{l+m+mi}{0}\PY{p}{)}\PY{p}{,} \PY{n}{thickness}\PY{o}{=}\PY{l+m+mi}{10}\PY{p}{)}
        
            \PY{n+nb}{print}\PY{p}{(}\PY{l+s+s1}{\PYZsq{}}\PY{l+s+s1}{If your solution is correct, the red lines will match to the same points in both images below:}\PY{l+s+s1}{\PYZsq{}}\PY{p}{)}
            \PY{n}{show\PYZus{}image}\PY{p}{(}\PY{n}{np}\PY{o}{.}\PY{n}{clip}\PY{p}{(}\PY{n}{whole\PYZus{}img}\PY{p}{,} \PY{l+m+mi}{0}\PY{p}{,} \PY{l+m+mi}{1}\PY{p}{)}\PY{p}{)} 
        
        \PY{n}{test\PYZus{}homography}\PY{p}{(}\PY{p}{)}
\end{Verbatim}


    \begin{Verbatim}[commandchars=\\\{\}]
Your solution differs from our solution by: 0.000000
If your solution is correct, the red lines will match to the same points in both images below:

    \end{Verbatim}

    \begin{center}
    \adjustimage{max size={0.9\linewidth}{0.9\paperheight}}{output_7_1.png}
    \end{center}
    { \hspace*{\fill} \\}
    
    \section{Problem 2: Warping}\label{problem-2-warping}

When we map a source image to its destination image using a homography,
we may encounter a problem where multiple pixels of the source image are
mapped to the same point of its destination image. What's more, some
pixels of the destination image may not be mapped to any pixels of
source image. What should we do?

Suppose we had homography \(H\), source pixel \(s\) with coordinates
\((x_s, y_s)\), and destination pixel \(d\) with coordinates
\((x_d, y_d)\). Then, \(H \cdot \tilde{s} = \tilde{d}\) (where, \(s\),
\(d\) are in homogenous space).

To deal with this problem, we warp in the opposite direction: we map the
pixels of the destination image back to source image, and then use the
color in the source image as its color. More precisely, for each
destination pixel \(d = (x_d, y_d)\), we take \(H^{-1} \cdot \tilde{d}\)
to obtain the coordinate of its associated source pixel, \(\tilde{s}\)
(from which \(s\) can be found). If \(s\) is within the bounds of the
source image, we take the intensity of \(s\) to be the intensity of
\(d\).

Repeating this process over the entire destination image ensures that
there are no gaps in the final result. This process is called "backward
warping".

    \begin{Verbatim}[commandchars=\\\{\}]
{\color{incolor}In [{\color{incolor}5}]:} \PY{k}{def} \PY{n+nf}{backward\PYZus{}warp\PYZus{}img}\PY{p}{(}\PY{n}{src\PYZus{}img}\PY{p}{,} \PY{n}{H}\PY{p}{,} \PY{n}{dst\PYZus{}img\PYZus{}size}\PY{p}{)}\PY{p}{:}
            \PY{l+s+sd}{\PYZsq{}\PYZsq{}\PYZsq{}Backward warping of the source image using a homography.}
        \PY{l+s+sd}{    }
        \PY{l+s+sd}{    Input:}
        \PY{l+s+sd}{        src\PYZus{}img: source image, shape (m, n, 3)}
        \PY{l+s+sd}{        H: homography from destination to source image, shape (3, 3)}
        \PY{l+s+sd}{        dst\PYZus{}img\PYZus{}size: height and width of destination image, shape (2,)}
        \PY{l+s+sd}{    Output:}
        \PY{l+s+sd}{        dst\PYZus{}img: destination image, shape (m, n, 3)}
        \PY{l+s+sd}{    }
        \PY{l+s+sd}{    TODO: Implement the backward\PYZus{}warp\PYZus{}img function. }
        \PY{l+s+sd}{    \PYZsq{}\PYZsq{}\PYZsq{}}
            \PY{n}{dict\PYZus{}matrix} \PY{o}{=} \PY{n}{np}\PY{o}{.}\PY{n}{ones}\PY{p}{(}\PY{p}{(}\PY{l+m+mi}{3}\PY{p}{,}\PY{n}{dst\PYZus{}img\PYZus{}size}\PY{p}{[}\PY{l+m+mi}{0}\PY{p}{]}\PY{o}{*}\PY{n}{dst\PYZus{}img\PYZus{}size}\PY{p}{[}\PY{l+m+mi}{1}\PY{p}{]}\PY{p}{)}\PY{p}{)}
            \PY{n}{dict\PYZus{}matrix}\PY{p}{[}\PY{l+m+mi}{1}\PY{p}{,}\PY{p}{:}\PY{p}{]} \PY{o}{=} \PY{n}{np}\PY{o}{.}\PY{n}{array}\PY{p}{(}\PY{p}{[}
                \PY{p}{[}\PY{n}{i}\PY{p}{]} \PY{o}{*} \PY{n}{dst\PYZus{}img\PYZus{}size}\PY{p}{[}\PY{l+m+mi}{1}\PY{p}{]} \PY{k}{for} \PY{n}{i} \PY{o+ow}{in} \PY{n+nb}{range}\PY{p}{(}\PY{n}{dst\PYZus{}img\PYZus{}size}\PY{p}{[}\PY{l+m+mi}{0}\PY{p}{]}\PY{p}{)}
            \PY{p}{]}\PY{p}{)}\PY{o}{.}\PY{n}{flatten}\PY{p}{(}\PY{p}{)}
            \PY{n}{dict\PYZus{}matrix}\PY{p}{[}\PY{l+m+mi}{0}\PY{p}{,}\PY{p}{:}\PY{p}{]} \PY{o}{=} \PY{n}{np}\PY{o}{.}\PY{n}{array}\PY{p}{(}\PY{p}{[}
                \PY{p}{[}\PY{n}{i} \PY{k}{for} \PY{n}{i} \PY{o+ow}{in} \PY{n+nb}{range}\PY{p}{(}\PY{n}{dst\PYZus{}img\PYZus{}size}\PY{p}{[}\PY{l+m+mi}{1}\PY{p}{]}\PY{p}{)}\PY{p}{]} \PY{o}{*} \PY{n}{dst\PYZus{}img\PYZus{}size}\PY{p}{[}\PY{l+m+mi}{0}\PY{p}{]}
            \PY{p}{]}\PY{p}{)}\PY{o}{.}\PY{n}{flatten}\PY{p}{(}\PY{p}{)}
            
            \PY{n}{dict\PYZus{}matrix} \PY{o}{=} \PY{n}{np}\PY{o}{.}\PY{n}{dot}\PY{p}{(}\PY{n}{H}\PY{p}{,}\PY{n}{dict\PYZus{}matrix}\PY{p}{)}
            \PY{n}{dict\PYZus{}matrix} \PY{o}{=} \PY{n}{dict\PYZus{}matrix} \PY{o}{/} \PY{n}{dict\PYZus{}matrix}\PY{p}{[}\PY{l+m+mi}{2}\PY{p}{,}\PY{p}{:}\PY{p}{]}
            \PY{n}{dict\PYZus{}matrix} \PY{o}{=} \PY{n}{dict\PYZus{}matrix}\PY{o}{.}\PY{n}{astype}\PY{p}{(}\PY{n}{np}\PY{o}{.}\PY{n}{int32}\PY{p}{)}
            
            \PY{n}{dst\PYZus{}img} \PY{o}{=} \PY{n}{np}\PY{o}{.}\PY{n}{zeros}\PY{p}{(}\PY{p}{(}\PY{n}{dst\PYZus{}img\PYZus{}size}\PY{p}{[}\PY{l+m+mi}{0}\PY{p}{]}\PY{p}{,} \PY{n}{dst\PYZus{}img\PYZus{}size}\PY{p}{[}\PY{l+m+mi}{1}\PY{p}{]}\PY{p}{,} \PY{l+m+mi}{3}\PY{p}{)}\PY{p}{)}
        
            \PY{k}{for} \PY{n}{r} \PY{o+ow}{in} \PY{n+nb}{range}\PY{p}{(}\PY{n}{dst\PYZus{}img\PYZus{}size}\PY{p}{[}\PY{l+m+mi}{0}\PY{p}{]}\PY{p}{)}\PY{p}{:}
                \PY{k}{for} \PY{n}{c} \PY{o+ow}{in} \PY{n+nb}{range}\PY{p}{(}\PY{n}{dst\PYZus{}img\PYZus{}size}\PY{p}{[}\PY{l+m+mi}{1}\PY{p}{]}\PY{p}{)}\PY{p}{:}
                    \PY{n}{x} \PY{o}{=} \PY{n}{dict\PYZus{}matrix}\PY{p}{[}\PY{l+m+mi}{0}\PY{p}{,} \PY{n}{r}\PY{o}{*}\PY{n}{dst\PYZus{}img\PYZus{}size}\PY{p}{[}\PY{l+m+mi}{1}\PY{p}{]}\PY{o}{+}\PY{n}{c}\PY{p}{]}
                    \PY{n}{y} \PY{o}{=} \PY{n}{dict\PYZus{}matrix}\PY{p}{[}\PY{l+m+mi}{1}\PY{p}{,} \PY{n}{r}\PY{o}{*}\PY{n}{dst\PYZus{}img\PYZus{}size}\PY{p}{[}\PY{l+m+mi}{1}\PY{p}{]}\PY{o}{+}\PY{n}{c}\PY{p}{]}
                    \PY{k}{if} \PY{n}{x} \PY{o}{\PYZgt{}}\PY{o}{=}\PY{l+m+mi}{0} \PY{o+ow}{and} \PY{n}{y} \PY{o}{\PYZgt{}}\PY{o}{=}\PY{l+m+mi}{0} \PY{o+ow}{and} \PY{n}{x} \PY{o}{\PYZlt{}} \PY{n}{src\PYZus{}img}\PY{o}{.}\PY{n}{shape}\PY{p}{[}\PY{l+m+mi}{1}\PY{p}{]} \PY{o+ow}{and} \PY{n}{y} \PY{o}{\PYZlt{}} \PY{n}{src\PYZus{}img}\PY{o}{.}\PY{n}{shape}\PY{p}{[}\PY{l+m+mi}{0}\PY{p}{]}\PY{p}{:}
                        \PY{n}{dst\PYZus{}img}\PY{p}{[}\PY{n}{r}\PY{p}{,}\PY{n}{c}\PY{p}{,}\PY{p}{:}\PY{p}{]} \PY{o}{=} \PY{n}{src\PYZus{}img}\PY{p}{[}\PY{n}{y}\PY{p}{,} \PY{n}{x}\PY{p}{,} \PY{p}{:}\PY{p}{]}
            \PY{k}{return} \PY{n}{dst\PYZus{}img}
        
        \PY{k}{def} \PY{n+nf}{binary\PYZus{}mask}\PY{p}{(}\PY{n}{img}\PY{p}{)}\PY{p}{:}
            \PY{l+s+sd}{\PYZsq{}\PYZsq{}\PYZsq{}Create a binary mask of the image content.}
        \PY{l+s+sd}{    }
        \PY{l+s+sd}{    Input:}
        \PY{l+s+sd}{        img: source image, shape (m, n, 3)}
        \PY{l+s+sd}{    Output:}
        \PY{l+s+sd}{        mask: image of shape (m, n) and type \PYZsq{}int\PYZsq{}. For pixel [i, j] of mask, if img[i, j] \PYZgt{} 0 }
        \PY{l+s+sd}{              in any of its channels, mask[i, j] = 1. Else, (if img[i, j] = 0), mask[i, j] = 0.}
        \PY{l+s+sd}{    \PYZsq{}\PYZsq{}\PYZsq{}}
        
            \PY{n}{mask} \PY{o}{=} \PY{p}{(}\PY{n}{img}\PY{p}{[}\PY{p}{:}\PY{p}{,} \PY{p}{:}\PY{p}{,} \PY{l+m+mi}{0}\PY{p}{]} \PY{o}{\PYZgt{}} \PY{l+m+mi}{0}\PY{p}{)} \PY{o}{|} \PY{p}{(}\PY{n}{img}\PY{p}{[}\PY{p}{:}\PY{p}{,} \PY{p}{:}\PY{p}{,} \PY{l+m+mi}{1}\PY{p}{]} \PY{o}{\PYZgt{}} \PY{l+m+mi}{0}\PY{p}{)} \PY{o}{|} \PY{p}{(}\PY{n}{img}\PY{p}{[}\PY{p}{:}\PY{p}{,} \PY{p}{:}\PY{p}{,} \PY{l+m+mi}{2}\PY{p}{]} \PY{o}{\PYZgt{}} \PY{l+m+mi}{0}\PY{p}{)}
            \PY{n}{mask} \PY{o}{=} \PY{n}{mask}\PY{o}{.}\PY{n}{astype}\PY{p}{(}\PY{l+s+s2}{\PYZdq{}}\PY{l+s+s2}{int}\PY{l+s+s2}{\PYZdq{}}\PY{p}{)}
            
            \PY{k}{return} \PY{n}{mask}
\end{Verbatim}


    Use the function below to help debug your implementation. If it is
correct, it should warp Van Gogh's self-portrait onto the building side.

    \begin{Verbatim}[commandchars=\\\{\}]
{\color{incolor}In [{\color{incolor}6}]:} \PY{k}{def} \PY{n+nf}{test\PYZus{}warp}\PY{p}{(}\PY{p}{)}\PY{p}{:}
            \PY{n}{src\PYZus{}img} \PY{o}{=} \PY{n}{load\PYZus{}image}\PY{p}{(}\PY{l+s+s1}{\PYZsq{}}\PY{l+s+s1}{portrait\PYZus{}small.png}\PY{l+s+s1}{\PYZsq{}}\PY{p}{)}
            \PY{n}{canvas} \PY{o}{=} \PY{n}{load\PYZus{}image}\PY{p}{(}\PY{l+s+s1}{\PYZsq{}}\PY{l+s+s1}{Osaka.png}\PY{l+s+s1}{\PYZsq{}}\PY{p}{)}
        
            \PY{n}{src\PYZus{}pts} \PY{o}{=} \PY{n}{np}\PY{o}{.}\PY{n}{matrix}\PY{p}{(}\PY{l+s+s1}{\PYZsq{}}\PY{l+s+s1}{1, 1; 1, 400; 326, 1; 326, 400}\PY{l+s+s1}{\PYZsq{}}\PY{p}{)}
            \PY{n}{canvas\PYZus{}pts} \PY{o}{=} \PY{n}{np}\PY{o}{.}\PY{n}{matrix}\PY{p}{(}\PY{l+s+s1}{\PYZsq{}}\PY{l+s+s1}{100, 18; 84, 437; 276, 71; 286, 424}\PY{l+s+s1}{\PYZsq{}}\PY{p}{)}
            \PY{n}{H} \PY{o}{=} \PY{n}{compute\PYZus{}homography}\PY{p}{(}\PY{n}{src\PYZus{}pts}\PY{p}{,} \PY{n}{canvas\PYZus{}pts}\PY{p}{)}
        
            \PY{n}{dst\PYZus{}img} \PY{o}{=} \PY{n}{backward\PYZus{}warp\PYZus{}img}\PY{p}{(}\PY{n}{src\PYZus{}img}\PY{p}{,} \PY{n}{np}\PY{o}{.}\PY{n}{linalg}\PY{o}{.}\PY{n}{inv}\PY{p}{(}\PY{n}{H}\PY{p}{)}\PY{p}{,} \PY{p}{[}\PY{n}{canvas}\PY{o}{.}\PY{n}{shape}\PY{p}{[}\PY{l+m+mi}{0}\PY{p}{]}\PY{p}{,} \PY{n}{canvas}\PY{o}{.}\PY{n}{shape}\PY{p}{[}\PY{l+m+mi}{1}\PY{p}{]}\PY{p}{]}\PY{p}{)}
            \PY{n}{dst\PYZus{}mask} \PY{o}{=} \PY{l+m+mi}{1} \PY{o}{\PYZhy{}} \PY{n}{binary\PYZus{}mask}\PY{p}{(}\PY{n}{dst\PYZus{}img}\PY{p}{)}
            \PY{n}{dst\PYZus{}mask} \PY{o}{=} \PY{n}{np}\PY{o}{.}\PY{n}{stack}\PY{p}{(}\PY{p}{(}\PY{n}{dst\PYZus{}mask}\PY{p}{,}\PY{p}{)} \PY{o}{*} \PY{l+m+mi}{3}\PY{p}{,} \PY{o}{\PYZhy{}}\PY{l+m+mi}{1}\PY{p}{)}
            \PY{n}{out\PYZus{}img} \PY{o}{=} \PY{n}{np}\PY{o}{.}\PY{n}{multiply}\PY{p}{(}\PY{n}{canvas}\PY{p}{,} \PY{n}{dst\PYZus{}mask}\PY{p}{)} \PY{o}{+} \PY{n}{dst\PYZus{}img}
        
            \PY{n}{warp\PYZus{}img} \PY{o}{=} \PY{n}{np}\PY{o}{.}\PY{n}{concatenate}\PY{p}{(}\PY{p}{(}\PY{n}{canvas}\PY{p}{,} \PY{n}{out\PYZus{}img}\PY{p}{)}\PY{p}{,} \PY{n}{axis}\PY{o}{=}\PY{l+m+mi}{1}\PY{p}{)}
        
            \PY{n}{show\PYZus{}image}\PY{p}{(}\PY{n}{np}\PY{o}{.}\PY{n}{clip}\PY{p}{(}\PY{n}{warp\PYZus{}img}\PY{p}{,} \PY{l+m+mi}{0}\PY{p}{,} \PY{l+m+mi}{1}\PY{p}{)}\PY{p}{)}
            
        \PY{n}{test\PYZus{}warp}\PY{p}{(}\PY{p}{)}
\end{Verbatim}


    \begin{center}
    \adjustimage{max size={0.9\linewidth}{0.9\paperheight}}{output_11_0.png}
    \end{center}
    { \hspace*{\fill} \\}
    
    \section{Problem 3: SIFT and RANSAC}\label{problem-3-sift-and-ransac}

\subsection{SIFT Keypoints}\label{sift-keypoints}

So far, we have manually defined corresponding keypoints for both
estimating homographies and warping. We want to automate this now.
However, if we just take two photos, how do we know which points
correspond? We could estimate SIFT keypoints, and take the nearest
neighbor between them. The code below computes SIFT keypoints, and
visualizes the matches.

    \begin{Verbatim}[commandchars=\\\{\}]
{\color{incolor}In [{\color{incolor}7}]:} \PY{k}{def} \PY{n+nf}{genSIFTMatchPairs}\PY{p}{(}\PY{n}{img1}\PY{p}{,} \PY{n}{img2}\PY{p}{)}\PY{p}{:}
            \PY{n}{sift} \PY{o}{=} \PY{n}{cv2}\PY{o}{.}\PY{n}{xfeatures2d}\PY{o}{.}\PY{n}{SIFT\PYZus{}create}\PY{p}{(}\PY{p}{)}
            \PY{n}{kp1}\PY{p}{,} \PY{n}{des1} \PY{o}{=} \PY{n}{sift}\PY{o}{.}\PY{n}{detectAndCompute}\PY{p}{(}\PY{n}{img1}\PY{p}{,} \PY{k+kc}{None}\PY{p}{)}
            \PY{n}{kp2}\PY{p}{,} \PY{n}{des2} \PY{o}{=} \PY{n}{sift}\PY{o}{.}\PY{n}{detectAndCompute}\PY{p}{(}\PY{n}{img2}\PY{p}{,} \PY{k+kc}{None}\PY{p}{)}
        
            \PY{n}{bf} \PY{o}{=} \PY{n}{cv2}\PY{o}{.}\PY{n}{BFMatcher}\PY{p}{(}\PY{n}{cv2}\PY{o}{.}\PY{n}{NORM\PYZus{}L2}\PY{p}{,} \PY{n}{crossCheck}\PY{o}{=}\PY{k+kc}{False}\PY{p}{)}
            \PY{n}{matches} \PY{o}{=} \PY{n}{bf}\PY{o}{.}\PY{n}{match}\PY{p}{(}\PY{n}{des1}\PY{p}{,} \PY{n}{des2}\PY{p}{)}
            \PY{n}{matches} \PY{o}{=} \PY{n+nb}{sorted}\PY{p}{(}\PY{n}{matches}\PY{p}{,} \PY{n}{key} \PY{o}{=} \PY{k}{lambda} \PY{n}{x}\PY{p}{:}\PY{n}{x}\PY{o}{.}\PY{n}{distance}\PY{p}{)}
            
            \PY{n}{pts1} \PY{o}{=} \PY{n}{np}\PY{o}{.}\PY{n}{zeros}\PY{p}{(}\PY{p}{(}\PY{l+m+mi}{250}\PY{p}{,}\PY{l+m+mi}{2}\PY{p}{)}\PY{p}{)}
            \PY{n}{pts2} \PY{o}{=} \PY{n}{np}\PY{o}{.}\PY{n}{zeros}\PY{p}{(}\PY{p}{(}\PY{l+m+mi}{250}\PY{p}{,}\PY{l+m+mi}{2}\PY{p}{)}\PY{p}{)}
            \PY{k}{for} \PY{n}{i} \PY{o+ow}{in} \PY{n+nb}{range}\PY{p}{(}\PY{l+m+mi}{250}\PY{p}{)}\PY{p}{:}
                \PY{n}{pts1}\PY{p}{[}\PY{n}{i}\PY{p}{,}\PY{p}{:}\PY{p}{]} \PY{o}{=} \PY{n}{kp1}\PY{p}{[}\PY{n}{matches}\PY{p}{[}\PY{n}{i}\PY{p}{]}\PY{o}{.}\PY{n}{queryIdx}\PY{p}{]}\PY{o}{.}\PY{n}{pt}
                \PY{n}{pts2}\PY{p}{[}\PY{n}{i}\PY{p}{,}\PY{p}{:}\PY{p}{]} \PY{o}{=} \PY{n}{kp2}\PY{p}{[}\PY{n}{matches}\PY{p}{[}\PY{n}{i}\PY{p}{]}\PY{o}{.}\PY{n}{trainIdx}\PY{p}{]}\PY{o}{.}\PY{n}{pt}
            
            \PY{k}{return} \PY{n}{pts1}\PY{p}{,} \PY{n}{pts2}\PY{p}{,} \PY{n}{matches}\PY{p}{[}\PY{p}{:}\PY{l+m+mi}{250}\PY{p}{]}\PY{p}{,} \PY{n}{kp1}\PY{p}{,} \PY{n}{kp2}
        
        \PY{k}{def} \PY{n+nf}{test\PYZus{}matches}\PY{p}{(}\PY{p}{)}\PY{p}{:}
            \PY{n}{img1} \PY{o}{=} \PY{n}{cv2}\PY{o}{.}\PY{n}{imread}\PY{p}{(}\PY{l+s+s1}{\PYZsq{}}\PY{l+s+s1}{mountain\PYZus{}left.png}\PY{l+s+s1}{\PYZsq{}}\PY{p}{)}
            \PY{n}{img2} \PY{o}{=} \PY{n}{cv2}\PY{o}{.}\PY{n}{imread}\PY{p}{(}\PY{l+s+s1}{\PYZsq{}}\PY{l+s+s1}{mountain\PYZus{}center.png}\PY{l+s+s1}{\PYZsq{}}\PY{p}{)}
        
            \PY{n}{pts1}\PY{p}{,} \PY{n}{pts2}\PY{p}{,} \PY{n}{matches}\PY{p}{,} \PY{n}{kp1}\PY{p}{,} \PY{n}{kp2} \PY{o}{=} \PY{n}{genSIFTMatchPairs}\PY{p}{(}\PY{n}{img1}\PY{p}{,} \PY{n}{img2}\PY{p}{)}
        
            \PY{n}{matching\PYZus{}result} \PY{o}{=} \PY{n}{cv2}\PY{o}{.}\PY{n}{drawMatches}\PY{p}{(}\PY{n}{img1}\PY{p}{,} \PY{n}{kp1}\PY{p}{,} \PY{n}{img2}\PY{p}{,} \PY{n}{kp2}\PY{p}{,} \PY{n}{matches}\PY{p}{,} \PY{k+kc}{None}\PY{p}{,} \PY{n}{flags}\PY{o}{=}\PY{l+m+mi}{2}\PY{p}{,} \PY{n}{matchColor}\PY{o}{=}\PY{p}{(}\PY{l+m+mi}{0}\PY{p}{,}\PY{l+m+mi}{0}\PY{p}{,}\PY{l+m+mi}{255}\PY{p}{)}\PY{p}{)}
            \PY{n}{plt}\PY{o}{.}\PY{n}{imshow}\PY{p}{(}\PY{n}{cv2}\PY{o}{.}\PY{n}{cvtColor}\PY{p}{(}\PY{n}{matching\PYZus{}result}\PY{p}{,} \PY{n}{cv2}\PY{o}{.}\PY{n}{COLOR\PYZus{}BGR2RGB}\PY{p}{)}\PY{p}{)}
        
        \PY{n}{test\PYZus{}matches}\PY{p}{(}\PY{p}{)}
\end{Verbatim}


    \begin{center}
    \adjustimage{max size={0.9\linewidth}{0.9\paperheight}}{output_13_0.png}
    \end{center}
    { \hspace*{\fill} \\}
    
    Notice that the matches are not all correct. There is a substantial
amount of noise or incorrect matches. If we include these wrong matches
in our homography estimation, what will happen? Think about this, and
convince yourself why it will not work well.

    \subsection{RANSAC}\label{ransac}

Instead, we will use RANSAC, which is an optimization algorithm that
finds correspondences while also discarding the outliers. Implement the
RANSAC function below.

    \begin{Verbatim}[commandchars=\\\{\}]
{\color{incolor}In [{\color{incolor}8}]:} \PY{k}{def} \PY{n+nf}{RANSAC}\PY{p}{(}\PY{n}{Xs}\PY{p}{,} \PY{n}{Xd}\PY{p}{,} \PY{n}{max\PYZus{}iter}\PY{p}{,} \PY{n}{eps}\PY{p}{)}\PY{p}{:}
            \PY{l+s+sd}{\PYZsq{}\PYZsq{}\PYZsq{}Finds correspondences between two sets of points using the RANSAC algorithm.}
        \PY{l+s+sd}{    }
        \PY{l+s+sd}{    Input:}
        \PY{l+s+sd}{        Xs: the first set of points (source), shape [n, 2]}
        \PY{l+s+sd}{        Xd: the second set of points (destination) matched to the first set, shape [n, 2]}
        \PY{l+s+sd}{        max\PYZus{}iter: max iteration number of RANSAC}
        \PY{l+s+sd}{        eps: tolerance of RANSAC}
        \PY{l+s+sd}{    Output:}
        \PY{l+s+sd}{        inliers\PYZus{}id: the indices of matched pairs when using the homography given by RANSAC}
        \PY{l+s+sd}{        H: the homography, shape [3, 3]}
        \PY{l+s+sd}{    }
        \PY{l+s+sd}{    TODO: Implement the RANSAC function. }
        \PY{l+s+sd}{    \PYZsq{}\PYZsq{}\PYZsq{}}
            
            \PY{n}{inliers\PYZus{}id} \PY{o}{=} \PY{p}{[}\PY{p}{]}
            \PY{n}{best\PYZus{}count} \PY{o}{=} \PY{l+m+mi}{0}
        \PY{c+c1}{\PYZsh{}     best\PYZus{}H = None}
            \PY{k}{for} \PY{n}{\PYZus{}} \PY{o+ow}{in} \PY{n+nb}{range}\PY{p}{(}\PY{n}{max\PYZus{}iter}\PY{p}{)}\PY{p}{:}
                \PY{n}{samples} \PY{o}{=} \PY{n}{np}\PY{o}{.}\PY{n}{random}\PY{o}{.}\PY{n}{choice}\PY{p}{(}\PY{n}{Xs}\PY{o}{.}\PY{n}{shape}\PY{p}{[}\PY{l+m+mi}{0}\PY{p}{]}\PY{p}{,} \PY{l+m+mi}{4}\PY{p}{)}
                \PY{n}{H} \PY{o}{=} \PY{n}{compute\PYZus{}homography}\PY{p}{(}\PY{n}{Xs}\PY{p}{[}\PY{n}{samples}\PY{p}{,}\PY{p}{:}\PY{p}{]}\PY{p}{,} \PY{n}{Xd}\PY{p}{[}\PY{n}{samples}\PY{p}{,}\PY{p}{:}\PY{p}{]}\PY{p}{)}
                \PY{n}{dst} \PY{o}{=} \PY{n}{apply\PYZus{}homography}\PY{p}{(}\PY{n}{Xs}\PY{p}{,} \PY{n}{H}\PY{p}{)}
                \PY{n}{distance} \PY{o}{=} \PY{n}{np}\PY{o}{.}\PY{n}{linalg}\PY{o}{.}\PY{n}{norm}\PY{p}{(}\PY{n}{dst}\PY{o}{\PYZhy{}}\PY{n}{Xd}\PY{p}{,}\PY{n}{axis}\PY{o}{=}\PY{l+m+mi}{1}\PY{p}{)}
                
                \PY{n}{correct\PYZus{}num} \PY{o}{=} \PY{l+m+mi}{0}
                \PY{n}{samples} \PY{o}{=} \PY{p}{[}\PY{p}{]}
                \PY{k}{for} \PY{n}{index}\PY{p}{,} \PY{n}{diff} \PY{o+ow}{in} \PY{n+nb}{enumerate}\PY{p}{(}\PY{n}{distance}\PY{p}{)}\PY{p}{:}
                    \PY{k}{if} \PY{n}{diff} \PY{o}{\PYZlt{}} \PY{n}{eps}\PY{p}{:}
                        \PY{n}{correct\PYZus{}num} \PY{o}{+}\PY{o}{=} \PY{l+m+mi}{1}
                        \PY{n}{samples}\PY{o}{.}\PY{n}{append}\PY{p}{(}\PY{n}{index}\PY{p}{)}
                        
                \PY{k}{if} \PY{n}{correct\PYZus{}num} \PY{o}{\PYZgt{}} \PY{n}{best\PYZus{}count}\PY{p}{:}
                    \PY{n}{best\PYZus{}count} \PY{o}{=} \PY{n}{correct\PYZus{}num}
                    \PY{n}{inliers\PYZus{}id} \PY{o}{=} \PY{n}{samples}
        \PY{c+c1}{\PYZsh{}             best\PYZus{}H = H}
                
                \PY{n}{H} \PY{o}{=} \PY{n}{compute\PYZus{}homography}\PY{p}{(}\PY{n}{Xs}\PY{p}{[}\PY{n}{samples}\PY{p}{,}\PY{p}{:}\PY{p}{]}\PY{p}{,} \PY{n}{Xd}\PY{p}{[}\PY{n}{samples}\PY{p}{,}\PY{p}{:}\PY{p}{]}\PY{p}{)}
                
            \PY{k}{return} \PY{n}{inliers\PYZus{}id}\PY{p}{,} \PY{n}{H}
\end{Verbatim}


    Now, let's visualize the matches between keypoints after using your
RANSAC implementation. If you implemented RANSAC correctly, the outlier
matches should be automatically discarded.

    \begin{Verbatim}[commandchars=\\\{\}]
{\color{incolor}In [{\color{incolor}9}]:} \PY{k}{def} \PY{n+nf}{test\PYZus{}ransac}\PY{p}{(}\PY{p}{)}\PY{p}{:}
            \PY{n}{img1} \PY{o}{=} \PY{n}{cv2}\PY{o}{.}\PY{n}{imread}\PY{p}{(}\PY{l+s+s1}{\PYZsq{}}\PY{l+s+s1}{mountain\PYZus{}left.png}\PY{l+s+s1}{\PYZsq{}}\PY{p}{)}
            \PY{n}{img2} \PY{o}{=} \PY{n}{cv2}\PY{o}{.}\PY{n}{imread}\PY{p}{(}\PY{l+s+s1}{\PYZsq{}}\PY{l+s+s1}{mountain\PYZus{}center.png}\PY{l+s+s1}{\PYZsq{}}\PY{p}{)}
        
            \PY{n}{pts1}\PY{p}{,} \PY{n}{pts2}\PY{p}{,} \PY{n}{matches}\PY{p}{,} \PY{n}{kp1}\PY{p}{,} \PY{n}{kp2} \PY{o}{=} \PY{n}{genSIFTMatchPairs}\PY{p}{(}\PY{n}{img1}\PY{p}{,} \PY{n}{img2}\PY{p}{)}
            
            \PY{n}{inliers\PYZus{}idx}\PY{p}{,} \PY{n}{H} \PY{o}{=} \PY{n}{RANSAC}\PY{p}{(}\PY{n}{pts1}\PY{p}{,} \PY{n}{pts2}\PY{p}{,} \PY{l+m+mi}{500}\PY{p}{,} \PY{l+m+mi}{20}\PY{p}{)}
        
            \PY{n}{new\PYZus{}matches} \PY{o}{=} \PY{p}{[}\PY{p}{]}
            \PY{k}{for} \PY{n}{i} \PY{o+ow}{in} \PY{n+nb}{range}\PY{p}{(}\PY{n+nb}{len}\PY{p}{(}\PY{n}{inliers\PYZus{}idx}\PY{p}{)}\PY{p}{)}\PY{p}{:}
                \PY{n}{new\PYZus{}matches}\PY{o}{.}\PY{n}{append}\PY{p}{(}\PY{n}{matches}\PY{p}{[}\PY{n}{inliers\PYZus{}idx}\PY{p}{[}\PY{n}{i}\PY{p}{]}\PY{p}{]}\PY{p}{)}
        
            \PY{n}{matching\PYZus{}result} \PY{o}{=} \PY{n}{cv2}\PY{o}{.}\PY{n}{drawMatches}\PY{p}{(}\PY{n}{img1}\PY{p}{,} \PY{n}{kp1}\PY{p}{,} \PY{n}{img2}\PY{p}{,} \PY{n}{kp2}\PY{p}{,} \PY{n}{new\PYZus{}matches}\PY{p}{,} \PY{k+kc}{None}\PY{p}{,} \PY{n}{flags}\PY{o}{=}\PY{l+m+mi}{2}\PY{p}{,} \PY{n}{matchColor}\PY{o}{=}\PY{p}{(}\PY{l+m+mi}{0}\PY{p}{,}\PY{l+m+mi}{0}\PY{p}{,}\PY{l+m+mi}{255}\PY{p}{)}\PY{p}{)}
            \PY{n}{plt}\PY{o}{.}\PY{n}{imshow}\PY{p}{(}\PY{n}{cv2}\PY{o}{.}\PY{n}{cvtColor}\PY{p}{(}\PY{n}{matching\PYZus{}result}\PY{p}{,} \PY{n}{cv2}\PY{o}{.}\PY{n}{COLOR\PYZus{}BGR2RGB}\PY{p}{)}\PY{p}{)}
            
        \PY{n}{test\PYZus{}ransac}\PY{p}{(}\PY{p}{)}
\end{Verbatim}


    \begin{center}
    \adjustimage{max size={0.9\linewidth}{0.9\paperheight}}{output_18_0.png}
    \end{center}
    { \hspace*{\fill} \\}
    
    \section{Problem 4: Image Blending}\label{problem-4-image-blending}

We have now implemented code to estimate correspondences between photos,
estimate the homography, and warp one image into the other image. Before
we can build our panorama making application, the next piece we need is
code to seamlessly blend two images together.

    \begin{Verbatim}[commandchars=\\\{\}]
{\color{incolor}In [{\color{incolor}10}]:} \PY{k+kn}{from} \PY{n+nn}{scipy}\PY{n+nn}{.}\PY{n+nn}{ndimage}\PY{n+nn}{.}\PY{n+nn}{morphology} \PY{k}{import} \PY{n}{distance\PYZus{}transform\PYZus{}edt} \PY{k}{as} \PY{n}{euc\PYZus{}dist}
         
         \PY{k}{def} \PY{n+nf}{blend\PYZus{}image\PYZus{}pair}\PY{p}{(}\PY{n}{src\PYZus{}img}\PY{p}{,} \PY{n}{src\PYZus{}mask}\PY{p}{,} \PY{n}{dst\PYZus{}img}\PY{p}{,} \PY{n}{dst\PYZus{}mask}\PY{p}{,} \PY{n}{mode}\PY{p}{)}\PY{p}{:}
             \PY{l+s+sd}{\PYZsq{}\PYZsq{}\PYZsq{}Given two images and their binary masks, blend the two images.}
         \PY{l+s+sd}{    }
         \PY{l+s+sd}{    Input:}
         \PY{l+s+sd}{        src\PYZus{}img: First image to be blended, shape (m, n, 3)}
         \PY{l+s+sd}{        src\PYZus{}mask: src\PYZus{}img\PYZsq{}s binary mask, shape (m, n)}
         \PY{l+s+sd}{        dst\PYZus{}img: Second image to be blended, shape (m, n, 3)}
         \PY{l+s+sd}{        dst\PYZus{}mask: dst\PYZus{}img\PYZsq{}s binary mask, shape (m, n)}
         \PY{l+s+sd}{        mode: Blending mode, either \PYZdq{}overlay\PYZdq{} or \PYZdq{}blend\PYZdq{}}
         \PY{l+s+sd}{    Output:}
         \PY{l+s+sd}{        Blended image of shape (m, n, 3)}
         \PY{l+s+sd}{    }
         \PY{l+s+sd}{    TODO: Implement the blend\PYZus{}image\PYZus{}pair function.}
         \PY{l+s+sd}{    \PYZsq{}\PYZsq{}\PYZsq{}}
             \PY{k}{if} \PY{n}{mode} \PY{o}{==} \PY{l+s+s1}{\PYZsq{}}\PY{l+s+s1}{blend}\PY{l+s+s1}{\PYZsq{}}\PY{p}{:}
                 \PY{n}{blend\PYZus{}img} \PY{o}{=} \PY{n}{np}\PY{o}{.}\PY{n}{zeros\PYZus{}like}\PY{p}{(}\PY{n}{src\PYZus{}img}\PY{p}{)}
                 \PY{n}{w1}  \PY{o}{=} \PY{n}{euc\PYZus{}dist}\PY{p}{(}\PY{n}{src\PYZus{}mask}\PY{p}{)}
                 \PY{n}{w2}  \PY{o}{=} \PY{n}{euc\PYZus{}dist}\PY{p}{(}\PY{n}{dst\PYZus{}mask}\PY{p}{)}
         
                 \PY{n}{w1} \PY{o}{=} \PY{n}{np}\PY{o}{.}\PY{n}{expand\PYZus{}dims}\PY{p}{(}\PY{n}{w1}\PY{p}{,}\PY{n}{axis}\PY{o}{=}\PY{o}{\PYZhy{}}\PY{l+m+mi}{1}\PY{p}{)}
                 \PY{n}{w2} \PY{o}{=} \PY{n}{np}\PY{o}{.}\PY{n}{expand\PYZus{}dims}\PY{p}{(}\PY{n}{w2}\PY{p}{,}\PY{n}{axis}\PY{o}{=}\PY{o}{\PYZhy{}}\PY{l+m+mi}{1}\PY{p}{)}
         
                 \PY{n}{weight\PYZus{}sum} \PY{o}{=} \PY{n}{w1} \PY{o}{+} \PY{n}{w2}
                 \PY{n}{weight\PYZus{}img} \PY{o}{=} \PY{n}{w1} \PY{o}{*} \PY{n}{src\PYZus{}img} \PY{o}{+} \PY{n}{w2} \PY{o}{*} \PY{n}{dst\PYZus{}img}
                 \PY{n}{flag} \PY{o}{=} \PY{p}{(}\PY{n}{weight\PYZus{}sum} \PY{o}{==} \PY{l+m+mi}{0}\PY{p}{)}\PY{o}{.}\PY{n}{astype}\PY{p}{(}\PY{n}{np}\PY{o}{.}\PY{n}{float}\PY{p}{)}
                 \PY{n}{weight\PYZus{}sum} \PY{o}{+}\PY{o}{=} \PY{n}{flag}
         
                 \PY{n}{blend\PYZus{}img} \PY{o}{=} \PY{n}{weight\PYZus{}img} \PY{o}{/} \PY{n}{weight\PYZus{}sum}
                 
             \PY{k}{else}\PY{p}{:}
                 \PY{n}{dst\PYZus{}mask} \PY{o}{=} \PY{n}{np}\PY{o}{.}\PY{n}{tile}\PY{p}{(}\PY{n}{np}\PY{o}{.}\PY{n}{expand\PYZus{}dims}\PY{p}{(}\PY{n}{dst\PYZus{}mask}\PY{p}{,}\PY{n}{axis}\PY{o}{=}\PY{o}{\PYZhy{}}\PY{l+m+mi}{1}\PY{p}{)}\PY{p}{,}\PY{p}{(}\PY{l+m+mi}{1}\PY{p}{,}\PY{l+m+mi}{1}\PY{p}{,}\PY{l+m+mi}{3}\PY{p}{)}\PY{p}{)}
                 \PY{n}{blend\PYZus{}img} \PY{o}{=} \PY{n}{src\PYZus{}img} \PY{o}{\PYZhy{}} \PY{n}{src\PYZus{}img}\PY{o}{*}\PY{n}{dst\PYZus{}mask} \PY{o}{+} \PY{n}{dst\PYZus{}img}
                 
             \PY{k}{return} \PY{n}{blend\PYZus{}img}\PY{o}{.}\PY{n}{astype}\PY{p}{(}\PY{n}{src\PYZus{}img}\PY{o}{.}\PY{n}{dtype}\PY{p}{)}
\end{Verbatim}


    To test your implementation, you can use the function below. It supports
two modes. Setting mode="blend" should seamlessly blend the two images.
Setting mode="overlay" will just combine them without any blending.

    \begin{Verbatim}[commandchars=\\\{\}]
{\color{incolor}In [{\color{incolor}11}]:} \PY{k}{def} \PY{n+nf}{test\PYZus{}blend}\PY{p}{(}\PY{n}{mode}\PY{p}{)}\PY{p}{:}
             \PY{n}{fish\PYZus{}img} \PY{o}{=} \PY{n}{load\PYZus{}image}\PY{p}{(}\PY{l+s+s2}{\PYZdq{}}\PY{l+s+s2}{escher\PYZus{}fish.png}\PY{l+s+s2}{\PYZdq{}}\PY{p}{)}\PY{p}{[}\PY{p}{:}\PY{p}{,} \PY{p}{:}\PY{p}{,} \PY{p}{:}\PY{l+m+mi}{3}\PY{p}{]}
             \PY{n}{horse\PYZus{}img} \PY{o}{=} \PY{n}{load\PYZus{}image}\PY{p}{(}\PY{l+s+s2}{\PYZdq{}}\PY{l+s+s2}{escher\PYZus{}horsemen.png}\PY{l+s+s2}{\PYZdq{}}\PY{p}{)}\PY{p}{[}\PY{p}{:}\PY{p}{,} \PY{p}{:}\PY{p}{,} \PY{p}{:}\PY{l+m+mi}{3}\PY{p}{]}
         
             \PY{n}{blend\PYZus{}img} \PY{o}{=} \PY{n}{blend\PYZus{}image\PYZus{}pair}\PY{p}{(}\PY{n}{fish\PYZus{}img}\PY{p}{,} \PY{n}{binary\PYZus{}mask}\PY{p}{(}\PY{n}{fish\PYZus{}img}\PY{p}{)}\PY{p}{,} \PY{n}{horse\PYZus{}img}\PY{p}{,} \PY{n}{binary\PYZus{}mask}\PY{p}{(}\PY{n}{horse\PYZus{}img}\PY{p}{)}\PY{p}{,} \PY{n}{mode}\PY{p}{)}
         
             \PY{n}{f}\PY{p}{,} \PY{n}{axarr} \PY{o}{=} \PY{n}{plt}\PY{o}{.}\PY{n}{subplots}\PY{p}{(}\PY{l+m+mi}{1}\PY{p}{,}\PY{l+m+mi}{3}\PY{p}{)}
             \PY{n}{axarr}\PY{p}{[}\PY{l+m+mi}{0}\PY{p}{]}\PY{o}{.}\PY{n}{imshow}\PY{p}{(}\PY{n}{fish\PYZus{}img}\PY{p}{,} \PY{n}{cmap}\PY{o}{=}\PY{l+s+s1}{\PYZsq{}}\PY{l+s+s1}{gray}\PY{l+s+s1}{\PYZsq{}}\PY{p}{)}
             \PY{n}{axarr}\PY{p}{[}\PY{l+m+mi}{1}\PY{p}{]}\PY{o}{.}\PY{n}{imshow}\PY{p}{(}\PY{n}{horse\PYZus{}img}\PY{p}{,} \PY{n}{cmap}\PY{o}{=}\PY{l+s+s1}{\PYZsq{}}\PY{l+s+s1}{gray}\PY{l+s+s1}{\PYZsq{}}\PY{p}{)}
             \PY{n}{axarr}\PY{p}{[}\PY{l+m+mi}{2}\PY{p}{]}\PY{o}{.}\PY{n}{imshow}\PY{p}{(}\PY{n}{blend\PYZus{}img}\PY{p}{,}\PY{n}{cmap}\PY{o}{=}\PY{l+s+s1}{\PYZsq{}}\PY{l+s+s1}{gray}\PY{l+s+s1}{\PYZsq{}}\PY{p}{)}
             
         \PY{n}{test\PYZus{}blend}\PY{p}{(}\PY{l+s+s2}{\PYZdq{}}\PY{l+s+s2}{blend}\PY{l+s+s2}{\PYZdq{}}\PY{p}{)}
         \PY{n}{test\PYZus{}blend}\PY{p}{(}\PY{l+s+s2}{\PYZdq{}}\PY{l+s+s2}{overlay}\PY{l+s+s2}{\PYZdq{}}\PY{p}{)}
\end{Verbatim}


    \begin{center}
    \adjustimage{max size={0.9\linewidth}{0.9\paperheight}}{output_22_0.png}
    \end{center}
    { \hspace*{\fill} \\}
    
    \begin{center}
    \adjustimage{max size={0.9\linewidth}{0.9\paperheight}}{output_22_1.png}
    \end{center}
    { \hspace*{\fill} \\}
    
    \section{Problem 5: Creating
Panoramas}\label{problem-5-creating-panoramas}

We are now ready to make a panorama from the three images at the
beginning. The function below receives a Python list of images, which
you should stitch together to form one large image. You will need to
call most of the functions defined above in order to successfully do
this.

To receive full credit, make sure you have stitched the three images
together with very little seam between them.

    \begin{Verbatim}[commandchars=\\\{\}]
{\color{incolor}In [{\color{incolor}12}]:} \PY{k}{def} \PY{n+nf}{stitch\PYZus{}img}\PY{p}{(}\PY{n}{imgs}\PY{p}{)}\PY{p}{:}
             \PY{l+s+sd}{\PYZsq{}\PYZsq{}\PYZsq{}Stitch a list of images together.}
         \PY{l+s+sd}{    }
         \PY{l+s+sd}{    Input: }
         \PY{l+s+sd}{        imgs: a list of images.}
         \PY{l+s+sd}{    Output:}
         \PY{l+s+sd}{        stitched\PYZus{}img: a single stiched image.}
         \PY{l+s+sd}{        }
         \PY{l+s+sd}{    TODO: implement the stitch\PYZus{}img function. }
         \PY{l+s+sd}{    \PYZsq{}\PYZsq{}\PYZsq{}}
             \PY{n}{num\PYZus{}imgs} \PY{o}{=} \PY{n+nb}{len}\PY{p}{(}\PY{n}{imgs}\PY{p}{)}
             \PY{k}{if} \PY{n}{num\PYZus{}imgs} \PY{o}{\PYZlt{}} \PY{l+m+mi}{2}\PY{p}{:}
                 \PY{k}{raise} \PY{n+ne}{ValueError}\PY{p}{(}\PY{l+s+s1}{\PYZsq{}}\PY{l+s+s1}{insufficient images!}\PY{l+s+s1}{\PYZsq{}}\PY{p}{)}
                 
             \PY{n}{base\PYZus{}img} \PY{o}{=} \PY{n}{imgs}\PY{p}{[}\PY{l+m+mi}{0}\PY{p}{]}
             \PY{n}{canvas} \PY{o}{=} \PY{n}{np}\PY{o}{.}\PY{n}{zeros\PYZus{}like}\PY{p}{(}\PY{n}{base\PYZus{}img}\PY{p}{)}
             \PY{n}{canvas} \PY{o}{=} \PY{n}{np}\PY{o}{.}\PY{n}{tile}\PY{p}{(}\PY{n}{canvas}\PY{p}{,}\PY{p}{(}\PY{l+m+mi}{1}\PY{p}{,}\PY{n}{num\PYZus{}imgs} \PY{o}{\PYZhy{}} \PY{l+m+mi}{1}\PY{p}{,}\PY{l+m+mi}{1}\PY{p}{)}\PY{p}{)}
             \PY{n}{panorama} \PY{o}{=} \PY{n}{np}\PY{o}{.}\PY{n}{hstack}\PY{p}{(}\PY{p}{[}\PY{n}{canvas}\PY{p}{,}\PY{n}{base\PYZus{}img}\PY{p}{,}\PY{n}{canvas}\PY{p}{]}\PY{p}{)}
         
             \PY{n}{left\PYZus{}img} \PY{o}{=} \PY{n}{panorama}
             \PY{n}{right\PYZus{}img} \PY{o}{=} \PY{n}{panorama}
             
             \PY{n}{unused\PYZus{}imgs} \PY{o}{=} \PY{n}{imgs}\PY{p}{[}\PY{l+m+mi}{1}\PY{p}{:}\PY{p}{]}
             
             \PY{n}{trial} \PY{o}{=} \PY{l+m+mi}{0}
             
             \PY{k}{while} \PY{n+nb}{len}\PY{p}{(}\PY{n}{unused\PYZus{}imgs}\PY{p}{)} \PY{o}{!=} \PY{l+m+mi}{0} \PY{o+ow}{and} \PY{n}{trial} \PY{o}{\PYZlt{}} \PY{l+m+mi}{2} \PY{o}{*} \PY{n}{num\PYZus{}imgs}\PY{p}{:}
                 \PY{n}{trial} \PY{o}{+}\PY{o}{=} \PY{l+m+mi}{1}
                 \PY{n+nb}{print}\PY{p}{(}\PY{l+s+s1}{\PYZsq{}}\PY{l+s+s1}{Stitch }\PY{l+s+si}{\PYZob{}\PYZcb{}}\PY{l+s+s1}{ images}\PY{l+s+s1}{\PYZsq{}}\PY{o}{.}\PY{n}{format}\PY{p}{(}\PY{n}{trial}\PY{p}{)}\PY{p}{)}
                 \PY{n}{current\PYZus{}img} \PY{o}{=} \PY{n}{unused\PYZus{}imgs}\PY{o}{.}\PY{n}{pop}\PY{p}{(}\PY{l+m+mi}{0}\PY{p}{)}
             
                 \PY{n}{cur\PYZus{}pts}\PY{p}{,} \PY{n}{pano\PYZus{}pts}\PY{p}{,}\PY{n}{\PYZus{}}\PY{p}{,}\PY{n}{\PYZus{}}\PY{p}{,}\PY{n}{\PYZus{}} \PY{o}{=} \PY{n}{genSIFTMatchPairs}\PY{p}{(}\PY{n}{current\PYZus{}img}\PY{p}{,} \PY{n}{panorama}\PY{p}{)}
             
                 \PY{k}{if} \PY{n+nb}{len}\PY{p}{(}\PY{n}{cur\PYZus{}pts}\PY{p}{)} \PY{o}{\PYZlt{}} \PY{l+m+mi}{4}\PY{p}{:}
                     \PY{n}{unused\PYZus{}imgs}\PY{o}{.}\PY{n}{append}\PY{p}{(}\PY{n}{current\PYZus{}img}\PY{p}{)}
                     \PY{n+nb}{print}\PY{p}{(}\PY{l+s+s1}{\PYZsq{}}\PY{l+s+s1}{Current image cannot be stitched. Would try later...}\PY{l+s+s1}{\PYZsq{}}\PY{p}{)}
                     \PY{k}{continue}
             
         
                 \PY{n}{\PYZus{}}\PY{p}{,} \PY{n}{H} \PY{o}{=} \PY{n}{RANSAC}\PY{p}{(}\PY{n}{cur\PYZus{}pts}\PY{p}{,} \PY{n}{pano\PYZus{}pts}\PY{p}{,} \PY{l+m+mi}{5000}\PY{p}{,} \PY{l+m+mi}{15}\PY{p}{)}
                     
                 \PY{n}{dst} \PY{o}{=} \PY{n}{backward\PYZus{}warp\PYZus{}img}\PY{p}{(}\PY{n}{current\PYZus{}img}\PY{p}{,} \PY{n}{np}\PY{o}{.}\PY{n}{linalg}\PY{o}{.}\PY{n}{inv}\PY{p}{(}\PY{n}{H}\PY{p}{)}\PY{p}{,} \PY{p}{[}\PY{n}{panorama}\PY{o}{.}\PY{n}{shape}\PY{p}{[}\PY{l+m+mi}{0}\PY{p}{]}\PY{p}{,} \PY{n}{panorama}\PY{o}{.}\PY{n}{shape}\PY{p}{[}\PY{l+m+mi}{1}\PY{p}{]}\PY{p}{]}\PY{p}{)}    
                 \PY{n}{panorama} \PY{o}{=} \PY{n}{blend\PYZus{}image\PYZus{}pair}\PY{p}{(}\PY{n}{panorama}\PY{p}{,} \PY{n}{binary\PYZus{}mask}\PY{p}{(}\PY{n}{panorama}\PY{p}{)}\PY{p}{,} \PY{n}{dst}\PY{p}{,} \PY{n}{binary\PYZus{}mask}\PY{p}{(}\PY{n}{dst}\PY{p}{)}\PY{p}{,} \PY{l+s+s1}{\PYZsq{}}\PY{l+s+s1}{blend}\PY{l+s+s1}{\PYZsq{}}\PY{p}{)}
                         
             \PY{n}{mask} \PY{o}{=} \PY{p}{(}\PY{n}{panorama}\PY{p}{[}\PY{p}{:}\PY{p}{,} \PY{p}{:}\PY{p}{,} \PY{l+m+mi}{0}\PY{p}{]} \PY{o}{\PYZgt{}} \PY{l+m+mi}{0}\PY{p}{)} \PY{o}{|} \PY{p}{(}\PY{n}{panorama}\PY{p}{[}\PY{p}{:}\PY{p}{,} \PY{p}{:}\PY{p}{,} \PY{l+m+mi}{1}\PY{p}{]} \PY{o}{\PYZgt{}} \PY{l+m+mi}{0}\PY{p}{)} \PY{o}{|} \PY{p}{(}\PY{n}{panorama}\PY{p}{[}\PY{p}{:}\PY{p}{,} \PY{p}{:}\PY{p}{,} \PY{l+m+mi}{2}\PY{p}{]} \PY{o}{\PYZgt{}} \PY{l+m+mi}{0}\PY{p}{)}
             \PY{n}{mask} \PY{o}{=} \PY{n}{np}\PY{o}{.}\PY{n}{sum}\PY{p}{(}\PY{n}{mask}\PY{p}{,} \PY{n}{axis} \PY{o}{=} \PY{l+m+mi}{0}\PY{p}{)}
             \PY{n}{mask} \PY{o}{=} \PY{n}{np}\PY{o}{.}\PY{n}{where}\PY{p}{(}\PY{n}{mask} \PY{o}{!=} \PY{l+m+mi}{0}\PY{p}{)}
         
             \PY{k}{return} \PY{n}{panorama}\PY{p}{[}\PY{p}{:}\PY{p}{,}\PY{n}{mask}\PY{p}{[}\PY{l+m+mi}{0}\PY{p}{]}\PY{p}{[}\PY{l+m+mi}{0}\PY{p}{]}\PY{p}{:}\PY{n}{mask}\PY{p}{[}\PY{l+m+mi}{0}\PY{p}{]}\PY{p}{[}\PY{o}{\PYZhy{}}\PY{l+m+mi}{1}\PY{p}{]}\PY{p}{,}\PY{p}{:}\PY{p}{]}      
             
\end{Verbatim}


    Use the below code to test your implementation. This code just reads in
the images, calls the stitch\_img() function, and plots the results.

    \begin{Verbatim}[commandchars=\\\{\}]
{\color{incolor}In [{\color{incolor}13}]:} \PY{n}{center\PYZus{}img} \PY{o}{=} \PY{n}{cv2}\PY{o}{.}\PY{n}{imread}\PY{p}{(}\PY{l+s+s2}{\PYZdq{}}\PY{l+s+s2}{mountain\PYZus{}center.png}\PY{l+s+s2}{\PYZdq{}}\PY{p}{)}
         \PY{n}{left\PYZus{}img} \PY{o}{=} \PY{n}{cv2}\PY{o}{.}\PY{n}{imread}\PY{p}{(}\PY{l+s+s2}{\PYZdq{}}\PY{l+s+s2}{mountain\PYZus{}left.png}\PY{l+s+s2}{\PYZdq{}}\PY{p}{)}
         \PY{n}{right\PYZus{}img} \PY{o}{=} \PY{n}{cv2}\PY{o}{.}\PY{n}{imread}\PY{p}{(}\PY{l+s+s2}{\PYZdq{}}\PY{l+s+s2}{mountain\PYZus{}right.png}\PY{l+s+s2}{\PYZdq{}}\PY{p}{)}
         
         \PY{n}{final\PYZus{}img} \PY{o}{=} \PY{n}{stitch\PYZus{}img}\PY{p}{(}\PY{p}{[}\PY{n}{center\PYZus{}img}\PY{p}{,} \PY{n}{left\PYZus{}img}\PY{p}{,} \PY{n}{right\PYZus{}img}\PY{p}{]}\PY{p}{)}
         
         \PY{n}{plt}\PY{o}{.}\PY{n}{imshow}\PY{p}{(}\PY{n}{cv2}\PY{o}{.}\PY{n}{cvtColor}\PY{p}{(}\PY{n}{final\PYZus{}img}\PY{o}{.}\PY{n}{astype}\PY{p}{(}\PY{l+s+s2}{\PYZdq{}}\PY{l+s+s2}{uint8}\PY{l+s+s2}{\PYZdq{}}\PY{p}{)}\PY{p}{,} \PY{n}{cv2}\PY{o}{.}\PY{n}{COLOR\PYZus{}BGR2RGB}\PY{p}{)}\PY{p}{)}\PY{p}{;}
\end{Verbatim}


    \begin{Verbatim}[commandchars=\\\{\}]
Stitch 1 images
Stitch 2 images

    \end{Verbatim}

    \begin{center}
    \adjustimage{max size={0.9\linewidth}{0.9\paperheight}}{output_26_1.png}
    \end{center}
    { \hspace*{\fill} \\}
    
    \subsection{Make Your Own Panarama}\label{make-your-own-panarama}

Use a digital camera, such as from your phone, and take three or more
photos to create your own panaroma. Remember to be stand in place and
only rotate the camera (Think about: why?). Include them in your
submission, and we will show the best ones during lecture.

    \begin{Verbatim}[commandchars=\\\{\}]
{\color{incolor}In [{\color{incolor}21}]:} \PY{n}{\PYZus{}1} \PY{o}{=} \PY{n}{cv2}\PY{o}{.}\PY{n}{imread}\PY{p}{(}\PY{l+s+s2}{\PYZdq{}}\PY{l+s+s2}{1.jpg}\PY{l+s+s2}{\PYZdq{}}\PY{p}{)}
         \PY{n}{\PYZus{}2} \PY{o}{=} \PY{n}{cv2}\PY{o}{.}\PY{n}{imread}\PY{p}{(}\PY{l+s+s2}{\PYZdq{}}\PY{l+s+s2}{2.jpg}\PY{l+s+s2}{\PYZdq{}}\PY{p}{)}
         \PY{n}{\PYZus{}3} \PY{o}{=} \PY{n}{cv2}\PY{o}{.}\PY{n}{imread}\PY{p}{(}\PY{l+s+s2}{\PYZdq{}}\PY{l+s+s2}{3.jpg}\PY{l+s+s2}{\PYZdq{}}\PY{p}{)}
         
         \PY{n}{img\PYZus{}list} \PY{o}{=} \PY{p}{[}\PY{n}{\PYZus{}2}\PY{p}{,} \PY{n}{\PYZus{}1}\PY{p}{,} \PY{n}{\PYZus{}3}\PY{p}{]}  
         \PY{l+s+sd}{\PYZsq{}\PYZsq{}\PYZsq{} TODO: Load your own images here and create a panorama. \PYZsq{}\PYZsq{}\PYZsq{}}
         
         \PY{n}{final\PYZus{}img} \PY{o}{=} \PY{n}{stitch\PYZus{}img}\PY{p}{(}\PY{n}{img\PYZus{}list}\PY{p}{)}
         
         \PY{n}{plt}\PY{o}{.}\PY{n}{imshow}\PY{p}{(}\PY{n}{cv2}\PY{o}{.}\PY{n}{cvtColor}\PY{p}{(}\PY{n}{final\PYZus{}img}\PY{o}{.}\PY{n}{astype}\PY{p}{(}\PY{l+s+s2}{\PYZdq{}}\PY{l+s+s2}{uint8}\PY{l+s+s2}{\PYZdq{}}\PY{p}{)}\PY{p}{,} \PY{n}{cv2}\PY{o}{.}\PY{n}{COLOR\PYZus{}BGR2RGB}\PY{p}{)}\PY{p}{)}\PY{p}{;}
\end{Verbatim}


    \begin{Verbatim}[commandchars=\\\{\}]
Stitch 1 images
Stitch 2 images

    \end{Verbatim}

    \begin{center}
    \adjustimage{max size={0.9\linewidth}{0.9\paperheight}}{output_28_1.png}
    \end{center}
    { \hspace*{\fill} \\}
    

    % Add a bibliography block to the postdoc
    
    
    
    \end{document}
